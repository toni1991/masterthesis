%\todo[inline,color=green]{Nicht von Kununu zitieren}
%\todo[inline,color=green]{TIOBE: vorbelastet, weil Hostentwickler Internet nicht als nr.1 qulle haben}
%\todo[inline,color=green]{Mein erstes COBOL-Programm -> IDE-Kapitel}
%\todo[inline,color=green]{S. 14 Präfixe (Identifier nicht mit Zahl)}
%\todo[inline,color=green]{Interviews: Umfang ansprechen und rausnemen + zitieren}
%\todo[inline,color=green]{S. 5 ``wichtige Feinheiten''}
%\todo[inline,color=green]{S. 4 ``gesamtheitlich'' falscher Begriff -> sehr detailliert }
%\todo[inline,color=green]{S. 5 ``repräsentativ'' -> keine Stichprobe / Fragenkatalog}

{\huge Todo}
\todo[inline,color=red]{Tests}
\todo[inline,color=red]{Generics beschreiben}
\todo[inline,color=red]{3. -Langlebigkeit\&Wartbarkeit -> Infrastruktur, nicht Sprache}
\todo[inline,color=red]{3. -Verlässlichkeit -> Infrastruktur, nicht Sprache}
\todo[inline,color=red]{3. Batchbetrieb -> Schnittstellen (Host nicht COBOL)}
\todo[inline,color=red]{5.3 -> 4. Deutlicher, dass Java-Klassen ``mächtiger'' sind}
\todo[inline,color=red]{4. -> viel als Schlüsselwort in COBOL, in Java nicht, sondern als Library}
\todo[inline,color=red]{Einleitungsüberschrift ändern}
\todo[inline,color=red]{Präzisere Formulierungen: Was ist häufig zu sehen und was selten? Was ist Beobachtung was ist Empfehlung?}
\todo[inline,color=red]{S. 28 (Java Komponenten) Initializer (\{\}) in Klassen}
\todo[inline,color=red]{Kap. 4 COBOL Arithmetische Ausdrücke: Stellen eingeschränkt und unüblich -> Rundung}
\todo[inline,color=red]{SetImplementierung in COBOL ineffizient und nicht generisch}
\todo[inline,color=red]{Bibkonzept in Java: 1. Erwähnen, dass es das generell gibt mit Bibs zu arbeiten 2. oo führt zu Algorithmen mit beliebigen Objekten, die eine bestimmte Eigenschaft erfüllen}
\todo[inline,color=red]{S. 2 Abs. 2 Wieso teuer und riskant? -> Erklärungen, Belege!}
\todo[inline,color=red]{Intention der Arbeit klar machen: Nicht Java-Entwickler umschulen, sondern schulen für Migration, Wartung und Flexibilität}
\todo[inline,color=red]{3.3 Storyline}
\todo[inline,color=red]{Rechengenauigkeit: unterschied zwischen binärem/dezimalem Fließ-/Festkomma => Geld -> Festkommasemantik}
\todo[inline,color=red]{Programmablauf ist gar nicht so unterschiedlich, sieht aber so aus. Nicht mit unterschieden einsteigen sondern mit gemeinsamkeiten!}
\todo[inline,color=red]{S. 18 Gliedern oder komprimieren: goto/lables Randnotizen in Java, Continue Unterschiede COBOL/Java, goto in COBOL kann unterschiedlich verwendet werden. Sprünge auf EXIT Paragraphen, da kein return.}

\subsection*{Am Ende prüfen}
\todo[inline,color=red]{Keine Vorwärtsreferenzen}
\todo[inline,color=red]{In der Literatur werden nicht alle Autoren aufgeführt}
\todo[inline,color=red]{Caption Abstände}
\todo[inline,color=red]{Seitenumbrüche + Paragraphenden ``schön''}
\todo[inline,color=red]{Overfull \& underfull + Steht etwas über den Rand?}
\todo[inline,color=red]{Keine TODOs mehr}