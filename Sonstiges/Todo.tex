\chapter*{TODOs}
\todo[inline]{S. 5 letzter Absatz liest sich nicht besonders klar.}

\todo[inline]{S. 15:  -> nicht jeder Anwendungsfall als Teil der Sprache implementiert werden kann -> Unter Anwendungsfall würde ich fachliche Logik verstehen - die kann ja kaum inbegriffen sein. Es geht hierbei eher um querschnittliche Funktionalität, ggf. auch kleinere fachliche Blöcke.}

\todo[inline]{S. 16  -> Wie in Abschnitt 3.2 beschrieben, findet man diesen Copy-Paste-Stil wegen fehlender oder teurer Möglichkeiten zu testen. -> Das müsstest du genauer erläutern (ist ein wichtiger Grund, aber nicht trivial zu erkennen. Wenn man ein Bibliothekskonzept hätte dann könnte man ja auch eine getestete Funktionalität aus der Bibliothek beziehen. Da spielt also das fehlende Bibliothekskonzept eher eine Rolle als das Testen. Das Thema Testen kommt m.E.rein wenn es darum geht existierende Funktionalität wiederzuverwenden. Wenn ich schwer testen kann dann ist das Herausschneiden und Kapseln gefährlich für die vorhandenen Use Cases. Daher lasse ich den Code dort lieber wie er ist und kopiere ihn.}

\todo[inline,color=red]{Möglichst wenig Vorwärtsreferenzen (Abschnitte evtl. anders anordnen)}
\todo[inline,color=red]{Füllsel und unsichere Formulierungen entfernen}
\todo[inline,color=red]{Seitenumbrüche + Paragraphenden ``schön''}
\todo[inline,color=red]{Overfull \& underfull + Steht etwas über den Rand? + Gedankenstriche am Ende einer Zeile}
\todo[inline,color=red]{Richtige Worttrennungen}
\todo[inline,color=red]{Keine TODOs mehr}