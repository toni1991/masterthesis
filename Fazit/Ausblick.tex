\section{Ausblick}
Durch den Wandel der letzten Jahre \bzw Jahrzehnte in komplexen Softwaresystemen wird es immer schwerer, reine COBOL-Systeme zu nutzen und sinnvoll in eine heterogene IT-Landschaft einzubetten. Kunden sind von Applikationen für verschiedenste Systeme und Plattformen abhängig und müssen sich darauf verlassen können, dass die Erweiterung um neue Back- und Frontends ohne Einfluss auf bestehende Komponenten vonstattengehen kann. Auch ist es nötig, die Basis für interne Systemergänzungen zu schaffen, indem festgelegte Interfaces bedient, verwendet und angeboten werden. Während diese Anforderungen mit modernen Sprachen wie Java mühelos erreicht werden können, bieten Altsysteme nur spärliche Möglichkeiten in dieser Richtung.

Aber auch finanzielle Aspekte sorgen über kurz oder lang dafür, dass Unternehmen Altsysteme migrieren und Geschäftslogik zunehmend in neuen Technologien abbilden. Dieser Wandel ist jedoch auch dafür verantwortlich, dass in Zukunft weiterhin Entwickler mit COBOL-Know-how gefragt sein werden, um schlecht dokumentierte Altsysteme mit all ihren Zusammenhängen zu verstehen und in neuen Code zu überführen.

\subsection*{Verknüpfung von COBOL und Java} \label{cobolandjava}

Abschließend bleibt zu sagen, dass auch der gleichzeitige Betrieb von COBOL und Java in der Praxis möglich ist. Das verspricht kleinere Einheiten von Programmcode, die migriert und in Java neu implementiert oder weiterverwendet werden können.

\mintedCobolNotfree{converted.cbl}{Ursprünglicher COBOL-Code}{convertedCobol}
\mintedJava{converted.java}{Generierter Java-Code}{convertedJava}

Herr Lamperstorfer wies darauf hin, dass es dazu die Möglichkeit gibt, COBOL-Code innerhalb der JVM laufen zu lassen. Man spricht auch von \textit{Rehosting}. Dazu wird ein spezieller Compiler benötigt, der den COBOL- in Byte-Code der JVM übersetzt. Damit wird es möglich, COBOL-Module direkt aus Java-Code heraus zu nutzen. Außerdem existieren Werkzeuge, die COBOL-Code in Java-Code übersetzen, sodass dieser anschließend -- im Gegensatz zu generiertem Byte-Code -- weiter bearbeitet werden kann. \autoref{convertedCobol}\footnote{\label{convertedFootnote} Dieser Code ist Teil einer Beispieldatei (Zeile 51-56) des NACA-Projekts, das sich auf die Übersetzung von COBOL nach Java spezialisiert hat.\\\url{https://github.com/charleso/naca/blob/master/NacaSamples/src/online/ONLINE1.java} \visitedOn} und \autoref{convertedJava}\footref{convertedFootnote} zeigen allerdings, dass der generierte Java-Code in diesem Fall nicht zur weiteren Verwendung zu gebrauchen ist.

Mit dem \textit{Java Native Interface}, kurz \textit{JNI} genannt, ergibt sich eine weitere Möglichkeit COBOL und Java zu verknüpfen. Dabei handelt es sich um eine Schnittstelle, die es Java-Programmen ermöglicht kompilierte Bibliotheken zu laden und Programme aufzurufen, womit kompilierter COBOL-Code ausgeführt werden kann.

Auf \quotes{feingranularer Ebene} funktionieren diese Verbindungen, laut Herr Streit, nicht sehr gut, da \quotes{die Aufrufe schlecht lesbar sind}, \quotes{mehr Boilerplate Code nötig ist als bei einem Aufruf innerhalb der Sprache} und die Performanz weniger gut sei. Allerdings eigne sich eine solche Technologie gut, um große Einheiten in verschiedenen Sprachen zu verknüpfen. So lasse sich mit \quotes{klar definiert[en] und schmal[en]} Schnittstellen eine gute Basis zur \quotes{schrittweisen Migration} schaffen.