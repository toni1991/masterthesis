\section{Ausblick}

Durch den Wandel der letzten Jahre \bzw Jahrzehnte in komplexen Softwaresystemen wird es immer schwerer, reine COBOL-Systeme zu nutzen und sinnvoll in eine heterogene IT-Landschaft einzubetten. Kunden sind von Applikationen für verschiedenste Systeme und Plattformen abhängig und müssen sich darauf verlassen können, dass die Erweiterung um neue Back- und Frontends ohne Einfluss auf bestehende Komponenten vonstattengehen kann. Auch ist es nötig, die Basis für interne Systemergänzungen zu schaffen, indem festgelegte Interfaces bedient, verwendet und angeboten werden. Während diese Anforderungen mit modernen Sprachen wie Java mühelos erreicht werden können, bieten Altsysteme nur spärliche Möglichkeiten in dieser Richtung.

Aber auch finanzielle Aspekte sorgen über kurz oder lang dafür, dass Unternehmen Altsysteme migrieren und Geschäftslogik zunehmend in neuen Technologien abbilden. Dieser Wandel ist jedoch auch dafür verantwortlich, dass in Zukunft weiterhin Entwickler mit COBOL-Know-how gefragt sein werden, um schlecht dokumentierte Altsysteme mit all ihren Zusammenhängen zu verstehen und in neuen Code zu überführen.