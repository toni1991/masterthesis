\section{Fazit}

Zusammenfassend kann festgehalten werden, dass sich die Sprachen COBOL und Java auf den ersten Blick stärker unterscheiden, als dies in Wahrheit der Fall ist. Viele bekannte Konzepte lassen sich wiederfinden und Parallelen zwischen unterschiedlich wirkenden Konstrukten ziehen. 

Nichtsdestotrotz ist auch klar erkennbar, dass COBOL so manches vermissen lässt, was von neueren Sprachen wie Java zur Verfügung gestellt wird und an einigen Stellen andere Ansätze verfolgt. Dies gilt auch für die Werkzeuge und Umgebungen für die jeweiligen Sprachen. Dadurch wird die Analyse von Fehlern und das Erweitern solcher Systeme \idR komplexer als bei Systemen, die auf neuen Technologien basieren. Eine konsequente Einhaltung einiger Richtlinien sorgt jedoch dafür, dass COBOL leichter les- und wartbar wird. 

Selbst für erfahrene Entwickler können daher erste Kontakte mit COBOL-Systemen befremdlich wirken. Doch mit Wissen über häufig beobachtbare Muster, ungewohnte Stolpersteine und nicht direkt erkennbare Analogien zwischen COBOL-Konzepten und denen bekannter Sprachen, ist die Einarbeitung problemlos möglich.

\section{Ausblick}

Durch den Wandel der letzten Jahre \bzw Jahrzehnte in komplexen Softwaresystemen wird es immer schwerer, reine COBOL-Systeme zu nutzen und sinnvoll in eine heterogene IT-Landschaft einzubetten. Kunden sind von Applikationen für verschiedenste Systeme und Plattformen abhängig und müssen sich darauf verlassen können, dass die Erweiterung um neue Back- und Frontends ohne Einfluss auf bestehende Komponenten vonstattengehen kann. Auch ist es nötig, die Basis für interne Systemergänzungen zu schaffen, indem festgelegte Interfaces bedient, verwendet und angeboten werden. Während diese Anforderungen mit modernen Sprachen wie Java mühelos erreicht werden können, bieten Altsysteme nur spärliche Möglichkeiten in dieser Richtung.

Aber auch finanzielle Aspekte sorgen über kurz oder lang dafür, dass Unternehmen Altsysteme migrieren und Geschäftslogik zunehmend in neuen Technologien abbilden. Dieser Wandel ist jedoch auch dafür verantwortlich, dass in Zukunft weiterhin Entwickler mit COBOL-Know-how gefragt sein werden, um schlecht dokumentierte Altsysteme mit all ihren Zusammenhängen zu verstehen und in neuen Code zu überführen.