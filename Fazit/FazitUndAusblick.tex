\section{Fazit und Ausblick}

Durch den Wandel der letzten Jahre \bzw Jahrzehnte in komplexen Softwaresystemen wird es immer schwerer, reine COBOL-Systeme zu nutzen und sinnvoll in eine heterogene IT-Landschaft einzubetten. Kunden sind von Applikationen für verschiedenste Systeme abhängig und müssen sich darauf verlassen können, dass die Erweiterung um neue Back- und Frontends ohne Einfluss auf bestehende Komponenten vonstatten gehen kann. Auch ist es nötig, die Basis für interne Systemergänzungen zu schaffen, indem festgelegte Interfaces bedient und verwendet werden. Diese Anforderungen können mit modernen Sprachen wie Java mühelos erreicht werden, wohingegen Altsysteme nur spärliche Möglichkeiten in dieser Richtung bieten.


\mytodo{\todo[inline]{Verweis auf \autoref{cobolandjava}}}

\mytodo{\todo[inline]{Systeme sind teuer}}