\section{Fazit}

Zusammenfassend kann festgehalten werden, dass sich die Sprachen COBOL und Java auf den ersten Blick stärker unterscheiden, als dies in Wahrheit der Fall ist. Viele bekannte Konzepte lassen sich wiederfinden und Parallelen zwischen unterschiedlich wirkenden Konstrukten ziehen. 

Nichtsdestotrotz ist auch klar erkennbar, dass COBOL so manches vermissen lässt, was von neueren Sprachen wie Java zur Verfügung gestellt wird und an einigen Stellen andere Ansätze verfolgt. Dies gilt auch für die Werkzeuge und Umgebungen für die jeweiligen Sprachen. Dadurch wird die Analyse von Fehlern und das Erweitern solcher Systeme \idR komplexer als bei Systemen, die auf neuen Technologien basieren. Eine konsequente Einhaltung einiger Richtlinien sorgt jedoch dafür, dass COBOL leichter les- und wartbar wird. 

Selbst für erfahrene Entwickler können daher erste Kontakte mit COBOL-Systemen befremdlich wirken. Doch mit Wissen über häufig beobachtbare Muster, ungewohnte Stolpersteine und nicht direkt erkennbare Analogien zwischen COBOL-Konzepten und denen bekannter Sprachen, ist die Einarbeitung problemlos möglich.