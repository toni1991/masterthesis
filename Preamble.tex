\documentclass[listof=totoc]{scrartcl}

%% Language and font encodings
\usepackage[english,ngerman]{babel}
\usepackage[utf8]{inputenc}
\usepackage[T1]{fontenc}
\usepackage{textcomp}
\usepackage{color}
\usepackage{pdflscape}
\usepackage{rotating}
%\usepackage{tocbibind}

\definecolor{pblue}{rgb}{0.13,0.13,1}
\definecolor{pgreen}{rgb}{0,0.5,0}
\definecolor{pred}{rgb}{0.9,0,0}
\definecolor{pgrey}{rgb}{0.46,0.45,0.48}


%% Sets page size and margins
\usepackage[includeheadfoot]{geometry}

\usepackage{listings}
\newcommand{\includecode}[2][Java]{
	\subsection{#2}
	\lstinputlisting[caption=#2]{#2}
	\pagebreak
}
\lstset{language=Java,
	belowcaptionskip=1\baselineskip,
	frame=single,                  % adds a frame around the code   
	rulecolor=\color{black},
	numbers=left,
	numberstyle=\tiny,
	captionpos=b,             % sets the caption-position to bottom    
	showspaces=false,
	showtabs=false,
	breaklines=true,
	showstringspaces=false,
	breakatwhitespace=true,
	commentstyle=\color{pgreen},
	keywordstyle=\color{pblue},
	stringstyle=\color{pred},
	basicstyle=\ttfamily,
	moredelim=[il][\textcolor{pgrey}]{$$},
	moredelim=[is][\textcolor{pgrey}]{\%\%}{\%\%},
	inputpath=listings
}

%% cite
\usepackage[backend=biber]{biblatex}
\addbibresource{Masterarbeit.bib}

%% Useful packages
\usepackage{amsmath}
\usepackage{multicol}
\usepackage{multirow}
\usepackage{graphicx}
\usepackage[colorlinks=true, allcolors=black]{hyperref}
\usepackage{todonotes}
\usepackage{subcaption}
\usepackage{float}
\usepackage{csquotes}
\usepackage{framed}
\usepackage{wrapfig}
\usepackage{listings}
\usepackage{tabularx}
\newcolumntype{C}{>{\centering\arraybackslash}X}
\usepackage{pdfpages}
\usepackage{enumitem}

\makeatletter
\newcommand*{\toccontents}{\@starttoc{toc}}
\makeatother

\title{\vspace{10ex}Masterarbeit \\ Programmiersprachliche Konzepte von COBOL im Vergleich mit Java -- Eine praxisorientierte Einführung}
\subtitle{\vspace{5ex}Besprechung 13.11.2017}
\author{\vspace{-5ex}}
\date{\vspace{-5ex}}

\newcommand{\includesection}[1]{\pagebreak\section{#1}\input{#1}}