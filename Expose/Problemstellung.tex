\section{Problemstellung}

``Viele Millionen Cobol-Programme existieren weltweit und müssen laufend gepflegt werden.
Es ist bei dieser Situation undenkbar und unter wirtschaftlichen Gesichtspunkten unvertretbar in den nächsten Jahren eine Umstellung dieser Programme auf eine andere Sprache durchzuführen.'' \footnote{Dr. Horst Strunz, Fachbereichsleiter Standard-Systeme und Training des mbp, Mathematischer Beratungs- und Programmierungsdienst GmbH, Dortmund in \textit{Ist Cobol die Programmiersprache der Zukunft?}. \\ https://www.computerwoche.de/a/ist-cobol-die-programmiersprache-der-zukunft,1191656}

Was Herr Dr. Strunz neben vielen Anderen Experten bereits 1979 prophezeite hat auch heute noch Gültigkeit. Obwohl COBOL zum Ende der 50er Jahre entstand, 1959 veröffentlicht wurde und damit fast 60 Jahre alt ist, trifft man es auch heute noch häufig an. In der britischen Tageszeitung The Guardian, zitiert der Autor Scott Colvey in seinem Artikel ``Cobol hits 50 and keeps counting'' anlässlich des 50. Geburtstages von COBOL den Micro Focus Manager David Stephenson: `` `some 70\% to 80\% of UK plc business transactions are still based on Cobol' ''.
\footnote{https://www.theguardian.com/technology/2009/apr/09/cobol-internet-programming}

\todo{Prozent"=zahlen,Beleg}

\\

\todo[inline]{Wie viele Unis bieten COBOL an?}

\todo[inline]{TIOBE-Index https://www.tiobe.com/tiobe-index/}