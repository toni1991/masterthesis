\section{Problemstellung}
Der folgende Abschnitt soll die Problemstellung verdeutlichen, welche der Arbeit zu Grunde liegt. 
Dazu wird erläutert, welche Wichtigkeit COBOL genießt und anschließend mit der Bedeutung, die der Sprache tatsächlich beigemessen wird, gegenübergestellt.

\subsection{Wichtigkeit von COBOL}
``Viele Millionen Cobol-Programme existieren weltweit und müssen laufend gepflegt werden.
Es ist bei dieser Situation undenkbar und unter wirtschaftlichen Gesichtspunkten unvertretbar in den nächsten Jahren eine Umstellung dieser Programme auf eine andere Sprache durchzuführen.'' \footnote{Dr. Horst Strunz, Fachbereichsleiter Standard-Systeme und Training des mbp, Mathematischer Beratungs- und Programmierungsdienst GmbH, Dortmund in \textit{Ist Cobol die Programmiersprache der Zukunft?}. \\ https://www.computerwoche.de/a/ist-cobol-die-programmiersprache-der-zukunft,1191656}

Was Herr Dr. Strunz neben vielen Anderen Experten bereits 1979 prophezeite hat auch heute noch Gültigkeit. Obwohl COBOL zum Ende der 50er Jahre entstand, 1959 veröffentlicht wurde und damit fast 60 Jahre alt ist, trifft man es auch heute noch häufig an. In der britischen Tageszeitung The Guardian, zitiert der Autor Scott Colvey in seinem Artikel ``\citefield{colvey_cobol_2009}{title}'' \cite{colvey_cobol_2009} anlässlich des 50. Geburtstages von COBOL den Micro Focus Manager David Stephenson: `` `some 70\% to 80\% of UK plc business transactions are still based on Cobol' ''. 

Stephen Kelly, Geschäftsführer von Micro Focus, 
Weiter führt er darin Aussagen von IBM Software-Leiters Charles Chu an, welcher die Aussagen von Stephenson bestätigt: ``$[\ldots]$ there are 250bn lines of Cobol code working well worldwide. Why would companies replace systems that are working well?' ''. 
Nicht nur dass COBOL damit geschätzte X\% \todo{Gesamt"=zahl} des Marktes ausmacht wird daran deutlich, sondern auch die Bedeutung für die Zukunft: Wieso sollte funktionierender Code mit Hilfe von teuren und riskanten Prozessen ersetzt werden?

Da sich viele Unternehmen dieser Frage ausgesetzt sehen, auf die sich nur schwer eine Antwort finden lässt, welche die Risiken und Kosten aufwiegt, stieg die Anzahl \todo{Beleg} des weltweit in Produktion befindlichen COBOL-Codes über die vergangen Jahre sogar noch weiter an.

\todo[inline]{GOOGLE Anfragen vergleich Stackexchange}

\subsection{Bedeutung in Wirtschaft und Lehre}

Im TIOBE-Index\footnote{\url{https://www.tiobe.com/tiobe-index}} für Dezember 2017 rangiert COBOL derzeit auf Platz 29 mit einem Rating von 0.961\%. Dieser Index wir auf Basis von Suchanfragen nach den entsprechenden Programmiersprachen, auf den meistfrequentiertesten Internetseiten, erstellt. COBOL ist somit Teil von weniger als 1\% der Suchanfragen.

\todo[inline]{Wie viele Unis bieten COBOL an?}

\todo[inline]{Rente von COBOL-Entwicklern}