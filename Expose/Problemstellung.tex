\section{Problemstellung}

``Viele Millionen Cobol-Programme existieren weltweit und müssen laufend gepflegt werden.
Es ist bei dieser Situation undenkbar und unter wirtschaftlichen Gesichtspunkten unvertretbar in den nächsten Jahren eine Umstellung dieser Programme auf eine andere Sprache durchzuführen.'' \footnote{Dr. Horst Strunz, Fachbereichsleiter Standard-Systeme und Training des mbp, Mathematischer Beratungs- und Programmierungsdienst GmbH, Dortmund in \textit{Ist Cobol die Programmiersprache der Zukunft?}. \\ https://www.computerwoche.de/a/ist-cobol-die-programmiersprache-der-zukunft,1191656}

Was Herr Dr. Strunz neben vielen Anderen Experten bereits 1979 prophezeite hat auch heute noch Gültigkeit. Obwohl COBOL im Laufe der 50er Jahre entstand und damit über 60 Jahre alt ist, trifft man es vor allem im Bankensektor noch häufig an. \todo{Prozent"=zahlen,Beleg}

\\

\todo[inline]{Wie viele Unis bieten COBOL an?}

\todo[inline]{Wie viele Unis bieten COBOL an?}