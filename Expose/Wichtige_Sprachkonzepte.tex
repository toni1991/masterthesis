\section{Wichtige Sprachkonzepte}

\begin{table}[H]
\centering
\begin{tabularx}{\textwidth}{|C|C|}
\hline
\textbf{\large{COBOL}} & \textbf{\large{Java}} \\\hline
PROCEDURE DIVISION & main-Methode \\\hline
PERFORM & Funktionen \\\hline
DATA DIVISION & Variablendeklaration \\\hline
Datentypen & Primitive Datentypen \\\hline
OCCURES & Arrays \\\hline
& \makecell{\large{Datenstrukturen}\\List\\Set\\Map}\\\hline
Level Number 66 & \\\hline
Level Number 88 & \textit{onChangeListener, Special getter \& setter} \\\hline
COPY & import\\\hline
Schleifen
& \makecell{\large{OO-Konzepte}\\static / member\\visibility modifier}\\\hline
\end{tabularx}
\end{table}

\pagebreak

\subsection{COBOL Stufennummer 88}
Neben den bereits angesprochenen Stufennummern stellt die \textit{88} eine weitere Besonderheit dar. Mit ihr ist es möglich einer Variable einen Wahrheitswert zuzuweisen, der von einem anderen Variablenwert abhängt. 

\begin{listing}[H]
  \inputminted[bgcolor=mintedgrey,xleftmargin=20pt,linenos,fontsize=\footnotesize]{cobol}{code/88_section.cbl.txt}
  \caption{Beispiel für COBOL Stufennummer 88}
  \label{88_cobol_listing}
\end{listing} 

\autoref{88_cobol_listing} zeigt die beispielhafte Verwendung der Stufennummer 88. Die Variable \mintinline{text}{VAR} kann dabei zweistellige numerische Werte enthalten. Liegt der Wert zwischen 0 und 9 (\mintinline{cobol}{VALUE 0 THRU 9}) so weist die Variable \mintinline{text}{ISLOWERTEN} einen den Wahrheitswert \mintinline{cobol}{TRUE} auf.

Der so entstandene Wahrheitswert kann folglich immer dann verwendet werden, wenn getestet werden soll, ob die Variable \mintinline{text}{VAR} im Bereich zwischen 0 und 9 liegt.

Zeile 18 des Programms zeigt einen weiteren Anwendungsfall der Stufennummer 88. So lässt sich der Wert der eigentlichen Variable setzen, indem der bedingten Variable der Wahrheitswerte \mintinline{cobol}{TRUE} zugewiesen wird.

Die Ausgabe des Programms in \autoref{88_cobol_listing} wäre folgende:
\begin{minted}[bgcolor=mintedgrey,xleftmargin=20pt,fontsize=\footnotesize]{text}
VAR is over 10.
VAR = 00
\end{minted}

\subsubsection*{Abbildung in Java}
Java besitzt kein Sprachkonstrukt, um die Funktionalität der Stufennummer 88 nachzubilden. Eine Möglichkeit gleiches Verhalten darzustellen bietet allerdings die Implementierung spezieller getter- und setter-Methoden. Dies soll \autoref{88_java_listing} veranschaulichen.

\begin{listing}[H]
  \inputminted[bgcolor=mintedgrey,xleftmargin=20pt,linenos,fontsize=\footnotesize]{java}{code/88_section.java.txt}
  \caption{COBOL Stufennummer 88 in Java}
  \label{88_java_listing}
\end{listing} 

Die Methoden \mintinline{text}{getISLOWERTEN} bezieht sich dabei nicht auf eine Variable sondern gibt einen Wahrheitswert in Abhängigkeit des Variablenwertes zurück. \mintinline{text}{setISLOWERTEN} setzt diesen Variablenwert wenn der Wahrheitswert \mintinline{java}{true} ist. Diese Einschränkung in \mintinline{text}{setISLOWERTEN} wurde hinzugefügt, um das Verhalten der meisten COBOL-Compiler nachzubilden, welche nur das Setzen des Wertes \mintinline{cobol}{TRUE} erlauben.