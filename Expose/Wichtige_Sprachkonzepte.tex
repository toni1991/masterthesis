\section{Wichtige Sprachkonzepte}

\begin{table}[H]
\centering
\begin{tabularx}{\textwidth}{|C|C|}
\hline
\textbf{\large{COBOL}} & \textbf{\large{Java}} \\\hline
PROCEDURE DIVISION & main-Methode \\\hline
PERFORM & Funktionen \\\hline
DATA DIVISION & Variablendeklaration \\\hline
Datentypen & Primitive Datentypen \\\hline
OCCURES & Arrays \\\hline
& \makecell{\large{Datenstrukturen}\\List\\Set\\Map}\\\hline
Level Number 66 & \\\hline
Level Number 88 & \textit{onChangeListener, Special getter \& setter} \\\hline
COPY & import\\\hline
Schleifen
& \makecell{\large{OO-Konzepte}\\static / member\\visibility modifier}\\\hline
\end{tabularx}
\caption{My caption}
\label{my-label}
\end{table}

\subsection*{COBOL Stufennummer 88}
Neben den bereits angesprochenen Stufennummern stellt die \textit{88} eine weitere Besonderheit dar. Mit ihr ist es möglich einer Variable einen Wahrheitswert zuzuweisen, der von einem anderen Variablenwert abhängt. 

\inputminted[]{cobol}{code/88_section.cbl}