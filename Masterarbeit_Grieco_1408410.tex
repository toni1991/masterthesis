% arara: pdflatex: { shell: on }
% arara: biber
% arara: pdflatex: { shell: on }
% arara: pdflatex: { shell: on }
%% arara: biber
%% arara: pdflatex: { shell: on }
%% arara: pdflatex: { shell: on }
% arara: clean: { files: [ Masterarbeit_Grieco_1408410.aux, Masterarbeit_Grieco_1408410.bbl, Masterarbeit_Grieco_1408410.bcf, Masterarbeit_Grieco_1408410.cod, Masterarbeit_Grieco_1408410.blg, Masterarbeit_Grieco_1408410.lof, Masterarbeit_Grieco_1408410.lot, Masterarbeit_Grieco_1408410.out, Masterarbeit_Grieco_1408410.toc, Masterarbeit_Grieco_1408410.log, Masterarbeit_Grieco_1408410.lol, Masterarbeit_Grieco_1408410.run.xml, _minted-Masterarbeit_Grieco_1408410 ] } 

% ------------------------------------------------------------------------------
% Formatvorlage f?r Masterarbeiten an der Ohm-Hochschule N?rnberg
% ------------------------------------------------------------------------------
%   erstellt von Stefan Macke, 24.04.2009
%   http://blog.stefan-macke.de
% Dokumentenkopf ---------------------------------------------------------------
%   Diese Vorlage basiert auf "scrreprt" aus dem koma-script.
% ------------------------------------------------------------------------------

\documentclass[
    12pt, % Schriftgröße
    DIV=10,
    ngerman, % für Umlaute, Silbentrennung etc.
    a4paper, % Papierformat
    oneside, % einseitiges Dokument
    titlepage, % es wird eine Titelseite verwendet
    parskip=full, % Abstand zwischen Absätzen (halbe Zeile)
    headings=normal, % Größe der Überschriften verkleinern
    listof=totoc, % Verzeichnisse im Inhaltsverzeichnis aufführen
    listof=entryprefix,
    bibliography=totoc, % Literaturverzeichnis im Inhaltsverzeichnis aufführen
    index=totoc, % Index im Inhaltsverzeichnis aufführen
    captions=tableheading, % Beschriftung von Tabellen unterhalb ausgeben
    final % Status des Dokuments (final/draft)
]{scrreprt}

% Weise compiler an, nicht bei Fehlern anzuhalten!
% \nonstopmode

% Meta-Informationen -----------------------------------------------------------
%   Informationen über das Dokument, wie z.B. Titel, Autor, Matrikelnr. etc
%   werden in der Datei Meta.tex definiert und können danach global
%   verwendet werden.
% ------------------------------------------------------------------------------
% Meta-Informationen -----------------------------------------------------------
%   Definition von globalen Parametern, die im gesamten Dokument verwendet
%   werden k�nnen (z.B auf dem Deckblatt etc.).
%
%   ACHTUNG: Wenn die Texte Umlaute oder ein Esszet enthalten, muss der folgende
%            Befehl bereits an dieser Stelle aktiviert werden:
%            \usepackage[latin1]{inputenc}
% ------------------------------------------------------------------------------

\newcommand{\titel}{Mobile ereignisbasierte \\Anlagen�berwachung in der Industrie 4.0}
\newcommand{\art}{Bachelorarbeit}
\newcommand{\fachgebiet}{Informatik}
\newcommand{\autor}{Antonio Grieco}
\newcommand{\studienbereich}{Software-Engineering}
\newcommand{\matrikelnr}{930 190}
\newcommand{\erstgutachter}{Prof. Dr.-Ing. Thorsten Sch�ler}
\newcommand{\zweitgutachter}{Dipl.-Inf. Lukas Podolski}
\newcommand{\jahr}{2015}
\newcommand{\ort}{Augsburg}

% Befehle, die Umlaute ausgeben, f�hren zu Fehlern, wenn sie hyperref als Optionen �bergeben werden
\hypersetup{
    pdftitle={\titel},
    pdfauthor={\autor},
    pdfcreator={\autor},
    pdfsubject={\titel},
    pdfkeywords={\titel}
}

% benötigte Packages -----------------------------------------------------------
%   LaTeX-Packages, die benötigt werden, sind in die Datei Packages.tex
%   "ausgelagert", um diese Vorlage möglichst übersichtlich zu halten.
% ------------------------------------------------------------------------------
% Anpassung des Seitenlayouts --------------------------------------------------
%   siehe Seitenstil.tex
% ------------------------------------------------------------------------------
\usepackage[
    automark, % Kapitelangaben in Kopfzeile automatisch erstellen
    headsepline, % Trennlinie unter Kopfzeile
    ilines % Trennlinie linksbündig ausrichten
]{scrpage2}

% Anpassung an Landessprache ---------------------------------------------------
\usepackage[ngerman]{babel}

% Umlaute ----------------------------------------------------------------------
%   Umlaute/Sonderzeichen wie äüöß direkt im Quelltext verwenden (CodePage).
%   Erlaubt automatische Trennung von Worten mit Umlauten.
% ------------------------------------------------------------------------------
\usepackage[utf8]{inputenc}
\usepackage[T1]{fontenc}
\usepackage{textcomp} % Euro-Zeichen etc.

% Schrift ----------------------------------------------------------------------
\usepackage{lmodern} % bessere Fonts
\usepackage{relsize} % Schriftgröße relativ festlegen

% Grafiken ---------------------------------------------------------------------
% Einbinden von JPG-Grafiken ermöglichen
\usepackage[dvips,final]{graphicx}
% hier liegen die Bilder des Dokuments
\graphicspath{{Bilder/}}

% Befehle aus AMSTeX für mathematische Symbole z.B. \boldsymbol \mathbb --------
\usepackage{amsmath,amsfonts}

% für Index-Ausgabe mit \printindex --------------------------------------------
\usepackage{makeidx}

% Einfache Definition der Zeilenabstände und Seitenränder etc. -----------------
\usepackage{setspace}
\usepackage{geometry}

% Symbolverzeichnis ------------------------------------------------------------
%   Symbolverzeichnisse bequem erstellen. Beruht auf MakeIndex:
%     makeindex.exe %Name%.nlo -s nomencl.ist -o %Name%.nls
%   erzeugt dann das Verzeichnis. Dieser Befehl kann z.B. im TeXnicCenter
%   als Postprozessor eingetragen werden, damit er nicht ständig manuell
%   ausgeführt werden muss.
%   Die Definitionen sind ausgegliedert in die Datei "Glossar.tex".
% ------------------------------------------------------------------------------
% \usepackage[intoc]{nomencl}
% \let\abbrev\nomenclature
% \renewcommand{\nomname}{Abkürzungsverzeichnis}
% \setlength{\nomlabelwidth}{.25\hsize}
% \renewcommand{\nomlabel}[1]{#1 \dotfill}
% \setlength{\nomitemsep}{-\parsep}

% Abkürzungsverzeichnis
\usepackage[printonlyused]{acronym} %Abkürzungsverzeichnis erstellen, Langtext als Fußnote, Auflistung nur bei Verwendung

% zum Umfließen von Bildern ----------------------------------------------------
\usepackage{floatflt}
\usepackage{float}

% Abstand der float caption
\usepackage[skip=5pt]{caption} % example skip set to 2pt

% ensure floats do not go into the next section.
\usepackage[section]{placeins}

% Farben
\usepackage[dvipsnames]{xcolor}
\definecolor{hellgrau}{rgb}{0.93,0.93,0.93}
\definecolor{colKeys}{rgb}{0,0,0.9}
\definecolor{colIdentifier}{rgb}{0,0,0}
\definecolor{colComments}{rgb}{0.6,0,0}
\definecolor{colString}{rgb}{0,0.7,0}

% zum Einbinden von Programmcode -----------------------------------------------
\usepackage{listings}
\usepackage[newfloat]{minted}

\setminted{
    bgcolor=hellgrau,
    xleftmargin=20pt,
    linenos=true,
    fontsize=\footnotesize
}

\lstset{
    float=hbp,
    basicstyle=\ttfamily\color{black}\small\smaller,
    keywordstyle=\color{colKeys},
    stringstyle=\color{colString},
    commentstyle=\color{colComments},
    columns=flexible,
    tabsize=4,
    frame=single,
    extendedchars=true,
    showspaces=false,
    showstringspaces=false,
    numbers=left,
    numberstyle=\tiny,
    breaklines=true,
    backgroundcolor=\color{hellgrau},
    captionpos=b,
    breakautoindent=true
}

% URL verlinken, lange URLs umbrechen etc. -------------------------------------
\usepackage{url}
\makeatletter
\g@addto@macro{\UrlBreaks}{\UrlOrds}
\makeatother

% PDF-Optionen -----------------------------------------------------------------
\usepackage[
    bookmarks,
    bookmarksopen=true,
    colorlinks=true,
% diese Farbdefinitionen zeichnen Links im PDF farblich aus
    linkcolor=black, % einfache interne Verknüpfungen
    anchorcolor=black,% Ankertext
    citecolor=black, % Verweise auf Literaturverzeichniseinträge im Text
    filecolor=black, % Verknüpfungen, die lokale Dateien öffnen
    menucolor=black, % Acrobat-Menüpunkte
    urlcolor=black, 
% diese Farbdefinitionen sollten für den Druck verwendet werden (alles schwarz)
    %linkcolor=black, % einfache interne Verknüpfungen
    %anchorcolor=black, % Ankertext
    %citecolor=black, % Verweise auf Literaturverzeichniseinträge im Text
    %filecolor=black, % Verknüpfungen, die lokale Dateien öffnen
    %menucolor=black, % Acrobat-Menüpunkte
    %urlcolor=black,
    plainpages=false, % zur korrekten Erstellung der Bookmarks
    pdfpagelabels, % zur korrekten Erstellung der Bookmarks
    hypertexnames=false, % zur korrekten Erstellung der Bookmarks
    linktoc=all, % Seitenzahlen und Text im Inhaltsverzeichnis verlinken
]{hyperref}

% Befehle, die Umlaute ausgeben, führen zu Fehlern, wenn sie hyperref als Optionen übergeben werden
\hypersetup{
    pdftitle={\mytitel},
    pdfauthor={\myautor},
    pdfcreator={\myautor},
    pdfsubject={\mytitel},
    pdfkeywords={\mytitel \myart \myautor}
}

%% Roman pagenumbers good aligned in toc
%\usepackage{tocloft}
%\cftsetpnumwidth{2em}
%\setlength\cftafterfigskip{10pt}
%\renewcommand\cftchapafterpnum{\vskip10pt}
%\renewcommand\cftsecafterpnum{\vskip15pt}

% fortlaufendes Durchnummerieren der Fußnoten ----------------------------------
\usepackage{chngcntr}

% für lange Tabellen -----------------------------------------------------------
\usepackage{longtable}
\usepackage{array}
\usepackage{ragged2e}

% seiten rotieren
\usepackage{pdflscape}

% Spaltendefinition rechtsbündig mit definierter Breite ------------------------
\newcolumntype{w}[1]{>{\raggedleft\hspace{0pt}}p{#1}}

% Formatierung von Listen ändern -----------------------------------------------
\usepackage{paralist}

% bei der Definition eigener Befehle benötigt
\usepackage{ifthen}

% definiert u.a. die Befehle \todo und \listoftodos
\usepackage{todonotes}

% sorgt dafür, dass Leerzeichen hinter parameterlosen Makros nicht als Makroendezeichen interpretiert werden
\usepackage{xspace}

% einbinden anderer PDF Dateien (Deckblatt etc.)
\usepackage{pdfpages}

% Bibliographics
% Naturwissenschaftliche Bibliographien
%\usepackage[square, comma, numbers]{natbib}
% Stil der Zitate und der Bibliographie
%\bibliographystyle{natdin}
\usepackage[backend=biber]{biblatex}
\addbibresource{Sonstiges/Literatur.bib}

% Tikz libraries
\usetikzlibrary{arrows.meta}
\usetikzlibrary{intersections}

% Spalten
\usepackage{multicol}

% Anhang
\usepackage[toc,page]{appendix}

% SVG Biler
\usepackage{svg}

% euro sign
\usepackage{eurosym}

% subfigures
\usepackage{subfig}
\usepackage{floatflt}

% Quotes in babel language
\usepackage{csquotes}

% Kopf- und Fußzeilen, Seitenränder etc. ---------------------------------------
% Zeilenabstand 1,5 Zeilen -----------------------------------------------------
\onehalfspacing

% Seitenränder -----------------------------------------------------------------
\setlength{\topskip}{\ht\strutbox} % behebt Warnung von geometry
%\setlength{\bottomskip}{\ht\strutbox} % behebt Warnung von geometry
\geometry{paper=a4paper,left=30mm,right=25mm,top=25mm,bottom=25mm}

% Kopf- und Fußzeilen ----------------------------------------------------------
\pagestyle{scrheadings}
% Kopf- und Fußzeile auch auf Kapitelanfangsseiten
\renewcommand*{\chapterpagestyle}{scrheadings} 
% Schriftform der Kopfzeile
\renewcommand{\headfont}{\normalfont}
\renewcommand{\footfont}{\normalfont}

% Kopfzeile
%\ihead{\mytitel}
\chead{}
\ohead{\textit{\headmark}}
\setlength{\headheight}{1.1\baselineskip}
%\setlength{\headheight}{21mm} % Höhe der Kopfzeile
% \setheadwidth[0pt]{textwithmarginpar} % Kopfzeile über den Text hinaus verbreitern
\setheadsepline[text]{0.4pt} % Trennlinie unter Kopfzeile
% \setfootsepline[text]{0.4pt} % Trennlinie über Fußzeile

% Fußzeile
\ifoot{}
\cfoot{\pagemark}
\ofoot{}
\setlength{\footheight}{1.1\baselineskip}

% sonstige typographische Einstellungen ----------------------------------------

% erzeugt ein wenig mehr Platz hinter einem Punkt
\frenchspacing 

% Schusterjungen und Hurenkinder vermeiden
\clubpenalty = 10000
\widowpenalty = 10000 
\displaywidowpenalty = 10000

% Quellcode-Ausgabe formatieren
\lstset{numbers=left, numberstyle=\tiny, numbersep=5pt, breaklines=true}
\lstset{emph={square}, emphstyle=\color{red}, emph={[2]root,base}, emphstyle={[2]\color{blue}}}

% Fußnoten fortlaufend durchnummerieren
\counterwithout{footnote}{chapter}


% urls in bib
%%% --- The following two lines are what needs to be added --- %%%
\setcounter{biburllcpenalty}{7000}
\setcounter{biburlucpenalty}{8000}

% eigene Definitionen für Silbentrennung
\input{Praeambel/Silbentrennung}

% shell
\usepackage[minted]{tcolorbox}
\tcbuselibrary{skins}
\definecolor{topbar}{RGB}{220,220,220}
\definecolor{main}{RGB}{60,60,60}
\definecolor{quit}{RGB}{248,73,73}
\definecolor{min}{RGB}{252,182,37}
\definecolor{max}{RGB}{41,198,52}
\colorlet{offwhite}{white!96!black}
\newtcblisting{shellwindow}{%
  listing engine=minted, 
  minted language=text, 
  title={\strut}, 
  listing only, 
  enhanced, 
  colbacktitle=topbar,
  boxrule=0cm,
  left=2mm,
  width=\textwidth,
  frame hidden, 
  colback=main, 
  coltext=offwhite, 
  overlay={
    \fill [fill=quit] ([xshift=3mm]title.west) circle (1mm);
    \fill [fill=min] ([xshift=6mm]title.west) circle (1mm);
    \fill [fill=max] ([xshift=9mm]title.west) circle (1mm);
  }%
}

\newcommand{\sepCodeAndOutput}[0]{\vspace{-1cm}}

% eigene LaTeX-Befehle
% Eigene Befehle und typographische Auszeichnungen für diese

% einfaches Wechseln der Schrift, \zB: \changefont{cmss}{sbc}{n}
\newcommand{\changefont}[3]{\fontfamily{#1} \fontseries{#2} \fontshape{#3} \selectfont}

% Abkürzungen mit korrektem Leerraum 
\newcommand{\bzw}{bzw.\ }
\newcommand{\dahe}{\mbox{d.\,h.\ }}
\newcommand{\engl}{engl.\ }
\newcommand{\evtl}{evtl.\ }
\newcommand{\ggf}{ggf.\ }
\newcommand{\iA}{\mbox{i.\,A.\ }}
\newcommand{\idR}{\mbox{i.\,d.\,R.\ }}
\newcommand{\lat}{lat.\ }
\newcommand{\sog}{sog.\ }
\newcommand{\szs}{szs.\ }
\newcommand{\ua}{\mbox{u.\,a.\ }}
\newcommand{\uA}{\mbox{u.\,A.\ }}
\newcommand{\uU}{\mbox{u.\,U.\ }}
\newcommand{\Vgl}{Vgl.\ }
\newcommand{\zB}{\mbox{z.\,B.\ }}

\newcommand{\abbildung}[1]{Abbildung~\ref{fig:#1}}

\newcommand{\bs}{$\backslash$}

% erzeugt ein Listenelement mit fetter Überschrift 
\newcommand{\itemd}[2]{\item{\textbf{#1}}\\{#2}}

% einige Befehle zum Zitieren --------------------------------------------------
%\newcommand{\Zitat}[2][\empty]{\ifthenelse{\equal{#1}{\empty}}{\citep{#2}}{\citep[#1]{#2}}}

% zum Ausgeben von Autoren
%\newcommand{\AutorName}[1]{\textsc{#1}}
%\newcommand{\Autor}[1]{\AutorName{\citeauthor{#1}}}

% verschiedene Befehle um Wörter semantisch auszuzeichnen ----------------------
\newcommand{\NeuerBegriff}[1]{\textbf{#1}}
\newcommand{\Fachbegriff}[1]{\textit{#1}}

\newcommand{\Eingabe}[1]{\texttt{#1}}
\newcommand{\Code}[1]{\texttt{#1}}
\newcommand{\Datei}[1]{\texttt{#1}}

\newcommand{\Datentyp}[1]{\textsf{#1}}
\newcommand{\XMLElement}[1]{\textsf{#1}}
\newcommand{\Webservice}[1]{\textsf{#1}}

\newcommand{\quotes}[1]{»#1«}

\newcommand{\citeWithTitle}[1]{\citetitle{#1} \cite{#1}}
\newcommand{\citeVgl}[1]{\cite[vgl.][]{#1}}

\newcommand{\till}[2]{#1 -- #2}

%Minted
\newcommand{\cob}[1]{\mintinline[breaklines]{cobolfree}{#1}}
\newcommand{\jav}[1]{\mintinline[breaklines]{java}{#1}}
\newcommand{\cobolNotFree}[1]{\inputminted[bgcolor=hellgrau,fontsize=\scriptsize]{cobol}{Code/#1}}
\newcommand{\cobol}[1]{\inputminted[bgcolor=hellgrau,fontsize=\scriptsize]{cobolfree}{Code/#1}}
\newcommand{\java}[1]{\inputminted[bgcolor=hellgrau,fontsize=\footnotesize]{java}{Code/#1}}
\newcommand{\mintedCaption}[2]{\begingroup\captionsetup{type=listing}\captionof{listing}{#1\label{#2}}\endgroup}
\newcommand{\mintedCobolWithOutput}[4]{\cobol{#1}#4\mintedCaption{#2}{#3}}
\newcommand{\mintedCobol}[3]{\cobol{#1}\mintedCaption{#2}{#3}}
\newcommand{\mintedJavaWithOutput}[4]{\java{#1}#4\mintedCaption{#2}{#3}}
\newcommand{\mintedJava}[3]{\java{#1}\mintedCaption{#2}{#3}}

% Interviews
\newcommand{\interviewExpert}[2]{\subsubsection*{#1}#2}
\newcommand{\toni}[1]{\subsubsection*{Antonio Grieco}\textit{#1}}
\newcommand{\jona}[1]{\interviewExpert{Jonathan Streit}{#1}}
\newcommand{\ivo}[1]{\interviewExpert{Ivaylo Bonev}{#1}}
\newcommand{\thomas}[1]{\interviewExpert{Thomas Lamperstorfer}{#1}}

% recap
\newcommand{\recappage}[1]{
    \begin{minipage}[c]{\linewidth}
        \begin{wrapfigure}{l}{.1\linewidth}
            \vspace{-15pt}
            \includegraphics[width=\linewidth]{Bilder/recap}
            \vspace{-25pt}
        \end{wrapfigure}
        #1
    \end{minipage}
}

\newcommand{\recap}[1]{
    \setlength{\fboxsep}{1.5em}
    \colorbox{gray!25}{\parbox{\linewidth-2\fboxsep}{\recappage{#1}}}
}

% detect forward references
\newwrite\refs
\openout\refs=\jobname.refs
\makeatletter
\renewcommand\@setref[3]{%
        \ifx#1\relax
                \write\refs{'#3' \thepage\space undefined}%
                \protect \G@refundefinedtrue
                \nfss@text{\reset@font\bfseries ??}%
                \@latex@warning{Reference `#3' on page \thepage\space
                                undefined}%
                \PackageWarning{todonotes}{Undefined reference!}
        \else
                \write\refs{'#3' \thepage\space
                            \expandafter\@secondoftwo#1}%
                \PackageWarning{todonotes}{Check references: '#3' \thepage\space
                            \expandafter\@secondoftwo#1}
                \expandafter#2#1\null
        \fi
}
\makeatother

% sonstige Präambel
% Workaround für lstlistoflistings --------------------------------
\makeatletter
\@ifundefined{float@listhead}{}{%
    \renewcommand*{\lstlistoflistings}{%
        \begingroup
    	    \if@twocolumn
                \@restonecoltrue\onecolumn
            \else
                \@restonecolfalse
            \fi
            \float@listhead{\lstlistlistingname}%
            \setlength{\parskip}{\z@}%
            \setlength{\parindent}{\z@}%
            \setlength{\parfillskip}{\z@ \@plus 1fil}%
            \@starttoc{lol}%
            \if@restonecol\twocolumn\fi
        \endgroup
    }%
}
\makeatother

% Nummerierungstiefe
\setcounter{secnumdepth}{3}
\setcounter{tocdepth}{3}

% Farben
\definecolor{LightGray}{gray}{0.95}

\lstdefinelanguage{QML} 
{morekeywords={color,background,margin, visible, width, height, title, id, fill, text, anchors, centerIn, onClicked, host, onMessageReceived, Component.onCompleted, Component, onCompleted}, 
	emph={ApplicationWindow, Rectangle, Button, QmlQmqtt},
	sensitive=false, 
	morecomment=[l]{//}, 
	morecomment=[s]{/*}{*/},
	morestring=[b]", 
} 


\lstdefinelanguage{json}  
{
	emph={:, \,},
	sensitive=false, 
	morecomment=[l]{//}, 
	morecomment=[s]{/*}{*/},
	morestring=[b]", 
} 

\newcommand{\specialcell}[2][c]{%
	\begin{tabular}[#1]{@{}c@{}}#2\end{tabular}}

\titlehead{
Universität Augsburg \\ Fakultät für angewandte Informatik \\ Institute for Software \& Systems Engineering
}
\title{Programmiersprachliche Konzepte von COBOL im Vergleich mit Java -- Eine praxisorientierte Einführung}
\subject{Masterarbeit \\\normalsize{Informatik und Multimedia}}
\author{
    \Huge{Antonio Grieco} \\\\
        \small{Matrikelnummer: 1498410} \\ 
        \small{antonio.grieco@gmx.de} \\
        \small{Rosenaustraße 70} \\ 
        \small{86152 Augsburg}
} 
\publishers{
     \vfill
      Erstgutachter: Prof. Dr. Alexander  Knapp\\
      Zweitgutachter: Prof. Dr. Bernhard  Bauer\\
      Betreuer: Jonathan Streit
}
\date{\vfill \vfill \vfill \today}

% muss als letztes geladen werden
\usepackage{scrhack}

%Markierung der overfull warnings im dokument
\overfullrule=1mm


% Das eigentliche Dokument -----------------------------------------------------
%   Der eigentliche Inhalt des Dokuments beginnt hier. Die einzelnen Seiten
%   und Kapitel werden in eigene Dateien ausgelagert und hier nur inkludiert.
% ------------------------------------------------------------------------------

\begin{document}
	
	% Deckblatt ---------------------------------
	\maketitle
	% \includepdf[pages=-]{Sonstiges/Deckblatt.pdf}

	% Seitennummerierung: R?mische Ziffern; Zur?cksetzen auf 1
	\pagenumbering{Roman}
	\setcounter{page}{1}
	
	% Ueberblick --------------------------------
	\chapter*{Zusammenfassung} \addcontentsline{toc}{chapter}{Zusammenfassung}

Da die \quotes{\textbf{Co}mmon \textbf{B}usiness \textbf{O}riented \textbf{L}anguage}, kurz \mbox{COBOL} genannt, bereits Ende der 1950er Jahre entstand und daher nur wenige moderne Sprachkonzepte bietet, wird der Fokus in der Ausbildung neuer Informatiker immer mehr auf Programmiersprachen mit moderneren objektorientierten Konzepten gelegt. Dem steht gegenüber, dass COBOL immer noch wichtiger Bestandteil bestehender betrieblicher Informationssysteme ist, die es zu warten und zu erweitern gilt. Während diese Sprache heutzutage also zunehmend seltener Teil der Ausbildung von Programmierern ist, besteht, durch die Vielzahl vorhandener COBOL-Systeme, weiterhin eine hohe Nachfrage nach Experten.

Diese Arbeit gibt einen generellen Überblick über Herausforderungen, die sich in Verbindung mit betrieblichen Informationssystemen ergeben, und zeigt, wie COBOL und Java diesen Problemen begegnen. Ferner wird COBOL konzeptuell erfasst und mit Java, als Vertreter moderner Sprachen, verglichen. Dabei steht stets die praktische Anwendung der Sprachen im Vordergrund, weshalb Experteninterviews geführt wurden, um neben bestehender Fachliteratur bestmögliche Einsicht in die Entwicklung und Wartung angesprochener Systeme zu erhalten. Damit entstand ein Leitfaden, der es Programmierern mit Java-Kenntnissen erlaubt, sich mit COBOL vertraut zu machen, indem bekannte Konzepte, Muster und Konstrukte gegenübergestellt werden. Zusätzlich wird, als Ergebnis der Experteninterviews, darauf hingewiesen, wie sich der Umgang mit diesen Konzepten in der Praxis gestaltet und gestalten sollte. 
\chapter*{Abstract} \addcontentsline{toc}{chapter}{Abstract}

COBOL, which stands for the \quotes{\textbf{Co}mmon \textbf{B}usiness \textbf{O}riented \textbf{L}anguage}, began to rise in the late 1950s as a project of the US government. The aim was to design a programming language, which enables persons without knowledge of programming to engineer software systems. At that time nobody could forebode that it's still used in many existing operational information systems. It's not uncommon that these have different requirements than modern desktop or web applications regarding the environment and development. So, whilst COBOL is getting less and less attention of new programmers, the demand for highly trained and experienced professionals is still high. This thesis outlines key challenges in terms of those operational information systems and reveals how COBOL copes with them. Furthermore, COBOL gets surveyed and conceptually compared to Java, which represents state of the art programming languages. The practical approach is always on focus in this comparison, and therefore, along with available literature, experts were interviewed to get the best possible insight of development and maintenance of those systems. The purpose was to devise a guide for Java developers, which enables them to familiarize with COBOL by contrasting known concepts, pattern and constructs. The interviews led to best practice advices in combination with those concept descriptions and hints on how those are used in practice and how they should be used.
	
%	\singlespacing{
		\clearpage \tableofcontents					% Inhaltsverzeichnis
	%	\clearpage \listoffigures					% Abbildungsverzeichnis
	%	\clearpage \lstlistoflistings				% Quellcode-Listings
	%	\clearpage \listoftables					% Tabellenverzeichnis
	%	\clearpage \chapter*{\hypertarget{listofnomenclaturelink}{Abkürzungsverzeichnis}}
\addcontentsline{toc}{chapter}{Abkürzungsverzeichnis}


\begin{acronym}[OPC HDA ]
	\acro{AMQP}{Advanced Message Queuing Protocol}
\acro{API}{Programmierschnittstelle}
\acro{APK}{Android application package}
\acro{CEP}{Complex Event Processing}
\acro{COM}{Component Object Model}
\acro{CPS}{Cyber-physisches System}
\acro{DCOM}{Distributed COM}
\acro{DDS}{Data Distribution Service}
\acro{DSG}{Distributed Systems Group}
\acro{GCC}{GNU Compiler Collection}
\acro{GCM}{Google Cloud Messaging}
\acro{GUI}{Graphical User Interface}
\acro{HMI}{Human-Machine-Interface}
\acro{IDE}{Integrated Development Environment}
\acro{IoT}{Internet der Dinge}
\acro{JDK}{Java Development Kit}
\acro{LHS}{left-hand side}
\acro{M2M}{Maschine-zu-Maschine}
\acro{moc}{Meta-Object Compiler}
\acro{MQTT}{Message Queue Telemetry Transport}
\acro{NDK}{Native Development Kit}
\acro{OASIS}{Organization for the Advancement of Structured Information Standards}
\acro{OPC A/E}{OPC Alarms and Events}
\acro{OPC DA}{OPC Data Access}
\acro{OPC DX}{OPC Data exchange}
\acro{OPC HDA}{OPC Historical Data Access}
\acro{OPC UA}{OPC Unified Architecture}
\acro{OPC}{Open Platform Communications}
\acro{PLC}{Programmable Logic Controller}
\acro{QML}{Qt Meta-object Language}
\acro{QoS}{Quality of Service}
\acro{RHS}{right-hand side}
\acro{SCADA}{Supervisory Control and Data Acquisition}
\acro{SDK}{Software Development Kit}
\acro{SPS}{Speicherprogrammierbare Steuerung}
\acro{WSN}{Wireless Sensor Network}
\end{acronym}	% Abk?rzungsverzeichnis
%	}
	
	% Zeilenabstand setzen
	\onehalfspacing
	
	% Seitennummerierung: Speichern der Seitennumber in RomanSiteCounter; Arabische Ziffern; Zur?cksetzen auf 1
	\pagebreak
	\newcounter{RomanSiteCounter}
	\setcounter{RomanSiteCounter}{\value{page}}
	\pagenumbering{arabic}
	\setcounter{page}{1}
	
	% Inhalt ------------------------------------
		% Einleitung ----------------------------
			\chapter{COBOL und seine Bedeutung} 

\label{ch:einleitung}
    %\section{Typographische Konventionen}

In diesem Abschnitt werden typographische Konventionen festgelegt, um das Verständnis zu erleichtern.

\begin{itemize}
	
	\item	Fachbegriffe werden \Fachbegriff{kursiv} geschrieben.
	
	\item	Zitate werden in \quotes{doppelten Anführungszeichen} geschrieben.
	
	\item	Quellcode wird in \lstinline!Festschrittschrift! geschrieben.

	\item	Neue und wichtige Begriffe werden \NeuerBegriff{fett} geschrieben.
	
	\item	Abkürzungen werden bei der ersten Verwendung ausgeschrieben und können zusätzlich im
			\hyperlink{listofnomenclaturelink}{Abkürzungsverzeichnis} nachgeschlagen werden.

\end{itemize}
    \section{Problemstellung}\label{problemstellung}
Der folgende Abschnitt soll die Problemstellung verdeutlichen, welche der Arbeit zu Grunde liegt. 
Dazu wird erläutert, welche Wichtigkeit COBOL genießt und anschließend mit der Bedeutung, die der Sprache in der Lehre tatsächlich beigemessen wird und den Folgen davon für den Arbeitsmarkt, gegenübergestellt.

\subsection*{Wichtigkeit von COBOL}\label{wichtigkeit}
\quotes{Viele Millionen Cobol-Programme existieren weltweit und müssen laufend gepflegt werden.
Es ist bei dieser Situation undenkbar und unter wirtschaftlichen Gesichtspunkten unvertretbar in den nächsten Jahren eine Umstellung dieser Programme auf eine andere Sprache durchzuführen.} \cite{_ist_1979}

Was Herr Dr. Strunz neben vielen anderen Experten bereits 1979 prophezeite hat auch heute noch Gültigkeit. Obwohl COBOL zum Ende der 50er Jahre entstand, 1959 veröffentlicht wurde und damit fast 60 Jahre alt ist, trifft man es auch heute noch häufig an. In der britischen Tageszeitung The Guardian, zitiert der Autor Scott Colvey in seinem Artikel %``\citefield{colvey_cobol_2009}{title}'' 
\cite{colvey_cobol_2009} anlässlich des 50. Geburtstages von COBOL den Micro Focus Manager David Stephenson: \quotes{`some 70\% to 80\% of UK plc business transactions are still based on Cobol'}. 
Weiter führt er darin Aussagen von IBM Software-Leiters Charles Chu an, welcher die Aussagen von Stephenson bestätigt: \quotes{$[\ldots]$ there are 250bn lines of Cobol code working well worldwide. Why would companies replace systems that are working well?'}. 
Stephen Kelly, Geschäftsführer von Micro Focus, betont zudem, dass sich Stand 2009 über 220 Milliarden COBOL-Codezeilen im produktiven Einsatz befanden, welche vermutlich 80\% der insgesamt weltweit aktiven Codezeilen ausmachten. Außerdem wurden damaligen Zeitpunkt wurden geschätzt 200-mal mehr COBOL-Transaktionen ausgeführt als Google Suchanfragen verzeichnen konnte. \cite{kelly_cobol_2009} Diese Aussagen decken sich mit den Angaben in \citeWithTitle{doke_cobol_2005}. Auch darin heißt es, dass mit 225 Milliarden Codezeilen, etwa 70\% des weltweiten Codes in COBOL geschrieben sind.

Nicht nur, dass COBOL in den vergangenen Jahren also einen enormen Marktanteil ausmachte wird also deutlich, sondern auch die Bedeutung für die Zukunft: Wieso sollte funktionierender Code mit Hilfe von teuren und riskanten Prozessen ersetzt werden?

Da sich viele Unternehmen der Frage eines Umstiegs von COBOL auf eine modernere Lösung ausgesetzt sehen, auf die sich nur schwer eine Antwort finden lässt, welche die Risiken und Kosten aufwiegt, stieg die Anzahl \todo[inline]{Beleg} des sich weltweit in Produktion befindlichen COBOL-Codes über die vergangen Jahre sogar noch weiter an.
Dieses Risiko ergibt sich vorrangig durch die Trasnaktion immenser Geldsummen, die mit COBOL-Systemen durchgeführt werden: \quotes{Täglich werden Transaktionen mit einem Volumen von schätzungsweise drei Billionen Dollar über Cobol-Systeme abgewickelt. Dabei geht es um Girokonten, Kartennetze, Geldautomaten und die Abwicklung von Immobilienkrediten. Weil die Banken aggressiv auf eine Digitalisierung ihres Geschäftes setzen, wird Cobol sogar noch wichtiger. Denn Apps für Smartphones etwa sind in modernen Sprachen geschrieben, müssen aber mit den alten Systemen harmonieren.} \cite{beat_balzli_cobol-programmierer_2017}

Im TIOBE-Index\cite{_tiobe_} für April 2018 rangiert COBOL auf Platz 25 mit einem Rating von 0.541\%. Dieser Index wird auf Basis von Suchanfragen nach den entsprechenden Programmiersprachen, auf den meist frequentiertesten Internetseiten, erstellt. COBOL ist somit zwar nur Teil jeder 200. Suchanfrage, rangiert jedoch damit trotzdem vor anderen etablierten oder aufstrebenden Sprachen wie \textit{Kotlin}, \textit{Scala} oder \textit{Haskell}. Außerdem gilt es hier zu beachten, dass COBOL zu einer Zeit entstand, in der das Internet noch lange nicht existierte und Informationen über die Sprache mittels Büchern verbreitet und vermittelt wurden. Daher ist auch heute noch das Internet nicht die vorrangige Quelle, um Wissen über COBOL zu akquirieren. Unter diesen Gesichtspunkten ist das TIOBE-Rating von COBOL als noch höher einzuschätzen.


\subsection*{Bedeutung in der Lehre}
Da COBOL bereits 60 Jahre alt ist haben heutzutage bereits viele einstige COBOL-Entwickler das Rentenalter erreicht. Im Artikel \citeWithTitle{beat_balzli_cobol-programmierer_2017} beschreibt der Autor exemplarisch den Fall eines 75-jährigen Entwickler, der ob seiner Erfahrung und trotz seines Alters immernoch in der Branche tätig ist.

Junge COBOL-Entwickler sind rar, da COBOL nur noch selten Teil der Ausbildung ist. \citeauthor{doke_cobol_2005} führen in \citeWithTitle{doke_cobol_2005} an, dass im Jahr 2002 lediglich in 36.2\% COBOL Teil des Gundstudiums war, obwohl im Jahr 1995 noch 89.7\% der befragten Bildungseinrichtungen angaben COBOL-Kurse als festen Bestandteil der Ausbildung zu haben. Sieht man sich dagegen die Zahlen zu Java als Vertreter moderner Programmiersprachen an, lässt sich ein klarer Trend erkennen. Erst 1995 entstanden, stieg die Zahl der  Universitäten, die Java lehrten von 42.5\% im Jahr 1998 auf 90.0\% im Jahr 2002. Spinnt man diesen Wandel ins heutige Jahr weiter, zu dem sich in der Zwischenzeit noch eine Fülle neuerer lässt sich erahnen wie selten Lehrveranstaltung zum Thema COBOL inzwischen geworden sind.

Man sieht also, dass sich die Lehrer, obwohl der Bedarf an COBOL-Programmierern weiterhin immens ist, stark weg von COBOL fokussiert hat was die Wirtschaft zusammen mit dem zunehmenden Alter erfahrener COBOL-Entwickler vor Probleme stellt.

\subsection*{Kontroverse Beurteilungen von COBOL}
Die in \autoref{wichtigkeit} angeführten Aussagen und Meinungen stammen oftmals von Personen aus dem Umfeld von Unternehmen, die teils hohen Profit aus dem Weiterbestehen COBOLs herausschlagen. Diese Aussagen sind daher, wenn auch sicherlich nicht falsch, vorsichtig und vor allem sehr differenziert zu betrachten.

Der mehrfach prämierte Informatiker \citeauthor{edsger_wybe_dijkstra_how_1975} z.B. findet sehr klare andere Worte zu COBOL: \quotes{The use of COBOL cripples the mind; its teaching should, therefore, be regarded as a criminal offence.} \cite{edsger_wybe_dijkstra_how_1975}

\citeauthor{florian_hamann_banken_2017} nennt in seinem Artikel \citeWithTitle{florian_hamann_banken_2017} die bereits erwähnte zunehmende Knappheit von Arbeitskräften auch als einen wichtigen Faktor dafür, weshalb COBOL über kurz oder lang von moderneren Systemen und Sprachen verdrängt und abgelöst wird.

Trotz dieser Kontroversen kann festgehalten werden, dass es nach wie vor eine gleichbleibend hohen Bedarf an Entwicklern gibt, den es zu decken gilt.
    \section{Ziel der Arbeit}
Die vorliegende Arbeit soll einen Beitrag zur Lösung der in \autoref{problemstellung} beschriebenen Probleme leisten. Dies soll mit Hilfe eines Leitfadens geschehen, der fachkundigen Java-Entwicklern den Einstieg in COBOL erleichtert, indem gängige Sprachkonzepte gegenübergestellt und verglichen werden. 
So soll es möglich sein vorhandenes Wissen über Softwareentwicklung in einen COBOL-Kontext zu bringen und passende Sprachkonzepte nutzen zu lernen. 
Des weiteren soll aufgezeigt werden, welche konzeptuellen Herausforderungen sich bei der COBOL-Entwicklung und -Migration ergeben

Ziel der Arbeit ist es jedoch nicht vorhandene Java-Entwickler umzuschulen und zu COBOL-Entwicklern zu machen. Wichtig ist vielmehr ihr Wissensspektrum so zu verbreitern, dass es ihnen möglich wird flexibler eingesetzt zu und mit Migrations- und Renovierungsaufgaben betraut zu werden.
	\section{Aufbau der Arbeit}

Die vorhandene Literatur und das Vorgehen bei der Erstellung der Arbeit werden in \autoref{ch:methodik} erläutert. \autoref{ch:herausforderungen} behandelt die grundlegenden Herausforderungen bei der Entwicklung von betrieblichen Informationssystemen und zeigt, wie sich diese Problemstellungen in COBOL und Java adressieren lassen. Die wichtigsten Sprachmittel und Konzepte werden in \autoref{ch:sprachkonzepte} aufgezeigt und gegenübergestellt. \autoref{ch:pattern} veranschaulicht wichtige und häufig auftretende Muster der Sprachen und beschreibt, wie und ob diese in der jeweils anderen abgebildet werden können. Das \autoref{ch:fazit} beinhaltet eine Zusammenfassung und Interpretation der Thematik und gibt ein Re­sü­mee der Arbeit.
		% Methodik
			\chapter{Methodik der Arbeit} 
\label{ch:methodik}
	\section{Vorhandene Literatur}
Die aufgeführte Literatur gibt oftmals mehr einen gesamtheitlichen Einblick in COBOL und 
	\section{Experten-Interviews}\label{interviews}
Die vorliegende Arbeit soll vorhandene Expertise von Experten nutzen, um statt einem Nachschlagewerk für syntaktische Zwecke einen Leitfaden zu erarbeiten der von praktischer Relevanz getrieben die wichtigen Feinheiten von COBOL und Java beleuchtet und gegenüberstellt. Daher stellen neben den angesprochenen literarischen Quellen, vor allem Experteninterviews einen Kernpunkt dieser Arbeit dar. Um den Praxisbezug zu gewährleisten wurden diese geführt, transkribiert und darauf aufbauend relevante Themenbereiche und Praktiken ausgemacht und analysiert.
\\
Interviewt wurden Experten aus dem Hause der \textbf{itestra GmbH}, die sich als \quotes{ein international anerkannter Kompetenzträger in der Neuentwicklung von Individualsoftware sowie der Optimierung und Renovierung bestehender Lösungen}\footnote{https://www.kununu.com/de/itestra1} versteht. Die Mitglieder der Entwicklerteams können dabei auf mehrjährige Erfahrungen im Bereich der Renovierung und dem Reengineering von COBOL-Systemen blicken, was die befragten Personen zur wohl wichtigsten Quelle dieser Arbeit macht.
\\
Befragt wurden die drei kundigen COBOL-Entwickler \textit{Ivaylo Bonev}, \textit{Jonathan Streit} und \textit{Thomas Lamperstorfer}. Dabei ging es nicht um repräsentative Meinungen zu bestimmten vorausgewählten Fragen, sondern um die individuelle Einschätzung der Schwierigkeiten und Stolpersteine bei der Entwicklung, Wartung und dem Verständnis von bestehenden und neuen COBOL-Systemen, sowie der Einschätzung zu Parallelen und Diskrepanzen mit Java. 
\\
Daher wurde kein Fragenkatalog ausgearbeitet sondern offener Input der erwähnten Personen gefordert, um die gewünschten subjektiven Meinungen zu bekommen. Abschriften dieser Interviews finden sich in \autoref{appendixInterviews}.
    \section{Entwicklungsumgebungen}
Um Codebeispiele für diese Arbeit zu erstellen, zu kompilieren und auszuführen, wurden jeweils für Java und COBOL IDEs verwendet.

Für Java-Code wurde die bekannte Eclipse\footnote{\url{http://www.eclipse.org/} \visitedOn} Umgebung verwendet. Dabei handelt es sich um einen etablierte IDE, welche eine Vielzahl von Funktionen zur Entwicklung und zum Debugging liefert.

Der COBOL-Code dieser Arbeit wurde in der OpenCobolIDE\footnote{\url{https://github.com/OpenCobolIDE/OpenCobolIDE} \visitedOn} entwickelt. Dabei handelt es sich um eine minimalistische IDE, welche zum Beispiel Syntax-Highlighting oder eine übersichtliche Darstellung von Fehlern bietet. Der darunterliegende Compiler GnuCOBOL\footnote{\url{https://sourceforge.net/projects/open-cobol/} \visitedOn} wurde jedoch auch teilweise direkt als Kommandozeilenwerkzeug ausgeführt. Im Gegensatz zur sonst üblichen COBOL-Entwicklung auf einem Hostsystem ermöglicht dieser Compiler das Erzeugen von ausführbaren Dateien für gängige Linuxsysteme. Dies war in dieser Arbeit sehr wichtig, um nicht auf ein System angewiesen zu sein, welches meist nur in Produktivumgebungen betrieben wird und zu dem der Zugriff oft mühsam und -- durch die verschiedenen Abrechnungsmodelle dieser Hostrechner -- teuer oder schlichtweg nicht möglich ist.

\subsection*{Das erste COBOL-Programm}

Um bereits an dieser Stelle einen kleinen Einblick in COBOL, die Programmierung und die Ausführung mit der OpenCobolIDE zu bekommen, wird ein kurzes COBOL-Programm implementiert. Die einzelnen Bestandteile davon werden im Laufe der Arbeit genauer beschrieben.

\mintedCobolNotfree{newCobol.cbl}{Erstellen eines neuen COBOL Programms}{firstCOBOLProgram}

Legt man in der OpenCobolIDE ein neues Programm an, so enthält die Datei das bekannte \quotes{Hello world} als Beispielprogramm. Wir haben dieses Programm nun so erweitert, dass es die Eingabe eines Benutzernamens erwartet und eine persönliche Begrüßung ausgibt. \autoref{firstCOBOLProgram} zeigt das fertige Programm.

Als erstes wird die \cob{PROGRAM-ID} festgelegt. Dies ist der Programmname, wie er auch nach außen -- für eventuelle andere Programme -- sichtbar wird, und sollte daher eindeutig sein. Wichtig hierbei ist auch das Setzen des Namens in der letzten Zeile, die das \cob{END PROGRAM} enthält. Diese Zeile kann entfallen, muss aber den richtigen Programmnamen enthalten wenn sie verwendet wird.

Anschließend wurde eine Variable mit dem Namen \cob{USERNAME} angelegt, die aus 20 alphanumerischen Zeichen (\cob{PIC X(20)}) besteht und mit Leerzeichen (\cob{VALUE SPACES}) initialisiert wird.

Mittels \cob{DISPLAY} wird der Nutzer aufgefordert, seinen Namen einzugeben, den das \cob{ACCEPT}-Schlüsselwort dann in die angesprochene Variable schreibt.

\begin{codeWithCaption}{Erstes COBOL-Programm in der Kommandozeile}{cobc}
    \begin{shellwindow}
    $ cobc -x HELLO_USER.cbl
    $ ./HELLO_USER
    Your name: Toni
    Hello Toni
    \end{shellwindow}
\end{codeWithCaption}

Anschließend wird geprüft, ob der Nutzer eine Eingabe gemacht hat. Ist dies der Fall, wird eine persönliche Begrüßung ausgegeben. Andernfalls erscheint die generische Meldung \quotes{\cob{Hello world}}. Diese Ausgaben werden wie bereits die Eingabeaufforderung mit \cob{DISPLAY} ausgegeben.

Kompiliert wird das Programm nun mit den Tasten \textit{F8} (kompilieren) \bzw \textit{F5} (kompilieren und ausführen). Der GnuCOBOL-Compiler kann durch ein einfaches Kommando direkt ausgeführt werden, um eine ausführbare Datei zu erstellen. Dies wird in \autoref{cobc} dargestellt.
		% Konzeptuelle Herausforderungen -------------------------------
			\chapter{Herausforderungen für COBOL und Java in betrieblichen Informationssystemen}
Bei der Entwicklung von betrieblichen Informationssystemen sehen sich Entwickler mit grundlegenden Fragen und Anforderungen an die einzusetzenden Technologien und Programmiersprachen konfrontiert.

Dieses Kapitel soll einen Überblick über die wichtigsten Entscheidungskriterien für die Herangehensweise geben und aufzeigen, welchen Herausforderungen sich -- im Speziellen COBOL und Java -- in diesen Informationssystemen stellen müssen.

\label{ch:herausforderungen}
    \section{Datenmengen und Dimensionierung}

Betriebliche Informationssysteme sind in der Regel dafür konzipiert, große Datenmengen zu verarbeiten. Daher ist bereits bei der Planung eines solchen Systems wichtig, den späteren Datenumfang abzuschätzen, um nachträglichen Erweiterungen entgegenzuwirken, auch wenn sich diese nie ganz ausschließen lassen.

\mytodo{\todo[inline]{Java?}}

\mytodo{\todo[inline]{COBOL beispiel Array zu klein}}
    \section{Langlebigkeit \& Verlaesslichkeit}
\todo{Wartbarkeit}
    \section{Modularisierung}
Linkage über programmnamen in variable.
    \section{Darstellungsgenauigkeit -- Binäre und dezimale Kommaarithmetik}\label{rechengenauigkeit}
Vor allem in betrieblichen Informationssystemen -- die oftmals Geldbeträge durch eine gewisse Anzahl von Rechenschritten errechnen -- ist es unerlässlich, einen Blick auf die Rechengenauigkeit des Systems und der verwendeten Sprachen zu werfen. Diese ist oftmals eine Folge der Speicherrepräsentation rationaler Zahlen, die erheblichen Einfluss auf den Darstellungsbereich hat. Man unterscheidet grundsätzlich zwischen Speicherungen in Fließ- und Festkomma-Darstellung.
 
\subsection*{Fließkommaarithmetik}
In modernen Programmiersprachen wie Java werden Datentypen für rationale Zahlen in der Fließkommarepräsentation gespeichert. Daher auch der Name \jav{float} (\engl \quotes{floating point}). Diese Darstellung hat den großen Vorteil, dass sowohl kleine Zahlen, die gegen Null gehen, als auch sehr große Zahlen mit dem gleichen Speicherbedarf dargestellt werden können, da quasi das Trennzeichen der Vor- und Nachkommastellen verschoben wird. Java verwendet zur Darstellung standardmäßig den Datentypen \jav{double}, ein \jav{float} mit doppelter Darstellungsgenauigkeit \bzw doppeltem Speicherbedarf, der es ermöglicht, als kleinsten Absolutwert $2^{-1074}$ und als größten $(2 - 2^{-52}) \cdot 2^{1023}$ darzustellen.

Diese Fließkommatypen werden zur Basis 2 berechnet und heißen daher auch binäre Fließkommatypen. Dabei werden Zahlen nach \textit{IEEE 754}-Standard in Vorzeichen, Exponent und Mantisse umgerechnet und gespeichert. Ohne näher auf diesen eingehen zu wollen, kann angemerkt werden, dass dieser einen Algorithmus festlegt, mit dessen Hilfe Variablen in einem Speicherbereich repräsentiert werden. Auch die Größe dieses Speicherbereichs ist durch den Standard vorgegeben und daher fest. Dadurch und durch den Umstand, dass sowohl Exponent als auch Mantisse ins Dualsystem umgerechnet werden, ergibt sich die Problematik, dass eine Dezimalzahl \uU nicht exakt repräsentiert und lediglich die nächste Repräsentation gespeichert werden kann, da eine Kommaverschiebung nur um dyadische Zahlenwerte passieren kann. Dieser Effekt ist schwer absehbar und kann in der Praxis zu ungenauen (Zwischen-)Ergebnissen führen.

\begin{codeWithCaption}{Ungenauigkeit am Beispiel einer float-Variable}{floatJava}
    \java{PrecisionExample.java} \cFollow
    \begin{shellwindow}
    $ java PrecisionExample
    0,50000000 = 111111000000000000000000000000
    0,50000006 = 111111000000000000000000000001
    0,50000000 = 111111000000000000000000000000
    \end{shellwindow}
\end{codeWithCaption}

\autoref{floatJava} illustriert beispielhaft, wie die Repräsentation eines Wertes vom tatsächlichen abweichen kann. Die erste Ausgabe des Programms stellt den \jav{float}-Wert $0,5$ in binärer Speicherrepräsentation dar. Diese wurde für die zweite Ausgabe um die kleinstmögliche Einheit, einen Bitwert, erhöht. Die nächstgrößere darstellbare Zahl ist demnach $0,50000006$, was bedeutet, dass keine Zahlenwerte dazwischen abgebildet werden können. Bei der dritten Ausgabe wird gezeigt, dass sich der Wert $0,50000002$ nicht als \jav{float} darstellen lässt, sondern die nächstmögliche Repräsentation $0,5$ gewählt wird.

Durch die Weiterverwendung eines solchen, nicht-exakt repräsentierten Werts würden sich unter Umständen Folgefehler in Berechnungen ergeben. Außerdem können Vergleiche von Zahlen, insbesondere von Berechnungsergebnissen, dadurch fehlerbehaftet sein, weshalb Fließkommadaten stets auf ein Werteintervall statt auf Gleichheit geprüft werden sollten.

In der \jav{java.math}-Bibliothek findet sich jedoch auch ein Objekttyp \jav{BigDecimal}, welcher einen Fließkommawert zur Basis 10 -- ein dezimales Fließkomma -- darstellt. Die Speicherung beruht in­des­sen auf zwei \jav{Integer}-Werten, die einen unskalierten Faktor und einen Exponenten zur Skalierung repräsentieren. Außerdem ist dieser Typ steuerbar was die Rundung, die Exaktheit von Ergebnissen und das Verhalten bei nicht-darstellbaren Werten angeht. Hierbei lässt sich festhalten, dass \jav{BigDecimal}s nur über andere \jav{BigDecimal}-Objekte oder \jav{String}s zuverlässig instanziiert werden können, da andere Konstruktoren die übergebenen Werte in Datentypen zwischenspeichern, die zu eben diese ungewünschten Fehlern in der Repräsentation führen. \jav{BigDecimal} bietet somit eine Möglichkeit, Dezimalwerte exakt abzuspeichern. Mit diesem Objekttypen gehen jedoch Speicher- und Laufzeit-Overheads einher, die nicht vernachlässigt werden dürfen. Außerdem müssen Zwischenergebnisse \idR zusätzlich abgeschnitten oder gerundet werden, da der \jav{BigDecimal}-Typ keine festgelegte Anzahl an Nachkommastellen hat und diese sich durch Berechnungen verändern können.

\subsection*{Festkommaarithmetik}
Um die angesprochenen Probleme mit binären Fließkommatypen zu umgehen, verwenden manche Sprachen Festkommaarithmetik, um rationale Zahlen zu speichern, oder bieten zumindest Datentypen, um eine derartige Speicherrepräsentation zu erreichen. 

Dabei wird im Gegensatz zu Fließkommazahlen festgelegt, wie viele Stellen einer Zahl vor- \bzw nach dem Komma gespeichert werden. Jede Ziffer wird dabei für sich -- je nach Implementierung durch eine bestimmte Codierung -- gespeichert und erlaubt somit absolute Genauigkeit im Werte- \bzw Darstellungsbereich. Auch ist der Umgang mit Überläufen fest definiert und führt zu konsistentem und abschätzbarem Verhalten. COBOL verwendet diese Festkommaarithmetik. Ergebnisse werden zur Speicherung \quotes{abgeschnitten}, außer man definiert explizit, dass gerundet wird. \autoref{decimalsInCobol} enthält Beispiele zu beiden Varianten. \cob{PIC 9V9(2)} deklariert eine Variable mit genau einer Vor- und zwei Nachkommastellen. Damit wäre beispielsweise sichergestellt, dass alle Geldbeträge $<10$ -- auch nach Berechnungen -- korrekt dargestellt werden können.

Da Ergebnisse zeichenweise gespeichert werden und so keine Rundungsfehler oder Fehler aufgrund von unzureichendem Speicherplatz zur Abbildung zulassen, ergibt sich jedoch ein erhöhter Speicherbedarf. Dieser kann in COBOL jedoch zum Beispiel durch das Nutzen von \cob{PACKED DECIMAL}s mit dem Schlüsselwort \cob{COMP-3} hinter der \cob{PICTURE}-Anweisung reduziert werden. Hierbei wird lediglich ein Nibble (\nicefrac{1}{2} Byte) pro Ziffer benötigt.

\begin{codeWithCaption}{Dezimalzahlen in COBOL}{decimalsInCobol}
    \cobol{PRECISION_EXAMPLE.cbl} \cFollow
    \begin{shellwindow}
    0.99
    1.00
    \end{shellwindow}
\end{codeWithCaption}

\recap{rechengenauigkeit}{In betrieblichen Informationssystemen und speziell bei der Verarbeitung von Geldbeträgen ist es unerlässlich, die Sicherheit einer exakten Darstellung von Werten zu haben. Während die binäre Fließkommadarstellung Speicherplatz-Vorteile und eine Flexibilität des Wertebereichs einer Zahl bietet, jedoch Werte unter Umständen nicht exakt repräsentieren kann, stellt eine Festkommaarithmetik sicher, dass Zahlen exakt und vorhersehbar repräsentiert werden. Dies wird durch erhöhten Speicherbereich und fehlende Flexibilität erkauft, ist jedoch in der Praxis oftmals unerlässlich. Eine Möglichkeit, diese Sicherheit in Java zu erreichen, ist das Nutzen des \jav{BigDecimal}-Typen, der viele Nachteile und vor allem Unsicherheiten gegenüber binären Fließkommatypen aus dem Weg räumt. Jedoch führt dieser unter Umständen zu Performanz- \bzw Speichereinbußen und bedarf \uU nach Berechnungen weiterer Bearbeitung. COBOL bietet mit Verwendung der Festkommaarithmetik bereits standardmäßig eine Darstellungssicherheit und Vorhersagbarkeit von Dezimalzahlen, die vielen modernen Sprachen fehlt.} %!!!!!!!!!!!!!!!!!
    \section{Schnittstellen und Datenquellen}
\todo{datenquellen}
In betrieblichen Informationssystemen stellen außerdem Schnittstellen ein wichtiges Thema dar. Sowohl das Bereitstellen von standardisierten und dokumentierten Interfaces als auch das Nutzen von anderen Systemen über ihre Schnittstellen ist stets Teil aller Anwendungsfälle. 

Vor allem in der heutigen Zeit, in der Informationssysteme nicht mehr als alleinige Verarbeitungs-, Reporting- und Darstellungsschicht fungieren, sondern eingebettet in einen größeren Kontext aus verschiedensten Modulen, mobilen Applikationen und Websites funktionieren sollen und mit diesen kommunizieren müssen ist ein ausgereiftes Schnittstellenkonzept und die standardisierte Bereitstellung und Nutzung von Daten und Diensten unerlässlich.

Für Java sind an dieser Stelle ein Fülle an Bibliotheken erhältlich, welche Netzwerkkommunikation über verschiedenste Protokolle auf unterschiedlichen Ebenen ermöglichen. Neben diversen Fremdbiblitheken bietet bereits das JDK bereits unterschiedliche Methoden zur Kommunikation mit Fremdsystemen.

Zudem ist es mit der möglichen Definition von Interfaces auch auf Klassenebene möglich Schnittstellen zu bieten, die eine einfache Erweiterung von und Verbindungen zu Neusystemen möglich machen.

COBOL hingegen lässt an dieser Stelle einige Funktionalität vermissen. Durch das fehlende Bibliothekskonzept -- wie in \autoref{wiederverwendbarkeit} erläutert -- und das gänzliche Fehlen von Netzwerkkommunikationsmechanismen ist es, in reinem COBOL, nicht möglich Netzwerkschnittstellen festzulegen die von außen erreichbar sind oder solche zu nutzen. Auch intern kann ein COBOL-System nur bedingt Standards definieren die zwischen unterschiedlichen Programmteilen für einheitliche Kommunikationskanäle sorgen. COBOL-Systeme basieren, wie \autoref{reporting} beschreibt, auf einfachen EVA-Pinzipien.

Durch den Wandel der letzten Jahre bzw. Jahrzehnte in hochdimensionalen und komplexen Softwaresystemen wird es immer schwerer reine COBOL-Systeme zu nutzen und sinnvoll in eine heterogene IT-Landschaft einzubetten. Kunden sind von Applikationen für verschiedenste Systeme abhängig und müssen sich darauf verlassen können, dass die Erweiterung um neue Back- und Frontends ohne Einfluss auf bestehende Komponenten vonstatten gehen kann. Auch ist es nötig die Basis für interne Systemerweiterungen zu schaffen, indem festgelegte Interfaces bedient und verwendet werden. Diese Anforderungen können mit modernen Sprachen wie Java mühelos erreicht werden, wohingegen Altsysteme nur spärliche Möglichkeiten in dieser Richtung bieten.
    \section{Reporting}\label{reporting}
\todo[inline]{EVA-Prinzip mit Zitat!}
		% Vergleich wichtiger Sprachkonzepte -------------------------------
			\chapter{Vergleich wichtiger Sprachkonzepte} 
\mylabel{ch:sprachkonzepte}
\section{Programmstruktur}\label{sec:structure}
Dieser Abschnitt behandelt die strukturellen Unterschiede von Java- und COBOL-Programmen. Dazu wird erläutert, in welche Einheiten sich die Programme der jeweiligen Sprache aufteilen lassen. 

\subsection*{Struktur eines Javaprogramms}
\autoref{javaStructureDiagram} gibt einen zusammenfassenden Überblick über die Teile eines Java-Programms und bildet graphisch ab, wie sich die jeweiligen Komponenten zusammensetzen können.

\begin{figure}[H]
    \centering
    \resizebox{.9\linewidth}{!}{
    \begin{tikzpicture}[node distance=0cm]
        \node (programRect) [inner sep=0pt, rectangle, draw, rounded corners, ultra thick, draw=black, fill=javaProgramm, minimum width=\linewidth, minimum height=.6\linewidth] at (0,0) {};
        \node[below = of programRect.north] () {Java Programm};

        \node (packageRect) [inner sep=0pt, rectangle, draw, rounded corners, ultra thick, draw=black, fill=javaPackage!50!white, minimum width=.9\linewidth, minimum height=.5\linewidth] at (0,0) {};
        \node[below = of packageRect.north] () {Packages};

        \node (classRect) [inner sep=0pt, rectangle, draw, rounded corners, ultra thick, draw=black, fill=javaClass!50!white, minimum width=.8\linewidth, minimum height=.4\linewidth] at (0,0) {};
        \node[below = of classRect.north] () {Klassen};

        \node (methodRect) [inner sep=0pt, rectangle, draw, rounded corners, ultra thick, draw=black, fill=javaMethod!50!white, minimum width=.325\linewidth, minimum height=.3\linewidth] at (.175\linewidth,0) {};
        \node[below = of methodRect.north] () {Funktionen / Methoden};

        \node (innerClassRect) [inner sep=0pt, rectangle, draw, rounded corners, ultra thick, draw=black, fill=javaClass!30!white, minimum width=.325\linewidth, minimum height=.067\linewidth] at (-.2\linewidth,.1\linewidth) {};
        \node[below = of innerClassRect.north] () {Geschachtelte Klassen};

        \node (statementRect) [inner sep=0pt, rectangle, draw, rounded corners, ultra thick, draw=black, fill=javaStatement!60!white, minimum width=.6\linewidth, minimum height=.133\linewidth] at (0,-.05\linewidth) {};
        \node[below = of statementRect.north] () {Statement};

        \node (anonClassRect) [inner sep=0pt, rectangle, draw, rounded corners, ultra thick, draw=black, fill=javaAnonClass!70!white, minimum width=.5\linewidth, minimum height=.067\linewidth] at (0,-.05\linewidth) {};
        \node[below = of anonClassRect.north] () {Anonyme Klasse};

    \end{tikzpicture}\unskip}
    \caption{Strukturelle Bestandteile eines Java-Programms \label{javaStructureDiagram}}
\end{figure}

Anhand dieses Diagramms werden die wichtigsten Konzepte der Strukturierung von Java-Code aufgezeigt. Die erste Zeile einer Quelldatei beinhaltet die Package-Deklaration, \dahe hiermit wird die Klasse dem genannten Package zugeordnet. Diese Deklaration muss gleich der Ordnerhierarchie sein, in denen die Java-Dateien verwaltet werden. 

Die nächstkleinere Einheiten eines Java-Programms stellen Klassen dar, von denen Objekte instanziiert werden können. Hierbei handelt es sich um das Kernkonzept der objektorientierten Programmierung. Um einen tieferen Einblick in diese Thematik zu erhalten, sei an dieser Stelle auf einschlägige Fachliteratur verwiesen. Diese Klasse muss dabei in einer Datei gespeichert sein, die den selben Namen trägt wie die Klasse selbst. Die Klasse \jav{MasterThesis} muss daher in der Datei \mintinline{text}{MasterThesis.java} stehen.

Teil dieser Klassen können wiederum Funktionen, in Java oft Methoden genannt, Variablendeklarationen und weitere Klassen sein. Diese inneren Klassen haben strukturell die selben Eigenschaften wie die umgebende Klasse. Die Bestandteile einer Klasse können jeweils statisch oder auch einer Instanz zugeordnet sein. Auch dabei handelt es sich um ein gängiges Konzept der objektorientierten Softwareentwicklung. Statische Methoden, Variablen und Klassen können Teil einer Klasse sein und benötigen kein konkret instanziiertes Objekt, während nicht-statische Komponenten stets ein konkretes Objekt einer Klasse benötigen. 

Methoden wiederum bestehen aus einzelnen Statements. Variablendeklarationen stellen auch Statements dar und sind an jeder Stelle innerhalb einer Klasse möglich, während andere Statements als Teil einer Klasse nur dann gültig sind, wenn diese in geschweiften Klammern stehen. Diese Blöcke werden -- der Reihe nach -- vor jedem Konstruktoraufruf ausgeführt und heißen deshalb auch \textit{Initializer}. Auch ist die Definition von statischen \textit{Initializer} möglich, die einmalig nach dem Laden einer Klasse ausgeführt werden. \autoref{initJava} führt Beispiele dafür an.

\mintedJava{Initializer.java}{Initializer in Java}{initJava}

Statements, die aus Variablendeklarationen, Zuweisungen oder Methodenaufrufen bestehen, müssen im Gegensatz zu Block-Statements, wie \zB Schleifen oder Verzweigungen, stets mit einem Semikolon beendet werden. 

Die letzten strukturellen Elemente sind anonyme Klassen und Funktionen, auch Lambda-Funktionen genannt, wobei anonyme Funktionen in Java genau genommen nur eine syntaktische Schreibweise einer speziellen anonymen Klasse sind. Die Verwendung wird in \autoref{anonymousJava} illustriert. Die Zeilen \till{9}{14} beinhalten eine anonyme Klasse, die das \jav{IntConsumer}-Interface implementiert. Die völlig identische  anonyme Klasse wird implizit durch die Lambda-Funktion in den Zeilen \till{16}{18} implementiert.

\mintedJava{AnonymousClassAndMethodExample.java}{Anonyme Klassen und Funktion in Java}{anonymousJava}

Neben der inhaltlichen Struktur kann festgehalten werden, dass Java Programme nur wenigen festen Formatierungsregeln folgen müssen. Neben den Eigenschaften, dass die Package-Deklaration vor Imports stehen, und diese wiederum vor der ersten Klasse stehen müssen, können Java-Programme nahezu beliebig formatiert werden.

\subsection*{Struktur eines COBOL-Programms}\label{cobolstructure}

\begin{figure}[H]
    \centering
    \resizebox{.9\linewidth}{!}{    \begin{tikzpicture}[node distance=0cm]
        \node (programRect) [inner sep=0pt, rectangle, draw, rounded corners, ultra thick, draw=black, fill=javaProgramm, minimum width=\linewidth, minimum height=.6\linewidth] at (0,0) {};
        \node[below = of programRect.north] () {COBOL Programm};

        \node (divisionRect) [inner sep=0pt, rectangle, draw, rounded corners, ultra thick, draw=black, fill=javaPackage!50!white, minimum width=.9\linewidth, minimum height=.5\linewidth] at (0,0) {};
        \node[below = of divisionRect.north] () {Divisions};

        \node (sectionRect) [inner sep=0pt, rectangle, draw, rounded corners, ultra thick, draw=black, fill=javaClass!50!white, minimum width=.5\linewidth, minimum height=.4\linewidth] at (-.15\linewidth,0) {};
        \node[below = of sectionRect.north] () {Sections};

        \node (paragraphRect) [inner sep=0pt, rectangle, draw, rounded corners, ultra thick, draw=black, fill=javaMethod!50!white, minimum width=.5\linewidth, minimum height=.3\linewidth] at (.15\linewidth,0) {};
        \node[below = of paragraphRect.north] () {Paragraphs};

        \node (sentenceRect) [inner sep=0pt, rectangle, draw, rounded corners, ultra thick, draw=black, fill=javaStatement!60!white, minimum width=.4\linewidth, minimum height=.2\linewidth] at (0,0) {};
        \node[below = of sentenceRect.north] () {Sentences};

        \node (statementRect) [inner sep=0pt, rectangle, draw, rounded corners, ultra thick, draw=black, fill=javaAnonClass!70!white, minimum width=.3\linewidth, minimum height=.1\linewidth] at (0,0) {};
        \node[below = of statementRect.north] () {Statements};

    \end{tikzpicture}\unskip}
    \caption{Strukturelle Bestandteile eines COBOL-Programms \label{cobolStructureDiagram}}
\end{figure}

\autoref{cobolStructureDiagram} zeigt die strukturellen Bestandteile eines COBOL-Programms. 
Ein Programm besteht dabei aus vier fest definierten Divisions:

\begin{itemize}
    \item \cob{IDENTIFICATION DIVISION} -- Hier werden grundlegende Daten zum Programm wie der Name oder der Autor festgelegt.
    \item \cob{ENVIRONMENT DIVISION} -- Definiert die Ein- und Ausgabe sowie Konfigurationen der Systemumgebung.
    \item \cob{DATA DIVISION} -- Diese Division beinhaltet die Definitionen von Daten. Dazu zählen Variablen oder auch Datei-Record-Definitionen.
    \item \cob{PROCEDURE DIVISION} -- Innerhalb dieser Division befindet sich der ausführbare Code.
\end{itemize}

Eine Division -- außer der \cob{IDENTIFICATION DIVISION} -- kann wiederum aus verschiedenen Sections bestehen, wobei diese nur innerhalb der \cob{PROCEDURE DIVISION} frei definiert werden können.

Die \cob{ENVIRONMENT DIVISION} kann zwei verschiedene Sections enthalten. Definitionen zum Zielsystem finden sich innerhalb der \cob{CONFIGURATION SECTION} und Angaben zu Dateizugriffen sowie zu Ein- und Ausgabeoperationen in der \cob{INPUT-OUTPUT SECTION}. 

\mintedCobol{SpecialNames.cbl}{SPECIAL-NAMES Paragraph in COBOL}{specialNames}

Erwähnenswert hierbei ist der \cob{SPECIAL-NAMES} Paragraph der \cob{CONFIGURATION SECTION}. Darin können Definitionen vorgenommen werden, die \ua Auswirkungen auf die Syntax haben können. \autoref{specialNames} beinhaltet die Neudefinition des Dezimaltrennzeichens und die Definition einer Klasse von Werten, die in diesem Beispiel valide Werte eines Zeichens innerhalb von Namen darstellen. Diese Klassen sind jedoch nicht mit dem objektorientierten Konzept einer Klasse zu vergleichen. 

Teil der \cob{DATA DIVISION} sind folgende Sections:
\begin{itemize}
    \item \cob{FILE SECTION} -- Definiert Dateien \bzw Dateischemata, auf die im Programm zugegriffen wird.
    \item \cob{WORKING-STORAGE SECTION} -- Enthält Variablendeklarationen, welche über mehrere Programmaufrufe hinweg bestehen bleiben.
    \item \cob{LOCAL-STORAGE SECTION} -- Enthält Variablendeklarationen, die bei jedem Programmaufrufe neu alloziert werden.
    \item \cob{LINKAGE SECTION} -- Enthält Definitionen von Variablen, welche bei einem Programmaufruf von außen übergeben werden können.
\end{itemize}

In der \cob{PROCEDURE DIVISION} finden sich schließlich vom Entwickler definierte Sections, welche ein COBOL-Pendant zu Funktionen in Java darstellen.

Die nächstkleinere Einheit eines COBOL-Programms stellen Paragraphs dar. Diese lassen sich -- mit kleinen Unterschieden -- im Allgemeinen wie Sections verwenden. In bestehenden COBOL-Programmen lassen sich daher zwei unterschiedliche Stile beobachten. Auf der einen Seite gibt es Programme, die lediglich aus Paragraphs bestehen, und auf der anderen existieren Systeme, in denen Sections verwendet wurden und durch Paragraphs untergliedert sind. Generell ist zweitere Variante vorzuziehen, wie auch \citeauthor{richards_enhancing_1984} in \citeWithTitle{richards_enhancing_1984} beschreibt, da dadurch sowohl die Programmstruktur lesbarer als auch die Fehleranfälligkeit verringert wird. Auf beide Eigenschaften wird im weiteren Verlauf der Arbeit eingegangen. An dieser Stelle soll lediglich festgehalten werden, dass es verschiedene Varianten gibt.

Sections und Paragraphs können wiederum aus Sentences bestehen. Dabei handelt es sich um ein oder mehr Statements. Ein Sentence wird stets von einem Punkt abgeschlossen. Während Sections und Paragraphs Analogien zu Methoden in Java sind, kann man Sentences am ehesten mit Block-Statements -- sobald diese geschachtelt werden, stimmt diese Analogie nicht mehr -- und Statements mit Semikolon-terminierten Statements in Java vergleichen. Herr Streit merkte dazu im Interview an, dass darauf zu achten sei, dass diese Punkte keine Semantik transportieren. Dies lässt sich dadurch erreichen, dass Statements wenn möglich durch das korrespondierende \cob{END}-Statement (\zB \cob{IF} $\rightarrow$ \cob{END IF}) abgeschlossen werden. Dadurch unterbindet der Compiler die Nutzung von Punkten innerhalb der Konstrukte, welche die Semantik ändern würden.

COBOL-Programme lassen sich nicht beliebig formatieren. So folgt ein COBOL-Pro"=gramm einem festgelegten spaltenweisen Aufbau:
\begin{itemize}
    \item \textbf{Spalte \till{1}{6}}\\
    In diesen Spalten befindet sich die sog. Sequenznummer. Damit können Programmzeilen nummeriert werden. Da der Zeichensatz dafür dem zugrundeliegenden System entspricht, können Zeilen auch beispielsweise mit Buchstaben versehen werden.
    \item \textbf{Spalte 7}\\
    In dieser Spalte kann ein Zeichen gesetzt werden, um dem Compiler die Bedeutung der Zeile kenntlich zu machen. Ein {\tt\textbf{*}} leitet \zB eine Kommentarzeile ein und durch {\tt\textbf{-}} kann die vorherige Zeile fortgeführt werden.
    \item \textbf{Spalte \till{8}{11} und Spalte \till{12}{72}}\\
    Diese Spalten enthalten Definitionen und ausführbaren Programmcode. Je nach COBOL-Dialekt sind diese beiden Bereiche jedoch im Hinblick auf Variablendeklarationen unterschiedlich. Während in ersterem nur die Stufennummern \textit{01} und \textit{77} deklariert werden dürfen, müssen alle anderen in dem Bereich ab Spalte 12 stehen. Dies gilt jedoch nicht auf allen Systemen.
    \item \textbf{Spalte \till{73}{80}}\\
    In klassischem COBOL dienen diese Spalten dazu, Kommentare zur aktuellen Zeile einzufügen. Wie Herr Streit betonte, sind diese in Altsystemen exzessiv genutzt, um Versionsinformationen -- wie Änderungsdatum oder Ticketnummern -- festzuhalten, sollten jedoch zum Wohl der Übersichtlichkeit entfernt und im Zuge einer Renovierung oder Migration durch eine modernen Versionsverwaltung, wie \zB SVN oder Git, ersetzt werden. Auch ein Änderungsvergleich zwischen Programmversionen in einem entsprechenden Werkzeug wird durch diese Kommentare erheblich erschwert. Wichtig ist es jedoch zu verstehen, dass diese Kommentare, zu Zeiten, in denen es keine Versionsverwaltungssoftware gab, sinnvoll waren.
\end{itemize}

Im sogenannten \textit{Free-Format}, welches von einigen COBOL-Dialekten unterstützt wird, gelten diese Beschränkungen nicht. Dabei gilt lediglich, dass Spalte 1 wie Spalte 7 zur Kennzeichnung von Kommentaren fungiert. Auch die Breite eine Zeile kann hierbei, im Gegensatz zum klassischen COBOL, 80 Zeichen überschreiten. 
\section{Deklaration und Gültigkeitsbereich von Variablen und Datentypen} \label{variables}
Eine wichtiges Sprachmittel von Programmiersprachen ist die Verwendungsmöglichkeit von Variablen. Je nach Programmiersprache haben diese Variablen unterschiedliche Eigenschaften und werden verschieden deklariert, initialisiert bzw. definiert. Dieser Abschnitt soll die Unterschiede dabei zwischen COBOL und Java herausarbeiten.

\subsection*{Variablen in Java}
Eine Variable in Java hat stets einen bestimmten Datentypen. Dies können primitive Datentypen -- \jav{float}, \jav{double}, \jav{byte}, \jav{char}, \jav{short}, \jav{int}, \jav{long}, \jav{boolean} -- aber auch komplexe Objekttypen sein. Variablen primitiver Zahltypen haben dabei stets ein Vorzeichen.

\mintedJava{VariableExample.java}{Variablendeklarationen in Java}{variablesJava}

\autoref{variablesJava} soll einige Konzepte der Variablendeklaration und -definition verdeutlichen:
\begin{itemize}
 \item Variablen können sowohl als Teil einer Klasse als auch lokal innerhalb einer Methode deklariert werden. 
 \item Variablen mit dem \textit{Modifier} \jav{final} sind Konstanten und können nicht mehr geändert werde.
 \item Die Deklaration erfolg nach dem Muster \quotes{<Datentyp> <Variablenname>}.
 \item Die Initialisierung einer Variable geschieht durch das zuweisen eines Wertes.
 \item Komplexe Objekttypen können den Wert \jav{null} haben. Das bedeutet, die Variable, die in diesem Fall eine Referenz auf einen Speicherbereich darstellt, ist leer. Hier gilt es zu beachten, dass primitive Datentypen nicht \jav{null} sein können. 
 \item Instanzen eines Objekttypen werden durch das Schlüsselwort \jav{new} und den Aufruf eines Konstruktors erzeugt. Primitive Datentypen haben keine Konstruktoren.
\end{itemize}
Die Deklaration von Variablen bestimmter Datentypen sorgt dafür, dass ausreichend Speicherplatz für diese reserviert wird. Die Stelle der Deklaration im Code ist dabei frei wählbar und muss lediglich vor der ersten Verwendung stehen. 

Der sog. Scope, zu deutsch Gültigkeitsbereich, gibt in der Programmierung an, in welchem Bereich eine Variable gültig ist. In Java ist der Scope einer Variablen meist einfach zu erkennen. Eine Variable ist innerhalb der geschweiften Klammern gültig, die die Variablendeklaration beinhalten. Dies verdeutlicht \autoref{javaScope}.

Die Variable \jav{memberVariable} ist innerhalb der gesamten Klasse \jav{ScopeExample}, also in jeder enthaltenen Methode, verschachtelten Klassen und wiederum deren Methoden, gültig. Eine Variable mit selbem Namen kann auch innerhalb einer Methode deklariert werden. Auf die Instanzvariable kann mit dem \jav{this}-Schlüsselwort zugegriffen werden. In geschachtelten Klassen muss zusätzlich der Klassenname vorangestellt werden wie Zeile 27 zeigt.  Ist keine lokale Variable mit selbem Namen vorhanden, so kann dieses Schlüsselwort auch weggelassen werden. 

Wie Zeile 19 zeigt ist es auch nicht möglich auf lokale Variablen einer anderen Funktion zuzugreifen. Gleiches gilt für Instanzvariablen verschachtelter Klassen.

\mintedJava{ScopeExample.java}{Variablendeklarationen mit verschiedenen Scopes}{javaScope}

\subsection*{Variablen in COBOL}
Die Deklaration von Variablen unterscheidet sich in COBOL stark von der in Java. Neben der Eigenschaft, dass Variablen nur innerhalb der \cob{DATA DIVISION} -- als Teil der \cob{WORKING-STORAGE SECTION} oder der \cob{LOCAL-STORAGE SECTION} -- deklariert werden können, ist in COBOL die Definition eines Datentyps gleichzeitig auch die Festlegung der Ausgabe-Repräsentation dieser Variable. 

Dies sorgt dafür, dass bereits an der Stelle der Variablendeklaration festgelegt werden muss, wie diese Daten im folgenden Programm dargestellt werden. Das Schlüsselwort dafür ist die \cob{PICTURE}- oder kurz \cob{PIC}-Anweisung.

\cobol{variable_declaration.cbl}
\sepCodeAndOutputCheck
\begin{shellwindow}
$ ./variableExample
Mustermann 178
Mustermann
178
1.78
\end{shellwindow}
\mintedCaption{Variablendeklarationen in COBOL}{variablesCobol}

\autoref{variablesCobol} demonstriert die Deklaration von vier Variablen. Variablen Anhand dieses Beispiels sollen wiederum verschiedene Konzepte der Variablendeklaration in COBOL illustriert werden

Jede Variablendeklaration beginnt mit einer Stufennummer. Diese Stufennummer sorgt für Gruppierung von Variablen. Zulässig sind dabei Zahlen zwischen \textit{01} und \textit{49}. Die Stufennummern sollten mit ausreichendem Abstand gewählt werden -- in der Praxis werden dazu 5er-Schritte gewählt --  um ein nachträgliches Einfügen zwischen zwei Stufennummern zu erleichtern. Die speziellen Stufennummern \textit{66}, \textit{77} und \textit{88} werden später separat behandelt. Wie das Beispiel zeigt, lassen sich auf einzelne Variablen auch über den Gruppennamen zugreifen. Konstanten sind hierbei in COBOL nicht möglich und so gilt es, wie Herr Bonev und Herr Lamperstorfer betonten, sicherzustellen, dass konstante Werte an keiner Stelle des Programms verändert werden.

Der Stufennummer folgt ein eindeutiger Name für die Variable. An dieser Stelle kann jedoch auch das Schlüsselwort \cob{FILLER} verwendet werden. Dies sorgt dafür, dass eine Platzhaltervariable angelegt wird, auf die jedoch später nicht direkt zugegriffen werden kann.

Die Festlegung der Repräsentation geschieht wie bereits erwähnt durch ein \cob{PIC}. Nach diesem \cob{PIC} wird festgelegt wie diese Variable dargestellt werden soll. \quotes{X} steht dabei für ein alphanumerisches, \quotes{9} für ein numerisches Zeichen und \quotes{A} für einen Buchstaben. Die Angabe der Stellen einer Variable wird durch die Wiederholung des jeweiligen Zeichens oder die verkürzte Notation mit der nachgestellten Anzahl der Wiederholungen in runden Klammern -- z.B. XXXX $\widehat{=}$ X(4) -- erreicht. Als Dezimaltrennzeichen wird wie gezeigt ein \quotes{V} verwendet und ein vorangestelltes \quotes{S} sorgt dafür, dass eine numerische Variable ein Vorzeichen führt. Es wird also genau festgelegt wie viele Vor- und Nachkommastellen eine Variable hat.

Die Initialisierung einer Variable erfolgt durch die \cob{VALUE}-Anweisung, gefolgt von dem Wert, welcher der Variablen zugewiesen werden soll. Dabei gibt es die Schlüsselwörter \cob{SPACE} bzw. \cob{SPACES} und \cob{ZERO} bzw. \cob{ZEROS}, die anstelle eines Wertes verwendet werden können, um eine Variable mit Leerzeichen bzw. Nullen zu initialisieren.

Durch die Definition der Repräsentation findet man in der Praxis oft Variablen, die auf den selben Speicherbereich wie eine andere Verweisen, jedoch die dort enthaltenen Daten anders darstellen bzw. interpretieren. Dies geschieht wie im Beispiel gezeigt mithilfe des Schlüsselworts \cob{REDEFINES}.

Eine Typsicherheit ist in COBOL nicht ausreichend gewährleistet. So kann eine Variable mit \cob{REDEFINES} oder eine Variable, welche die eigentliche gruppiert, den Speicherbereich einer anderen mit, unter Umständen ungültigen, Werten befüllen. Auch sind uninitialisierte Variablen teilweise mit falschen Datentypen vorbelegt. 

Auch der Scope von Variablen in COBOL unterscheidet sich sehr stark von Java. Allgemein kann festgehalten werden, dass auf eine Variable von jeder Stelle innerhalb eines Programms aus zugegriffen werden kann. Das sorgt dafür, dass schnell Fehler auftreten können, die sich durch unbeabsichtigten Zugriff auf falsche Variablen ergeben. In der Praxis sind diese, so alle befragten Experten übereinstimmend, häufiger beobachtbar und bergen hohes und vor allem schwer auszumachendes Fehlerpotential. An dieser Stelle sei lediglich der Unterschied der \cob{WORKING-STORAGE SECTION} und der \cob{LOCAL-STORAGE SECTION} erwähnt. Während Variablenwerte in ersterer über mehrere Programaufrufe hinweg erhalten bleiben, werden Variablen der \cob{LOCAL-STORAGE SECTION} bei jedem Aufruf neu instanziiert. Diesen Unterschiedes sind sich COBOL-Entwickler in der Praxis nicht immer bewusst, obwohl so zumindest teilweise eine Reduzierung von Seiteneffekten durch falsch belegte Variablen erreicht werden könnte. Daher war Herrn Streits Vorschlag, Variablen wenn möglich in der \cob{LOCAL-STORAGE SECTION} anzulegen.

Zum Abschluss dieses Abschnitts sei erwähnt, dass der Speicherplatz von Variablen weder in COBOL noch in Java händisch freigegeben werden. In Java sorgt der \textit{garbage collector} dafür, dass Speicherbereich, der nicht mehr verwendet wird wieder freigegeben wird. In COBOL geschieht dies mit dem Ende eines Programms. 
\subsection{Felder}\label{sec:felder}
Eine zentrale Datenstruktur in der Programmierung stellen Felder oder Arrays dar. Dabei handelt es sich um eine geordnete Sammlung von Werten des selben Typs auf die, im Gegensatz zu z.B. verketteten Listen, direkt zugegriffen werden kann.

\mintedJava{Arrays.java}{Felder in Java}{arraysJava}
Zeile 4 in \autoref{arraysJava} beschreibt das Anlegen eines Arrays in Java mittels \mintinline{java}{new}-Schlüsselwort, wohingegen Zeile 8 den Zugriff auf ein Element zeigt. Die Indizierung der Elemente beginnt dabei mit dem Element 0. Ein Feld der Größe 10 hat also die Indizes \till{0}{9}. Sowohl beim Anlegen als auch beim Zugreifen auf ein Element des Arrays wird der \mintinline{java}{[]}-Operator verwendet.

In COBOL können Felder durch \mintinline{cobolfree}{OCCURS}, gefolgt von der Anzahl der zu speichernden Werte und \mintinline{cobolfree}{TIMES} angelegt werden. Dies illustriert \autoref{arraysCobol}. Das \mintinline{cobolfree}{INDEXED BY}-Schlüsselwort kann dazu genutzt werden, eine Variable zu definieren, mit der das Array indiziert werden kann. Dies ist jedoch nicht zwangsläufig notwendig.

\mintedCobol{arrays.cbl}{Felder in COBOL}{arraysCobol}
In Zeile 12 ist der Zugriff auf ein einzelnes Feld-Element zu sehen. Dies geschieht in COBOL mittels runder Klammern. Zu beachten ist hierbei, dass COBOL die einzelnen Elemente beginnend mit 1 indiziert. Im Gegensatz zu Java hat ein Array der Größe 10 in COBOL also die Indizes \till{1}{10}.

Eine erwähnenswerte Besonderheit von Feldern in COBOL und in Java ist, dass diese auch mehrdimensional sein können. Jedes Element der ersten Dimension besteht also aus einem weiteren Feld. Ein Zugriff auf ein beispielhaftes zweidimensionales Array ist dann mittels \mintinline{java}{[][]} in Java bzw. \mintinline{cobolfree}{(X,Y)} in COBOL möglich. Java und COBOL unterscheiden sich jedoch dahingehend, dass COBOL auch einen Zugriff auf eine ganze Dimension ermöglicht wohingegen in Java stets ein einzelnes Element referenziert werden muss. Dies gilt sowohl für schreibenden als auch lesenden Zugriff.

Eine weitere Gemeinsamkeit ist die Tatsache, dass Felder in beiden Sprachen eine feste Größe haben. Nach dem Anlegen des Feldes kann diese Größe nicht mehr geändert werden. Jedoch muss in COBOL bereits zur Zeit der Kompilierung festgelegt werden, wie viele Elemente ein Array beinhalten soll. In Java kann diese Größe auch variabel zur Laufzeit des Programms festgelegt werden.

\section{Programmablauf und Kontrollfluß}\label{programmablauf}
Durch die in \autoref{sec:functionsAndReturnValues} erläuterten Unterschiede ergeben sich auch im Programmablauf Diskrepanzen. Allerdings bleibt hier zu erwähnen, dass diese Unterschiede nicht zu einem gänzlich anderen Ablauf führen, sondern eher dafür sorgen, dass Gemeinsamkeiten nicht auf den ersten Blick erkennbar sind, obwohl der Ablauf im Grunde sehr ähnlich ist. Sowohl Java- als auch COBOL-Programme werden im Allgemeinen von oben nach unten durchlaufen. Beiden Programmiersprachen ist gemein, dass sie einen definierten Einstiegspunkt in ein Programm haben. Während jedes Java-Programm in der \jav{main}-Methode startet wird ein COBOL-Programm stets sequenziell von oben nach unten abgearbeitet und durchlaufen und beginnt daher stets mit der ersten Zeile der \cob{PROCEDURE DIVISION}.

\subsection{Ablauf}
\java{MainMethod.java}
\sepCodeAndOutputCheck
\begin{shellwindow}
$ javac MainMethod.java 
$ java MainMethod
Running main method!
Running other method!
Running return value method!
Continue main method!
\end{shellwindow}
\mintedCaption{Java main-Methode}{javaMainMethod}

\autoref{javaMainMethod} demonstriert einen sehr simplen Programmablauf in Java. Wie bereits erwähnt ist der Startpunkt eines jeden Java-Programms die \jav{main}-Methode. Von dieser aus können weitere Methoden aufgerufen werden und sobald das Ende dieser Methode erreicht ist terminiert das Programm. Im vorliegenden Beispiel wird also nach einer Ausgabe in \jav{main}, die Funktion \jav{otherMethod} aufgerufen, bevor der Ablauf wieder in der \jav{main}-Methode fortgesetzt wird. Daran soll folgendes Verhalten deutlich werden: Endet eine aufgerufene Funktion wie geplant -- d.h. ohne eine \jav{Exception} -- wird stets mit der nächsten Anweisung nach dem Funktionsaufruf fortgefahren. 

In COBOL gestaltet sich der Programmablauf ähnlich. Das Programm wird stets von oben nach unten durchlaufen. Wobei dieser lineare Ablauf z.B. durch die Verwendung von \cob{PERFORM}-, \cob{CALL}-, \cob{GO TO}- oder \cob{NEXT} \cob{SENTENCE}-Anweisungen verändert werden kann.

Ein Unterprogramm wird mit \cob{CALL} aufgerufen und gibt hingegen mit \cob{GOBACK} die Kontrolle zurück an das aufrufende Programm. Die Ausführung eines COBOL-Programms endet beim Erreichen einer \cob{STOP RUN}-Anweisung oder mit dem Ende des Programms (\cob{END PROGRAM}). 

\cobol{simpleControlFlow.cbl}
\sepCodeAndOutputCheck
\begin{shellwindow}
$ ./simpleControlFlow 
Main paragraph
Enter some number: 0
Main paragraph again
$ ./simpleControlFlow 
Main paragraph
Enter some number: 1 
Main paragraph again
Enter some number: 2
\end{shellwindow}
\mintedCaption{Programmablauf in COBOL}{simpleControlFlowCobol}

Die beiden Ausführungen von \autoref{simpleControlFlowCobol} zeigen das angesprochene Verhalten eines COBOL-Programms. Beim ersten Durchlauf wird für die Variable \cob{INPUT-NUMBER} der Wert 0 eingegeben, was durch das Ausführen der \cob{STOP RUN}-Anweisung in Zeile 15, das Beenden des Programmes bewirkt. Beim zweiten Mal wird hingegen der Wert 1 eingegeben. Dieser Wert verhindert das Abschließen des Programms in Zeile 15, wodurch der Programmablauf in Zeile 17 fortgesetzt wird und somit erneut die Eingabeaufforderung erscheint.

Wie in \autoref{sec:structure} beschrieben besteht ein COBOL-Programm aus verschiedenen strukturellen Komponenten. Diese haben auch einen gewissen Einfluss auf den Programmablauf. Dies soll das Beispiel in \autoref{paragraphSecionControlFlowCobol} veranschaulichen.

\cobol{paragraphSecionControlFlow.cbl}
\sepCodeAndOutputCheck
\begin{shellwindow}
$ ./paragraphSecionControlFlow 
Calling section:
123 
Calling paragraphs with PERFORM THRU:
123 
Calling paragraph:
1
\end{shellwindow}
\mintedCaption{Programmablaufunterschiede in COBOL mit Sections und Paragraphs}{paragraphSecionControlFlowCobol}

Wird mittels \cob{PERFORM} eine Section aufgerufen, so werden alle Paragraphs innerhalb dieser Section der Reihe nach ausgeführt. Ruft man jedoch einen Paragraph auf, so wird nur dieser Paragraph ausgeführt. Eine weitere Möglichkeit ist die Kombination des \cob{PERFORM} mit dem \cob{THRU}-Schlüsselwort. Hierbei werden alle Paragraphs zwischen zwei festgelegten Paragraphs ausgeführt. Der Kontrollfluss geht bei jeder Variante stets an das Statement nach dem \cob{PERFORM}.

Um Verwirrungen vorzubeugen und lesbaren Code zu erhalten sollten alle Paragraphs stets Teil einer Section sein und auch nur diese Ziel einer \cob{PERFORM}-Anweisung sein. Der letzte Paragraph einer Section sollte dabei immer ein \cob{EXIT}-Paragraph sein, also nur das Schlüsselwort \cob{EXIT} beinhalten. So ist das Ende einer Section beim lesen des Codes klar erkennbar. Außerdem stellt dieser \cob{EXIT}-Paragraph oftmals eine Ausnahme zur Verwendung des \cob{GO TO}-Befehls dar. Dies ist nötig, da COBOL keinen Befehl wie das \jav{return} in Java enthält, um die Ausführungskontrolle an die aufrufende Stelle zurückzugeben. Dieses Vorgehen wurde auch von \citeauthor{richards_enhancing_1984} bereits \citeyear{richards_enhancing_1984} als best-practice beschrieben  \cite{richards_enhancing_1984}. Die meisten Code-Beispiele dieser Arbeit enthalten bewusst keinen separaten \cob{EXIT}-Paragraph, um den Umfang und die Übersichtlichkeit der Listings so gering wie möglich zu halten. 

\subsection{Verzweigungen}
Eine wichtige Eigenschaft von Programmiersprachen ist konditionelle Verzweigung, also die Ausführung von Programmteilen nur unter bestimmten Voraussetzungen. Sowohl Java als auch COBOL bieten hierfür die Schlüsselwörter \jav{if}-\jav{else} (Java) bzw. \cob{IF}-\cob{ELSE}-\cob{END-IF} (COBOL). Auch die Verwendung ist sehr ähnlich wie folgende Beispiele zeigen sollen.

\mintedJava{IfExample.java}{Verzweigung in Java}{ifJava}

In \autoref{ifJava} wird anhand einer Nutzereingabe eine Fallunterscheidung bzw. Verzweigung gemacht. Dabei soll gezeigt werden, dass es möglich ist sowohl mehrere Zeilen als auch nur eine Zeile konditionell auszuführen. Soll mehr als eine Zeile untergeordnet werden, ist eine Gruppierung als Block -- mit geschweiften Klammern -- nötig. Der \jav{else}-Zweig zeigt eine einzelne Anweisung als bedingt auszuführendes Statement. Das letzte Statement zeigt die Verwendung des konditionalen Operators \quotes{?}. Dabei handelt es sich lediglich um eine kurzschreibweise für ein \jav{if}-\jav{else} mit jeweils einer Anweisung. 

\mintedCobol{IF-EXAMPLE.cbl}{Verzweigung in COBOL}{ifCOBOL}

\autoref{ifCOBOL} bildet selbige Logik in COBOL ab. Die beiden Sections \cob{END-IF-EXAMPLE} und \cob{PERIOD-IF-EXAMPLE} zeigen dabei zwei unterschiedliche Wege diese zu konstruieren. Während erstere eine \cob{ELSE}- und eine \cob{END-IF} Anweisung nutzt, um das Konstrukt aufzubauen und zu terminieren, verwendet letztere die Eigenschaft, dass ein \cob{IF} auch durch ein Sentenceende -- siehe \autoref{sec:structure} -- abgeschlossen werden kann. Dies erlaubt jedoch keine verschachtelten Verzweigungen und kann -- wie die befragten Experten anmerkten -- in der Praxis schnell zu Fehlern oder zumindest zu schwer durchschaubarem Verhalten führen. Herr Streit betonte, dass bestehende Programme teilweise solche Konstrukte beinhalten, ein \cob{IF} jedoch stets mit einem \cob{END-IF} terminiert werden sollte. Dies sorgt dafür, dass es dem Compiler möglich ist Fehler in der Verzweigung zu erkennen und eine bessere Lesbarkeit zu erreichen.

\subsection{Schleifen}

Wie in vielen anderen Sprachen unterstützen Java und COBOL auch Schleifenkonstrukte. Während Java dafür dedizierte Schlüsselwörter bereitstellt fungiert in COBOL auch dafür das \cob{PERFORM}-Statement. Dies kann für Unklarheiten sorgen weil dieses Schlüsselwort wie später in der Arbeit beschrieben weitere Funktionen erfüllt. Jedoch soll an dieser Stelle lediglich auf die Verwendung als Schleifenkonstrukt eingegangen werden.

Java bietet mit \jav{while}-, \jav{do}-\jav{while}- und \jav{for}-Schleifen drei unterschiedliche Arten von Schleifen. \autoref{loopsJava} enthält alle drei Konstrukte. Während die ersten beiden kopf- und fußgesteuert eine Bedingung überprüfen wird eine \jav{for}-Schleife i.d.R. dazu genutzt um Werte einer bestimmten (Zahlen-)Menge zu durchlaufen.

\mintedJava{Loops.java}{Schleifen in Java}{loopsJava}

COBOL nutzt für alle Schleifen das \cob{PERFORM}-Schlüsselwort. In Verbindung mit weiteren Statements entstehen so unterschiedliche Schleifentypen. \autoref{loopsCOBOL} beschreibt die wichtigsten davon. Eine bedingte Schleifenausführung lässt sich mithilfe des \cob{UNTIL}-Schlüsselworts und einer nachfolgenden Bedingung erreichen. Eine Zählschleife, entsprechend eines \jav{for} in Java, kann durch \cob{VARYING}, \cob{FROM} und \cob{BY} konstruiert werden. Jede Schleife kann zusätzlich durch die Angabe von \cob{WITH TEST AFTER} von einer kopfgesteuerten zu einer fußgesteuerten Schleife gemacht werden, d.h. die Bedingung wird nach einem Schleifendurchlauf geprüft und nicht davor.

\mintedCobol{LOOP-EXAMPLE.cbl}{Schleifen in COBOL}{loopsCOBOL}

\subsection{Weitere Schlüsselwörter}

Weitere Schlüsselwörter die den Kontrollfluß -- vor allem im Zusammenhang mit Verzeigungen und Schleifen -- in Java steuern können sind außerdem \jav{break}, \jav{continue} und \jav{goto}. Zu beachten ist dabei, dass das \jav{goto}-Schlüsselwort zwar im Sprachstandard noch definiert ist, jedoch in keiner gängigen JVM implementiert ist. Die Verwendung führt zu Fehlern beim Kompilieren. In \autoref{javaBreakContinue} finden sich beispielhafte Verwendungen der beiden anderen Schlüsselwörter.

\java{BreakContinueExample.java}
\sepCodeAndOutputCheck
\begin{shellwindow}
$ javac BreakContinueExample.java 
$ java BreakContinueExample
== break example == 
(0)(1)
== continue example == 
(0)(1)(2)(8)(9)
\end{shellwindow}
\mintedCaption{Beispiele für die Verwendung von break und continue in Java}{javaBreakContinue}

Ein einfaches \jav{break} sorgt wie gezeigt dafür, dass die direkt umfassende Schleife verlassen wird. Auch ein simples \jav{continue} hat Auswirkungen auf die direkt beinhaltende Schleife. So sorgt es dafür, dass der aktuelle Schleifendurchlauf abgebrochen und mit dem nächsten fortgefahren wird. \autoref{mehrfachverzweigungen} zeigt eine weitere Verwendung des \jav{break}-Statements. Deutlich unüblicher -- jedoch nicht weniger relevant -- ist der Gebrauch eines Labels in Java. Dieses Label kann in der Verbindung mit einer \jav{break}- oder \jav{continue}-Anweisung genutzt werden, um mehrere umfassende Schleifen verlassen bzw. um mit dem nächsten Schleifendurchlauf einer weiter außen befindlichen Schleife fortgefahren zu werden. Die Anweisung betrifft dabei die Schleife, welche das Label trägt. 

In COBOL ist ebenfalls das Schlüsselwort \cob{CONTINUE} vorhanden. Allerdings ist hierbei Vorsicht geboten, da dieses abweichende Bedeutung vom gleichnamigen Java-Schlüsselwort hat. Während in Java, wie erwähnt, zum nächsten Schleifendurchlauf gesprungen werden kann, entspricht dieses Schlüsselwort in COBOL lediglich einer Anweisung bei der nichts ausgeführt wird. Dies ist in der Praxis häufig zu beobachten, um z.B. Verzeigungsteile leer zu lassen ohne die Bedingung negieren zu müssen.

Neben diesem ist \cob{NEXT SENTENCE} ist ein Schlüsselwort das häufig in älterem Code zu finden sei, wie Herr Lamperstorfer bestätigte. Dieses kann dazu genutzt werden, um den aktuellen Sentence zu verlassen und mit der Anweisung die darauf folgt fortzufahren. Zu beobachten sei die Verwendung auch häufig zur Negation einer Bedingung, indem der \cob{IF}-Zweig lediglich dieses Statement enthält und der \cob{ELSE}-Zweig die Logik bei nichtzutreffen der Bedingung enthält. Diese Konstrukte sollten jedoch vermieden werden und durch ein einfach Negieren mit \cob{NOT} geschrieben bzw. ersetzt werden. Aus Gründen der Unübersichtlichkeit ist dieses Schlüsselwort in GnuCOBOL standardmäßig verboten. 

Seltener zu finden ist dagegen die \cob{EXIT PERFORM}-Anweisung. Diese kann innerhalb von Schleifen dazu genutzt werden, um wie ein \jav{break} die Schleife zu verlassen oder durch \cob{EXIT PERFORM CYCLE} wie ein \jav{continue} in Java mit dem nächten Schleifendurchlauf fortzufahren. Dies soll \autoref{exitPerform} verdeutlichen.

\cobol{EXIT-PERFORM.cbl}
\sepCodeAndOutputCheck
\begin{shellwindow}
00
10
\end{shellwindow}
\mintedCaption{\cob{EXIT PERFORM} in COBOL}{exitPerform}

An dieser Stelle sei ausdrücklich erwähnt, dass die Verwendung des \cob{GO TO}\index{GO TO}-Befehls -- abgesehen von oben genanntem Einsatz als \textit{return}-Ersatz innerhalb einer Section -- in COBOL unterlassen werden sollte, oftmals sogar durch projekt- oder unternehmensspezifische Vorgaben verboten ist, da ansonsten sehr schwer verständlicher und wartbarer Code entstehen kann. Leider findet man sich in der Praxis oftmals mit Code konfrontiert, der \cob{GO TO}-Befehle zur Steuerung des Ablaufs verwendet. Sogar Schleifenkonstrukte sind in älteren Programmen oft damit realisiert, worauf Herr Streit hinwies.

\subsection{Ausnahmebehandlung}

In modernen Sprachen sind Ausnahmebehandlungsmechanismen vorhanden, um die Steuerung des Kontrollflußes klar von der Fehlerbehandlung zu trennen. So wird zum einen eine übersichtlichere Implementierung erlaubt, aber auch erreicht, dass bereits der Compiler auf Fehler hinweisen kann die bei der Ausführung auftreten können bzw. gänzlich das kompilieren bei ungenügender Fehlerbehandlung verweigert.

Java bietet dabei das Konzept der \jav{Exception}s. Diese lassen sich in sogenannte \textit{checked} und \textit{unchecked}-\jav{Exception}s unterteilen. Während \textit{checked}-\jav{Exception} stets einer ausreichenden Fehlerbehandlung im Code bedürfen und ansonsten zu Fehlern des Kompiliervorgangs führen, können \textit{unchecked}-\jav{Exception}s unbehandelt gelassen werden. Von der genauen Verwendungserklärung sei an dieser Stelle abgesehen und lediglich auf weiterführende Literatur wie \citeWithTitle{byrne_java_2009-1} von \citeauthor{byrne_java_2009-1} verwiesen.

COBOL bietet zur generellen Ausnahmebehandlung keine Methodik. Fehlerfälle müssen über Variablenwerte signalisiert, geprüft und entsprechend behandelt werden. Herr Streit wies darauf hin, dass eine ungenügende Prüfung hierbei zum kompletten Absturz des Programms führen kann. Jedoch ist es möglich vordefinierte Fehler bei Berechnungen oder String-Zuweisungen abzufangen und darauf zu reagieren wie \autoref{exceptionsCobol} zeigt. Dazu können \cob{ON SIZE ERROR} und \cob{ON OVERFLOW} genutzt werden.

\mintedCobol{ON-SIZE-ERROR.cbl}{Rudimentäre Fehlerbehandlung in COBOL}{exceptionsCobol}

\subsection{Nebenläufigkeit}

Als letzter wichtiger Punkt muss an dieser Stelle erwähnt werden, dass es in Java möglich und auch üblich ist, Programme nebenläufig zu entwickeln. Das heißt mehrere Threads arbeiten parallel und führen Verarbeitungen -- je nach Hardware nur scheinbar -- gleichzeitig aus. Dabei muss der Entwickler auf die Synchronisation von gemeinsam genutzten Speicherbereichen achten, um gültige Datenen zu gewehrleisten. Diese Nebenläufige Programmierung birgt zwar ein gewisses Fehlerpotential bei der Implementierung, sorgt jedoch dafür, dass Logik tendenziell effizienter ausgeführt wird.

In COBOL ist diese nebenläufige Ausführung nicht möglich. Ein COBOL-Programm führt Verarbeitungsschritte stets sequenziell aus und erlaubt keine parallelen Ausführungen. Über einen Transaktionsmonitor ist es jedoch teilweise möglich, dass verschiedene Programme gleichzeitig ausgeführt werden. Jedoch haben diese keine Kenntnis von anderen ausgeführten Programmen. Teilweise können verschiedene Programme auch die gleichen Speicherbereiche reservieren und so quasi miteinander arbeiten. Jedoch gibt es in COBOL keine Möglichkeit zur Synchronisation, weshalb dieses Vorgehen nur sehr selten beobachtet werden kann. Bei einem Transactionsmonitor handelt es sich um eine Art Middleware, vergleichbar zu Application-Servern in Java, welche Anfragen entgegennimmt, dafür sorgt, dass Ressourcen geöffnet und aufgeräumt werden, auf Hostsystemen Terminal-Masken zur Verfügung stellt und wie erwähnt entscheidet wie viele und welche Programme parallel ausgeführt werden. Ein Beispiel hierfür ist das \textit{Customer Information Control System} kurz \textit{CICS}. Um aus einem COBOL-Programm Teile des Transaktionsmonitor aufzurufen gibt es Befehle wie \cob{EXEC CICS}. %
\section{Dateien}
Wie in \autoref{schnittstellenDatenquellen} beschrieben wurde, stellen oftmals auch Dateien eine wichtige Datenressource dar. An dieser Stelle soll der Umgang -- das Lesen und Schreiben -- mit Dateien in Java und COBOL erläutert und gegenübergestellt werden.

Java bietet bereits mit Bibliotheksfunktionen des JDK umfangreiche Möglichkeiten, Dateien zu lesen und zu schreiben. Dies geschieht dabei in der Regel zeilenweise, wobei auch byte- \bzw zeichenweises Lesen möglich ist. Das Beispiel \autoref{javaIo} zeigt dabei zusätzlich die Verwendung der Klasse \jav{InputStream}. Diese sorgt dafür, dass die Datei nicht auf einmal in den Speicher geladen wird, sondern nur die gelesenen Daten im Speicher gehalten werden. Dies ist essenziell, um große Dateien zu lesen, da der Speicher des Systems ohne eine solche Methodik möglicherweise nicht ausreichend wäre, um die gesamte Datei zu speichern und so eine \jav{Exception} auftreten würde.

\mintedJava{FileInputOutput.java}{Datei-Ein- und Ausgabe in Java \cite{oracle_reading_}}{javaIo}

In Java sind darüberhinaus viele Bibliotheken erhältlich, die das Parsen von bestimmten, standardisierten Dateiformaten erleichtern können. Jedoch erfordert es ohne diese Biliotheken stets eigene Implementierungen, da das JDK an dieser Stelle nicht viel mehr als die gezeigten Abstraktionen und Möglichkeiten bietet.

In COBOL hingegen wird der Zugriff auf Dateien auf Basis sogenannter \textit{Records} bewerkstelligt. Dies entspricht weitestgehend dem gezeigten Java-Beispiel, jedoch hat der Entwickler hier die Möglichkeit zu definieren, wie eine Zeile der Datei aufgebaut ist. Dies setzt zwar voraus, dass der Dateiinhalt in einer Art Tabelle formatiert ist, sorgt allerdings dafür, dass kein Mehraufwand beim Parsen nötig ist. \autoref{cobolOutput} und \autoref{cobolInput} zeigen diese Verwendung der Datei-Ein- und Ausgabe in COBOL.

\mintedCobol{recordFile.txt}{Eingabedatei recordFile.txt}{recordFile}

\mintedCobol{PersonData.cpy}{Personendaten Copybook}{copyBookPersonData}

\mintedCobol{Files.cbl}{Lesen von Dateien in COBOL}{cobolOutput}

\mintedCobol{Files.cbl}{Schreiben von Dateien in COBOL}{cobolOutput}

\todo{complete section}
\section{Generische Programmierung}\label{generics}
Moderne Programmiersprachen wie Java oder C\# erlauben eine Programmierung mit sogenannten \textit{Generics}. Dabei handelt es sich um ein Konzept, bei dem Variablentypen generisch sein können, solange sie bestimmte Eigenschaften erfüllen. Diese Eigenschaften werden durch das Implementieren eines bestimmten Interfaces oder durch das Erben von einer bestimmten Klasse beschrieben. Dies soll \autoref{javaGenerics} verdeutlichen. Hieran wird auch deutlich, dass sowohl für Interfaces als auch für Oberklassen stets das Schlüsselwort \jav{extends} verwendet werden muss.

In diesem Beispiel werden die drei generischen Typen \textit{S}, \textit{T} und \textit{U} verwendet. \textit{T} muss dabei die Eigenschaft erfüllen, das \jav{Serializable} Interface zu implementieren. \textit{S} muss eine Unterklasse von \jav{Number} sein. Für \textit{U} hingegen wird keine bestimmte Eigenschaft definiert. Das bedeutet, dass an dieser Stelle jede Klasse, die von der Klasse \jav{Object} erbt -- in Java \textbf{jede} Klasse -- verwendet werden kann. 

Eine weitere Eigenschaft, die gezeigt werden soll, ist, dass es sowohl generische Klassen als auch generische Methoden geben kann. Generische Typen, die für eine Klasse definiert sind, stehen in der gesamten Klasse zur Verfügung, müssen jedoch bereits beim instanziieren festgelegt werden. Generische Methoden hingegen definieren generische Typen nur für den eigenen Scope.

\mintedJava{GenericsExample.java}{Generics in Java}{javaGenerics}

Dieses Konzept sorgt dafür, dass Algorithmen implementiert und als Bibliotheken bereitgestellt werden können, ohne Kenntnis über die tatsächlich verwendeten Datentypen zu haben, was die in \autoref{wiederverwendbarkeit} angesprochenen Modularisierungsmöglichkeiten unterstützt. Beispielsweise kann ein Sortieralgorithmus implementiert werden, welcher generische Objekte entgegennimmt, die das \jav{Comparable}-Interface -- ein Java Interface, welches dafür sorgt, dass zwei Objekte in eine Größenbeziehung gesetzt werden können -- implementieren. Folglich können mit diesem Algorithmus alle Objekte sortiert werden, die das Interface implementieren. Dieses Konzept ist in der objektorientierten Programmierung grundlegend und trägt maßgeblich zur Wiederverwendung und Kapselung bei.

COBOL hingegen unterstützt keine generischen Datentypen, was, wie in \autoref{wiederverwendbarkeit} gezeigt, deutlich weniger Möglichkeiten zur Wiederverwendung und Modularisierung mithilfe von Bibliotheken zur Folge hat.
\section{Konventionen}
Da bei Programmen oft eine Vielzahl von Entwicklern tätig ist und sich das Entwicklerteam auch über die Zeit ändern kann ist es nötig, dass alle Entwickler einen ähnlichen Stil verfolgen, sodass das Zurechtfinden in Code eines anderen Programmierers erleichtert wird. Diese Konventionen können zwar auch auf Projektebene festgelegt werden, jedoch entwickelen sich in Programmiersprachen oft Konventionen welche sich projekt-, personen- und unternehmensunabhängig etablieren. In diesem Kapitel werden die wichtisten dieser Konventionen beschrieben.

\subsection{Groß- und Kleinschreibung}
Bei der Programmierung behilft man sich oftmals der Groß- und Kleinschreibung, um ein höheres Maß an Struktur und Lesbarkeit des Codes zu erreichen.

Wichtig ist hierbei vorneweg zu beachten, dass Java \quotes{case-sensitive} ist, also zwischen Groß- und Kleinbuchstaben unterscheidet, während COBOL \quotes{case-insensitive} ist, also -- außer in Strings -- keinen Unterschied macht.
In Java ist es üblich Methodennamen und veränderbare Variablen mit einem kleinen Buchstaben beginnend zu benennen. Besteht der Name aus mehreren Wörtern so wird dieser im \quotes{camel-case} geschrieben. Das heißt, dass stets Großbuchstaben für den Beginn eines neuen Wortes verwendet werden. Beispiele hierfür wären \jav{getAdditionalData()} oder \jav{int currentAmountOfMoney}. Klassennamen werden in dem selben Muster geschrieben, beginnen jedoch mit einem Großbuchstaben: \jav{ToolBox}. Zu guter letzt sollten konstante Variablen und Werte innerhalb eines Aufzählungstyps durchgehend aus Großbuchstaben bestehen, wobei einzelne Wörter mit einem Unterstrich voneinander getrennt werden (\jav{final int MULIPLY_FACTOR = 2}). 

COBOL hingegen unterscheidet bei Variablennamen und Schlüsselwörtern nicht zwischen Groß- und Kleinschreibung. Es ist jedoch üblich sowohl Schlüsselwörter als auch Variablennamen komplett groß zu schreiben. Dies rührt daher, dass COBOL aus der Lochkartenzeit stammt, die meist lediglich Großbuchstaben im Zeichsatz hatten. Auch frühe Hostsysteme auf denen COBOL entwickelt und ausgeführt wurden, arbeiteten meist nur mit Großschreibung, sodass Programme auch über die Zeit hinweg nur so geschrieben wurden.


\subsection{Affixe} \label{affix}\label{affixCOBOL}
Ein weiteres wichtiges Werkzeug bei der Strukturierung von Programmcode ist das Versehen mit Affixen (Prä- oder Suffixen).

In Java ist es hierbei nicht ratsam Affixe zu verwenden. Da diese jedoch in einigen bestehende Codebasen Verwendung finden, werden an dieser Stelle die üblichsten behandelt.
Oftmals werden Interfaces in Java mit einem vorangestellten \quotes{I} gekennzeichnet. Diese Konvention sorgt jedoch dafür, dass Implementierungen eines Interfaces namentlich nicht immer klar abgegrenzt und definiert sind. Wird beispielsweise ein Interface \jav{IOutput} definiert so könnten valide Implementierungen \jav{ConsoleOutput} oder \jav{Printer-} \jav{Output} sein. Jedoch erlaubt dies namentlich auch die Interface-Implementierung namens \jav{Output}, bei der nicht ausreichend klar ist was der Zweck dieser Klasse ist.
Ein weiterer Codingstil der gelegentlich angewendet wird ist das Nutzen von \quotes{m} und \quotes{s} als Präfix von Variablen. Diese sollen kennzeichnen, dass eine Variable entweder Instanzvariable (\textbf{m}ember) oder statisch (\textbf{s}tatic) ist. Durch die Verwendung von modernen IDEs, die beide farblich unterschiedlich darstellen, und des Schlüsselwortes \jav{this}, welches exakt für Referenzen auf Instanzvariablen gedacht ist, werden Variablen jedoch bereits ausreichend gekennzeichnet, sodass der Code durch die Verwendung dieser Präfixe unnötigerweise schwerer lesbar gemacht wird.

In COBOL ist es in der Praxis dagegen sehr sinnvoll Affixe zu verwenden. Dadurch, dass es nicht möglich ist lokale Variablen zu definieren erlauben es Präfixe schnell und übersichtlich kenntlich zu machen, welche Variablen zu welchem Programmteil gehören. Dabei handelt es sich um eine Best-Practice-Methode, um den Code verständlicher und leichter lesbar zu machen, wie \textit{Herr Streit} im Fachinterview betonte, auch wenn in der Praxis oftmals darauf verzichtet wird.

So kann man den Namen einer Section oder eines Paragraph mit einem Präfix versehen und die darin genutzten Variablen mit demselbigen kennzeichnen. Beispielsweise sollte eine Variable \cob{PC-100-VALUE} nur in der Section \cob{PC-100-PROCESS} verwendet werden. 

Oft finden sich in bestehendem Code auch Präfixe welche die Art des Speichers -- \zB \cob{WS-} für \cob{WORKING-STORAGE} -- kennzeichnen. Eine Benennung nach den genutzten Programmteilen ist jedoch vorzuziehen und sorgt für klarere Struktur und Lesbarkeit.


\subsection{Schlüsselwörter}
In COBOL ist es möglich bestimmte Schlüsselwörter zu verkürzen. Dabei kann jedoch nicht beliebig gekürzt werden. Es sind lediglich weitere Schlüsselwörter mit selber Funktion definiert, die genutzt werden können. Ein Beispiel dafür ist die \cob{PICTURE}-Anweisung, die auch als \cob{PIC} geschrieben werden kann. Dabei handelt es sich zwar nicht um eine Konvention im eigentlichen Sinne, jedoch findet sich in der Praxis oftmals COBOL-Code welcher gekürzte Schlüsselwörter verwendet.

In Java ist ein verkürzen von Schlüsselwörtern hingegen nicht möglich.

\subsection{Formatierung des Quelltextes}

Neben den in \autoref{sec:structure} beschriebenen fest vorgegebenen strukturellen Eigenschaften von Java- und COBOL-Programmen, lassen sich auch Konventionen beschreiben, die das Formatieren des Quelltextes betreffen.

Für COBOL und Java gilt hierbei gleichermaßen, dass pro Zeile genau ein Statement stehen sollte. Alles innerhalb eines Blocks -- also eine oder mehrere untergeordnete Anweisungen -- werden konventionell um einen Tabulatorsprung eingerückt. Außerdem sollten zwischen Blöcken und Anweisungen Leerzeilen eingefügt werden, die die Lesbarkeit erhöhen. In COBOL ist jedoch bei der Einrückung das bereits beschriebene Spaltenprinzip zu beachten.
\section{Weitere Sprachkonzepte}
\section{Benannte Bedingungen -- Stufennummer 88}
Neben den bereits angesprochenen Stufennummern stellt die \textit{88} eine weitere Besonderheit in COBOL dar. Mit ihr ist es möglich einer Variable einen Wahrheitswert zuzuweisen, der von einem anderen Variablenwert abhängt. Es entsteht eine sogenannte benannte Begingung.

\begin{listing}[H]
  \inputminted{cobol}{Code/88_section.cbl.txt}
  \caption{Beispiel für COBOL Stufennummer 88}
  \label{88_cobol_listing}
\end{listing} 

\autoref{88_cobol_listing} zeigt die beispielhafte Verwendung der Stufennummer 88. Die Variable \mintinline{text}{VAR} kann dabei zweistellige numerische Werte enthalten. Liegt der Wert zwischen 0 und 9 (\mintinline{cobol}{VALUE 0 THRU 9}) so weist die Variable \mintinline{text}{ISLOWERTEN} den Wahrheitswert \mintinline{cobol}{TRUE} auf.

Der so entstandene Wahrheitswert kann folglich immer dann verwendet werden, wenn getestet werden soll, ob die Variable \mintinline{text}{VAR} im Bereich zwischen 0 und 9 liegt.

Zeile 18 des Programms zeigt einen weiteren Anwendungsfall der benannten Bedingungen. So lässt sich der Wert der eigentlichen Variable setzen, indem der bedingten Variable der Wahrheitswerte \mintinline{cobol}{TRUE} zugewiesen wird.

Die Ausgabe des Programms in \autoref{88_cobol_listing} wäre folgende:
\begin{minted}[bgcolor=hellgrau,xleftmargin=20pt,fontsize=\footnotesize]{text}
VAR is over 10.
VAR = 00
\end{minted}

Dieses Verhalten wird oftmals ausgenutzt um Variablen mit bestimmten Werten zu belegen. Dieses Verhalten beschreibt \autoref{88_cobol_value_set_listing}.

\begin{listing}[H]
  \inputminted{cobol}{Code/88_section_value_set.cbl.txt}
  \caption{Setzen von Werten mithilfe benannter Bedingungen}
  \label{88_cobol_value_set_listing}
\end{listing} 

\subsection*{Abbildung in Java}
Java besitzt kein Sprachkonstrukt, um die Funktionalität der Stufennummer 88 direkt nachzubilden. Eine Möglichkeit gleiches Verhalten darzustellen bietet allerdings die Implementierung spezieller getter- und setter-Methoden. Dies soll \autoref{88_java_listing} veranschaulichen.

\begin{listing}[H]
  \inputminted{java}{Code/88_section.java.txt}
  \caption{COBOL Stufennummer 88 in Java}
  \label{88_java_listing}
\end{listing} 

Die Methode \mintinline{text}{getISLOWERTEN} bezieht sich dabei nicht auf eine Variable sondern gibt einen Wahrheitswert in Abhängigkeit des Variablenwertes zurück. \mintinline{text}{setISLOWERTEN} setzt diesen Variablenwert wenn der Wahrheitswert \mintinline{java}{true} ist. Diese Einschränkung in \mintinline{text}{setIS-} \mintinline{text}{LOWERTEN} wurde hinzugefügt, um das Verhalten der meisten COBOL-Compiler nachzubilden, welche nur das Setzen des Wertes \mintinline{cobol}{TRUE} erlauben.

\begin{listing}[H]
  \inputminted{java}{Code/88_section_value_set.java.txt}
  \caption{Setzen eines konstanten Wertes mit benannter Variable in Java}
  \label{88_java_value_set_listing}
\end{listing} 

Der Anwendungsfall, dass eine benannte Bedingung in COBOL verwendet wird um bestimmte Werte zu setzen, lässt sich in Java in verschiedenen Weisen abbilden. \autoref{88_java_value_set_listing} zeigt die wohl einfachste dieser Möglichkeiten.

\begin{listing}[H]
  \inputminted{java}{Code/88_section_value_set2.java.txt}
  \caption{Exakte Abbildung des Setzens einer benannten Bedingung in Java}
  \label{88_java_value_set_listing2}
\end{listing} 

Dabei werden drei statische Konstanten definiert und der Variable \mintinline{text}{errorMessage} die gewünschte zugewiesen. Dieses Konstrukt bildet zwar nicht 1:1 den COBOL-Code nach, in dem es auch möglich wäre Konstanten zu verwenden, jedoch zeigt \autoref{88_java_value_set_listing2}, dass die vorhergehende Lösung in Java eleganter ist.%
\section{Mehrfachverzweigungen}
Ein wichtiges Konstrukt um den Programmfluß eines Programms zu steuern sind Mehrfachverzweigungen. Obwohl sowohl Java als auch COBOL Mehrfachverzweigungen bieten, sind diese doch leicht unterschiedlich zu verwenden. Im Folgenden sollen verschiedene Verwendungsmöglichkeiten der jeweiligen Konstrukte dargestellt werden.

In Java bildet das \mintinline{java}{switch}-\mintinline{java}{case}-Konstrukt eine Mehrfachverzweigung ab. \autoref{switch_case_java} zeigt dabei die wichtigsten Verwendungsmöglickeiten.

\begin{listing}[H]
  \inputminted{java}{Code/switch_case.java}
  \caption{Mehrfachverzweigungen in Java}
  \label{switch_case_java}
\end{listing} 

Zum einen ist zu beachten, dass Werte nur mit Literalen und Konstanten verglichen werden können. Ein Vergleich einer Variablen mit einer weiteren ist hierbei nicht zulässig, solange diese nicht als \mintinline{java}{final} deklariert ist. Auch der Vergleich auf bestimmte Wertebreiche ist nicht zulässig. 

Jedoch ist ein gewolltes \quotes{Durchfallen} möglich um bei verschiedenen Werten die gleichen Programmzweige zu durchlaufen. Dabei Spielt das Schlüsselwort \mintinline{java}{break} eine entscheidende Rolle. Wird kein \mintinline{java}{break} am Ende einer \mintinline{java}{case}-Anweisung verwendet, so wird automatisch in den ausführbaren Block der darauffolgenden \mintinline{java}{case}-Anweisung gesprungen. Dies zeigt sich in den Zeilen 15 und 16 von \autoref{switch_case_java}. Die Verwedung der \mintinline{java}{break}-Anweisung wird hingegen in Zeile 13 genutzt um den \mintinline{java}{switch}-Block zu verlassen. 

Das Fehlen eines \mintinline{java}{break} kann in der Praxis schnell zu unerwünschtem und unerklärlichem Verhalten führen. Deshalb folgt in der Regel jedem \mintinline{java}{case} ein \mintinline{java}{break}. 

Das Pendant in COBOL stellt das Schlüsselwort \mintinline{cobol}{EVALUATE} dar. Wenngleich es den gleichen Sinn wie das \mintinline{java}{switch}-\mintinline{java}{case}-Konstrukt in Java erfüllen soll, ist es vielseitiger einsetzbar wie die folgenden Beispiele illustrieren sollen.

\begin{listing}[H]
  \inputminted{cobol}{Code/switch_case.cbl}
  \caption{Mehrfachverzweigungen in COBOL}
  \label{switch_case_cobol1}
\end{listing} 

\autoref{switch_case_cobol1} beinhaltet dabei einige gängige Verwendungen der \mintinline{cobol}{EVALUATE}-Klausel.%
\subsection{Speicherausrichtung}
Dieser Abschnitt soll einen kurzen Abriss über die COBOL Stufennummer \textit{77} geben. Mit der Stufennummer 77 deklarierte Daten haben folgende beiden Eigenschaften:

\begin{itemize}
    \item Die Variable kann nicht weiter unter gruppiert werden.
    \item Die Variable wird an festen Grenzen des Speichers ausgerichtet.
\end{itemize}

Während die erste erwähnte Eigenschaft wenig Bewandtnis in der Praxis hat, war es früher nötig, Variablen für bestimmte Instruktionen an festen Speichergrenzen auszurichten. Üblicherweise mussten die Adressen dieser Grenzen je nach Compiler ganzzahlig durch 4 \bzw 8 teilbar sein. Dieses Verhalten ist heutzutage jedoch nicht mehr nötig. Dies und die Tatsache, dass durch forcierte Speicherausrichtung Speicherbereiche zwischen Daten mit Stufennummer 77 ungenutzt, aber reserviert bleiben, führen dazu, dass diese Stufennummer nicht mehr genutzt werden sollte.

In Java gibt es kein vergleichbares Konzept, da die Speicherverwaltung komplett abstrahiert ist und dem Entwickler nicht ermöglicht wird, direkten Einfluss darauf zu nehmen.%

\subsection{Reorganisation von Daten}
Dieser Abschnitt beschreibt die COBOL Stufennummer \textit{66}, die eine besondere Rolle spielt.

In \autoref{cobol66} wird die Stufennummer 66 in Verbindung mit der \cob{RENAMES}-Anweisung, was zwingend erforderlich ist, verwendet, um Teile der Personendaten neu zu gruppieren. Dies geschieht durch die Verwendung des \cob{THRU}-Schlüsselworts. Ohne die Angabe dieses Bereichs können auch einzelne Variablen umbenannt werden. Hierbei wird lediglich eine neue Referenz auf den selben Speicherbereich erstellt, nicht jedoch neuer Speicher alloziert.

\begin{codeWithCaption}{Stufennummer 66 und RENAMES-Befehl}{cobol66}
\cobol{66_section.cbl}
\begin{shellwindow}
Max       Mustermann
Musterstraße  7a   12345 Musterstadt   
\end{shellwindow}
\end{codeWithCaption}

Diese Stufennummer wird in der Praxis selten verwendet und auch ist in Java zu dieser Stufennummer kein exaktes Pendant zu finden. 
\subsubsection*{Abbildung in Java}
Die Gruppierung von Daten erfolgt in Java in eigenen Klassen, aus denen sich wiederum andere Objekte zusammensetzen können. Diese Aggregationsbeziehung ist im Diagramm in \autoref{javaAggregationClasses} dargestellt. 

\begin{figure}[H]
    \centering
    \begin{tikzpicture}
        \umlclass[y=0, x=0]{Person}{name : Name\\address : Address}{}
        \umlclass[y=0, x=-5]{Name}{firstName : String\\surname : String \\}{}
        \umlclass[y=0,x=5]{Address}{street : String\\houseNumber : String\\zipCode : String\\city : String}{}
        \umlaggreg{Person}{Name}
        \umlaggreg{Person}{Address}
    \end{tikzpicture}
    \caption{UML-Diagramm einer Aggregation}
    \label{javaAggregationClasses}
\end{figure}
Die Verwendung der Stufennummer 66 als reines Umbenennen eines Datums entfällt in Java, da in diesem Fall neue Variablen mit anderen Namen deklariert werden können, welchen der ursprüngliche Wert zugewiesen wird.
%
\subsection{Implizierte Variablennamen}
Dieser Abschnitt behandelt einen Fehler, der typischerweise zu Beginn in der Softwareentwicklung beobachtet werden kann. \autoref{ifVariableErrorJava} zeigt, wie Anfänger häufig versuchen, logische Ausdrücke zu konstruieren. Dies entspricht dem intuitiven Gedanken \quotes{Wenn Variable X größer 0 und kleiner 5 ist, dann \dots} oder der mathematischen Definition \quotes{$0 < X < 5$}.

\begin{codeWithCaption}{Keine implizierten Variablennamen in logischen Ausdrücken in Java}{ifVariableErrorJava}
\java{IfVariableError.java}
\begin{shellwindow}
$ javac -Xmaxerrs 3 IfVariableError.java 
IfVariableError.java:4: error: > expected
        if (System.currentTimeMillis() > 0 && < Long.MAX_VALUE) {
                                                              ^
IfVariableError.java:4: error: ')' expected
        if (System.currentTimeMillis() > 0 && < Long.MAX_VALUE) {
                                                               ^
IfVariableError.java:8: error: illegal start of type
        if (0 < System.currentTimeMillis() < Long.MAX_VALUE) {
        ^
3 errors
\end{shellwindow}
\end{codeWithCaption}

Versucht man, \autoref{ifVariableErrorJava} zu kompilieren, treten einige Fehler auf. In Java können nur vollständige logische Ausdrücke mit logischen Operatoren verknüpft werden. Zum anderen kann innerhalb eines logischen Ausdrucks lediglich maximal einmal ein Vergleichsoperator verwendet werden. 

\begin{codeWithCaption}{Verwendung von Variablennamen in logischen Ausdrücken in Java}{ifVariableNoErrorJava}
\java{IfVariableNoError.java}
\begin{shellwindow}
$ javac IfVariableNoError.java 
$ java IfVariableNoError
We get here everytime!
\end{shellwindow}
\end{codeWithCaption}

\autoref{ifVariableNoErrorJava} demonstriert eine funktionsfähige Implementierung des vorhergehenden Beispiels. Dieser Code kann fehlerfrei kompiliert und ausgeführt werden wie die Ausgabe zeigt.

\subsubsection*{Implizierte Variablennamen in COBOL}
In COBOL hingegen ist das Schreiben von logischen Ausdrücken mit implizierten Variablennamen möglich.

\mintedCobol{implicit_variable_names.cbl}{Implizierte Variablennamen in COBOL}{implicitNamesCobol}

\autoref{implicitNamesCobol} stellt in Zeile 11 die Verwendung von implizierten Variablennamen in COBOL dar, die im Gegensatz zu Java möglich ist.%
		% COBOL-Pattern in Java -------------------------------
			\chapter{COBOL-Pattern in Java} 
\label{ch:cobolinjava}
		% Schluss -------------------------------
					\chapter{Fazit (Sprechender Name je nach Ergebnissen)}
		\label{ch:chap6}
			\chapter*{Zusammenfassung} \addcontentsline{toc}{chapter}{Zusammenfassung}

Da die \quotes{\textbf{Co}mmon \textbf{B}usiness \textbf{O}riented \textbf{L}anguage}, kurz \mbox{COBOL} genannt, bereits Ende der 1950er Jahre entstand und daher nur wenige moderne Sprachkonzepte bietet, wird der Fokus in der Ausbildung neuer Informatiker immer mehr auf Programmiersprachen mit moderneren objektorientierten Konzepten gelegt. Dem steht gegenüber, dass COBOL immer noch wichtiger Bestandteil bestehender betrieblicher Informationssysteme ist, die es zu warten und zu erweitern gilt. Während diese Sprache heutzutage also zunehmend seltener Teil der Ausbildung von Programmierern ist, besteht, durch die Vielzahl vorhandener COBOL-Systeme, weiterhin eine hohe Nachfrage nach Experten.

Diese Arbeit gibt einen generellen Überblick über Herausforderungen, die sich in Verbindung mit betrieblichen Informationssystemen ergeben, und zeigt, wie COBOL und Java diesen Problemen begegnen. Ferner wird COBOL konzeptuell erfasst und mit Java, als Vertreter moderner Sprachen, verglichen. Dabei steht stets die praktische Anwendung der Sprachen im Vordergrund, weshalb Experteninterviews geführt wurden, um neben bestehender Fachliteratur bestmögliche Einsicht in die Entwicklung und Wartung angesprochener Systeme zu erhalten. Damit entstand ein Leitfaden, der es Programmierern mit Java-Kenntnissen erlaubt, sich mit COBOL vertraut zu machen, indem bekannte Konzepte, Muster und Konstrukte gegenübergestellt werden. Zusätzlich wird, als Ergebnis der Experteninterviews, darauf hingewiesen, wie sich der Umgang mit diesen Konzepten in der Praxis gestaltet und gestalten sollte. 
	
	% Seitennummerierung: R?mische Ziffern; Zur?cksetzen auf in RomanSiteCounter gespeicherten Wert.
	\pagebreak
	\pagenumbering{Roman}
	\setcounter{page}{\value{RomanSiteCounter}}
	
	% Literatur ---------------------------------
	\nocite{*}
	\ohead{\textit{Literaturverzeichnis}}
	\singlespacing
	\printbibliography
	\pagebreak
	
	% Anh?ge -----------------------------------
	%\ohead{\textit{Anhang}}
	%\input{Anhang/Anhang}

	\newgeometry{left=1cm,right=1cm,top=1cm,bottom=1cm,includeheadfoot}
\begin{landscape}
    \chapter{Zeitplan}
    \includegraphics[width=.9\linewidth]{Bilder/Zeitplan}
\end{landscape}
\restoregeometry
	
\end{document}
