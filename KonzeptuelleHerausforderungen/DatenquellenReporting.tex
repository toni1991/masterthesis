\section{Datenquellen}\mylabel{reporting}

Kernaspekte von Informationssystemen sind die Daten-Ein- und Ausgabe. Neben Dateien, deren Verwendung später in \autoref{dateien} beschrieben wird, und Dateisystemen, stellen dabei Datenbanken eine der gebräuchlichsten Arten der Datenhaltung dar.

Um Zugriff auf Daten innerhalb einer Datenbank zu erhalten, bietet Java mit der \quotes{Java Database Connectivity} API standardisierte Mechanismen, um auf Datenbanken zuzugreifen. Darauf aufbauend existieren Abstraktionsschichten wie \textit{Hibernate}\footnote{\url{http://hibernate.org/} \visitedOn}, die eine Abbildung von Entitäten der Datenbank auf Java-Klassen ermöglichen.

In COBOL können Datenbankabfragen mithilfe des \cob{EXEC SQL}-Befehls erfolgen. Damit lässt sich der betreffende SQL-Code direkt innerhalb des COBOL-Codes schreiben und Werte aus Variablen nutzen \bzw in diese schreiben. Diese Einbettung des SQL-Codes in COBOL macht es, im Gegensatz zu Java, möglich, dass bereits der Compiler die Syntax der SQL-Abfragen überprüfen kann. Dies ist mit \textit{Language Integrated Query}s in C\# zu vergleichen. Herr Bonev bemerkte hierzu, dass diese Datenbankabfragen oft Optimierungsmöglichkeiten bieten, da oft zu beobachten sei, dass Entwickler auf einfache Datenbankabfragen zurückgreifen und eine eventuelle Datenfilterung erst später innerhalb des Programms geschehe. Diese Filterung ist jedoch auch direkt durch eine geeignete Datenbankabfrage zu erreichen, was sehr viel performanter sei. 

Im Hinblick auf Datenquellen machen betriebliche Informationssysteme mit COBOL oftmals vom \textit{Eingabe-Verarbeitung-Ausgabe}-Prinzip, kurz EVA-Prinzip, Gebrauch. Dabei wird ein Arbeitsvorgang in verschiedene Schritte unterteilt, die nacheinander ausgeführt werden. Am Ende jedes Schrittes steht die Ausgabe der (Zwischen-)Ergebnisse, sodass der darauffolgende diese weiterverarbeiten kann. Um diese einzelnen Schritte zu steuern behilft man sich der \textit{Job Control Language}, kurz \textit{JCL}. Diese Skriptsprache steuert die angesprochenen Stapelverbeitungsvorgänge, \engl Batch-Jobs, und sorgt für die Ein- und Ausgabe von Daten, auch \textit{Reporting} genannt. 

Durch das EVA-Prinzip und die Tatsache, dass Host-Computer über sehr stark optimierte Datei-Ein- und Ausgabe verfügen, seien Datenbanken, laut Herrn Lamperstorfer, in COBOL weniger relevant.