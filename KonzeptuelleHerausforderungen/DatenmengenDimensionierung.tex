\section{Datenmengen und Dimensionierung}

Betriebliche Informationssysteme sind in der Regel dafür konzipiert, große Datenmengen zu verarbeiten, die \idR mit der Betriebszeit des Systems weiter zunehmen. Daher ist bereits bei der Planung wichtig, den späteren Datenumfang so abzuschätzen, dass nachträgliche Erweiterungen durch möglichst wenig Programmieraufwand zu bewerkstelligen sind.

Die Dimensionierung von Datenstrukturen nimmt in Java eine untergeordnete Rolle ein, da dynamisch Speicher alloziert werden kann, wodurch sich die Größe von Datenstrukturen dynamisch erweitern lässt, wie \autoref{datenstrukturen} genauer beschreibt. Um ein hohes Datenaufkommen in adäquater Zeit bewältigen zu können, spielt in Java neben der Algorithmik auch parallele Verarbeitung eine vorrangige Rolle. Ebenso beeinflusst oftmals die Plattform, auf der ein Java-System betrieben wird, wie sich die Performanz des Systems gestaltet.

In COBOL sind Vorüberlegungen zur Dimensionierung eines Systems weitaus wichtiger als in Java. Da COBOL, wie später in dieser Arbeit beschrieben, keine dynamischen Datenstrukturen bietet, muss bereits zu Beginn sehr genau überdacht werden, wie viele Daten ein System später be- und verarbeiten muss. Die befragten Experten gaben an, dass  Wartungsaufträge teilweise lediglich damit zu tun haben, dass Datenstrukturen -- beispielsweise Arrays oder Strings -- zu klein dimensioniert sind und künftig mehr oder längere Datensätze aufnehmen sollen. Herr Streit bestätigte dies durch ein Beispiel aus der Praxis, bei dem eine vierstellige Nummer nicht mehr ausreichend war, um Partnerunternehmen zu identifizieren. \quotes{Gelöst wurde das dann [...] durch das Zulassen von Buchstaben, weil so der Speicherbedarf nicht erhöht wurde und sich Datenstrukturen im Speicher nicht verschoben haben.} Dies illustriert einen wichtigen Aspekt von Datenstrukturen in COBOL, der später genauer beleuchtet wird: Obwohl solche Anpassungen, im Gegensatz zu Java, in COBOL vorkommen, muss stets bedacht werden, wo Variablen im Speicher liegen, da andere Daten- und Dateidefinitionen von dem bestehenden Aufbau der Datenstrukturen ausgehen.