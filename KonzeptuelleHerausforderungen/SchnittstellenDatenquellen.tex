\section{Schnittstellen und Datenquellen}\label{schnittstellenDatenquellen}
\todo{datenquellen, datenbanken}
\todo{3. Batchbetrieb -> Schnittstellen (Host nicht COBOL)}

In betrieblichen Informationssystemen stellen außerdem Schnittstellen ein wichtiges Thema dar. Sowohl das Bereitstellen von standardisierten und dokumentierten Interfaces als auch das Nutzen von anderen Systemen über ihre Schnittstellen ist stets Teil aller Anwendungsfälle. 

Vor allem in der heutigen Zeit, in der Informationssysteme nicht mehr als alleinige Verarbeitungs-, Reporting- und Darstellungsschicht fungieren, sondern eingebettet in einen größeren Kontext aus verschiedensten Modulen, mobilen Applikationen und Websites funktionieren sollen und mit diesen kommunizieren müssen ist ein ausgereiftes Schnittstellenkonzept und die standardisierte Bereitstellung und Nutzung von Daten und Diensten unerlässlich.

Für Java sind an dieser Stelle eine Fülle an Bibliotheken erhältlich, welche Netzwerkkommunikation über verschiedenste Protokolle auf unterschiedlichen Ebenen ermöglichen. Neben diversen Fremdbibliotheken, bietet bereits das JDK bereits unterschiedliche Methoden zur Kommunikation mit Fremdsystemen.

Zudem ist es, durch die Definition von Interfaces, auch auf Klassenebene möglich, Schnittstellen zu bieten, die eine einfache Erweiterung von und Verbindungen zu Neusystemen möglich machen.

COBOL hingegen lässt an dieser Stelle einige Funktionalität vermissen. Durch das fehlende Bibliothekskonzept -- wie in \autoref{wiederverwendbarkeit} erläutert -- und das gänzliche Fehlen von Netzwerkkommunikationsmechanismen ist es, in reinem COBOL, nicht möglich Netzwerkschnittstellen festzulegen die von außen erreichbar sind oder solche zu nutzen. Auch intern kann ein COBOL-System nur bedingt Standards definieren die zwischen unterschiedlichen Programmteilen für einheitliche Kommunikationskanäle sorgen. COBOL-Systeme basieren, wie \autoref{reporting} beschreibt, auf einfachen EVA-Pinzipien.

Durch den Wandel der letzten Jahre bzw. Jahrzehnte in hochdimensionalen und komplexen Softwaresystemen wird es immer schwerer reine COBOL-Systeme zu nutzen und sinnvoll in eine heterogene IT-Landschaft einzubetten. Kunden sind von Applikationen für verschiedenste Systeme abhängig und müssen sich darauf verlassen können, dass die Erweiterung um neue Back- und Frontends, ohne Einfluss auf bestehende Komponenten vonstatten gehen kann. Auch ist es nötig die Basis für interne Systemerweiterungen zu schaffen, indem festgelegte Interfaces bedient und verwendet werden. Diese Anforderungen können mit modernen Sprachen wie Java mühelos erreicht werden, wohingegen Altsysteme nur spärliche Möglichkeiten in dieser Richtung bieten.