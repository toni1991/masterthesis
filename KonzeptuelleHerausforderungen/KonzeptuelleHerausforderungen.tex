\chapter{Herausforderungen für COBOL und Java in betrieblichen Informationssystemen}
Bei der Entwicklung von betrieblichen Informationssystemen sehen sich Entwickler mit grundlegenden Fragen und Anforderungen an die einzusetzenden Technologien und Programmiersprachen konfrontiert.

Dieses Kapitel gibt einen Überblick über die wichtigsten Entscheidungskriterien für die Herangehensweise und zeigt auf, welchen Herausforderungen sich Programmiersprachen -- im Speziellen COBOL und Java -- in diesen Informationssystemen stellen müssen.

\label{ch:herausforderungen}
    \section{Datenmengen und Dimensionierung}

Betriebliche Informationssysteme sind in der Regel dafür konzipiert, große Datenmengen zu verarbeiten, die \idR mit der Betriebszeit des Systems weiter zunehmen. Daher ist bereits bei der Planung wichtig, den späteren Datenumfang so abzuschätzen, dass nachträgliche Erweiterungen durch möglichst wenig Programmieraufwand zu bewerkstelligen sind.

Die Dimensionierung von Datenstrukturen nimmt in Java eine untergeordnete Rolle ein, da dynamisch Speicher alloziert werden kann, wodurch sich die Größe von Datenstrukturen dynamisch erweitern lässt, wie \autoref{datenstrukturen} genauer beschreibt. Um ein hohes Datenaufkommen in adäquater Zeit bewältigen zu können, spielt in Java neben der Algorithmik auch parallele Verarbeitung eine vorrangige Rolle. Ebenso beeinflusst oftmals die Plattform, auf der ein Java-System betrieben wird, wie sich die Performanz des Systems gestaltet.

Wenn Daten gleichzeitig im Speicher gehalten werden müssen, sind in COBOL Vorüberlegungen zur Dimensionierung eines Systems und der darin genutzten Datenstrukturen weitaus wichtiger als in Java. Da COBOL, wie später in dieser Arbeit beschrieben, keine dynamischen Datenstrukturen bietet, muss bereits zu Beginn sehr genau überdacht werden, wie viele Daten ein System später gleichzeitig be- und verarbeiten soll. Die befragten Experten gaben an, dass Wartungsaufträge teilweise lediglich damit zu tun haben, dass Datenstrukturen -- beispielsweise Arrays oder Strings -- zu klein dimensioniert sind und künftig mehr oder längere Datensätze aufnehmen sollen. Herr Streit bestätigte dies durch ein Beispiel aus der Praxis, bei dem eine vierstellige Nummer nicht mehr ausreichend war, um Partnerunternehmen zu identifizieren. \quotes{Gelöst wurde das dann [...] durch das Zulassen von Buchstaben, weil so der Speicherbedarf nicht erhöht wurde und sich Datenstrukturen im Speicher nicht verschoben haben.} Dies illustriert einen wichtigen Aspekt von Datenstrukturen in COBOL, der später genauer beleuchtet wird: Obwohl solche Anpassungen, im Gegensatz zu Java, in COBOL vorkommen, muss stets bedacht werden, wo Variablen im Speicher liegen, da andere Daten- und Dateidefinitionen von dem bestehenden Aufbau der Datenstrukturen ausgehen.
    \section{Langlebigkeit, Wartbarkeit und Verlässlichkeit} \label{verlaesslichkeit}

Durch ihre hohe Komplexität werden betriebliche Informationssysteme \idR über viele Jahre oder sogar Jahrzehnte betrieben und dabei gewartet, erweitert und angepasst. Um die Langlebigkeit, Wartbarkeit und Verlässlichkeit solcher Systeme sicherzustellen, werden diese in der Praxis mehr oder weniger umfangreichen Tests unterzogen. Damit kann beispielsweise sichergestellt werden, dass bestehender Code auch nach Erweiterungen weiterhin funktioniert. Eine der wichtigsten Techniken hierbei sind sogenannte Unit-Tests. Dabei werden einzelne isolierte Einheiten getestet und so Einflüsse von anderen Programmteilen minimiert. 

In Java gibt es einige Frameworks, beispielsweise \textit{JUnit}\footnote{\url{https://junit.org/} \visitedOn} oder \textit{Mockito}\footnote{\url{http://site.mockito.org/} \visitedOn}, die das Testen direkt oder indirekt unterstützen. Die Wartbarkeit begünstigt in Java zusätzlich das relativ einfache Durchführen von Refactorings. So können Programme mithilfe von modernen Entwicklungsumgebungen teilweise neu geschrieben \bzw strukturiert werden und bestehender Code so im Zuge von Erweiterungen verbessert werden. Auch hierbei sind Tests zur Überprüfung der Korrektheit von Vorteil. 

COBOL bietet zum Testen weitaus weniger Möglichkeiten. Entwicklung, Testen und Debuggen direkt am Host sind teuer, da die Kosten eines Mainframes oft nach Rechenzeit berechnet werden, und daher wurden, laut Herrn Lamperstorfer vor allem in frühen COBOL-Systemen, oftmals funktionierende und manuell getestete \quotes{Schablonen} beispielsweise für die Dateiverarbeitung zur Erweiterung eines Systems wiederverwendet, sodass diese nur minimal angepasst werden mussten. Auch Refactorings nehme man in COBOL-Systemen tendenziell selten vor, da diese ein umfangreiches Testen erfordern würden.

In puncto Verlässlichkeit können sich COBOL-Anwendungen jedoch oftmals auf gut isolierte Infrastrukturen verlassen, die durch ihre Homogenität, wenig Fortentwicklung und eingebaute Ausfallsicherheitsmaßnahmen eine zuverlässige Basis bieten. Java-Systeme hingegen werden auf vielen unterschiedlichen Plattformen betrieben, die sich sehr schnelllebig verändern, wodurch zusätzliches Augenmerk auf die Sicherung der oben genannten Eigenschaften gelegt werden muss.
    \section{Modularisierung, Wiederverwendbarkeit \& Variabilität}\label{wiederverwendbarkeit}
Sehr wichtige Punkte bei der Entwicklung von betrieblichen Informationssystemen sind die Modularisierung und Wiederverwendbarkeit. Um ein System für die Zukunft wart- und erweiterbar zu machen ist eine gewisse Modularisierung anzustreben. Code muss somit nicht mehrmals geschrieben werden, was auch das spätere Einarbeiten in ein Projekt erleichtert, da der Projektumfang deutlich verringert werden kann. 

Zudem ist, sei es um um Beispiel verschiedene Mandanten, Tarife oder Sparten abzubilden, die im Grunde die selbe Logik beinhalten, in betrieblichen Informationssystemen häufig eine gewisse Variabilität gefordert. Auch diese kann durch Wiederverwendbarkeit und Modularisierung stark begünstigt werden.

\subsection*{Java}
Java ist eine hoch modulare Sprache. Alleine objektorientierte Paradigmen wie Kapselung, Polymorphie oder Aggregation/Komposition sorgen dafür, dass Code in hohem Maße wiederverwendet werden kann. Dabei ist vor allem die Gliederung in Klassen und Funktionen (siehe \autoref{sec:functionsAndReturnValues}) ausschlaggebend. Des weiteren können Bibliotheken als Java Archive (kurz \mintinline{java}{jar} genannt) dis­tri­bu­ie­rt werden und in anderen Projekten wiederverwendet werden. Dieses Konzept nutzt auch die Programmiersprache an sich bereits in hohem Maße aus und so werden viele Funktionalitäten über Packages (siehe \autoref{structure}) bereitgestellt. Die am häufigsten gebrauchten Bibliotheken sind dabei \mintinline{java}{java.util}, welche grundlegende Datenstrukturen wie zum Beipiel Listen (siehe \autoref{lists}) bereitstellt, \mintinline{java}{java.io} welche Datenein- und Ausgabe ermöglicht und allen voran \mintinline{java}{java.lang} welche -- wie der Name bereits andeutet -- Ergänzungen zu Programmiersprachlichen Mitteln liefert. 

Durch diese praktischen Modularisierungsmöglichkeiten ist es in Java auch gut möglich Variabilität zu erreichen. So kann bestehende Logik wiederverwendet oder beispielsweise durch Vererbung minimal angepasst und nachträglich erweitert werden und sorgt dafür, dass Wartungen am System die sich auf Erweiterungen des Umfangs beziehen -- z.B. das Einführen eines neuen Tarifs -- mit verhältnismäßig geringem Aufwand umgesetzt werden können.

Ein weiterer Punkt der Java zu einer Sprache macht, die dafür sorgt, dass Programme wiederverwendet werden können ist die Tatsache, dass Java in plattformunabhängigen Byte-Code übersetzt wird. Die Java-Virtual-Machine (\textit{JVM}) führt dann diesen Byte-Code aus und sorgt so dafür, dass bereits kompilierte Programme auf allen Systemen mit JVM ausführbar sind und so weiterverteilt werden können ohne neu kompiliert werden zu müssen. Diese JVM wiederum ist ein plattformabhänigiges System, welches jedoch für -- nahezu -- alle gängigen Systeme und Plattformen verfügbar ist.

\subsection*{COBOL}
Im Gegensatz zu Java lässt COBOL ein Modularisierungskonzept vermissen. Wie in \autoref{sec:functionsAndReturnValues} nachzulesen ist, fehlen grundlegende Spracheigenschaften um die Wiederverwendbarkeit von Code sicherzustellen. 

Wie ein Herr Lamperstorfer betonte, sieht man daher in der Praxis oftmals Code-Blöcke die ein und die selbe Logik abbilden, aber durch die Verwendung von anderen Daten nochmals im Copy-Paste-Stil in den Code integriert wurden. Das sorgt für ein hohes Maß an Redundanz. Um diese Redundanz zu vermeiden werden aber auch gängigerweise Datenstrukturen für mehr als nur einen Zweck im Programm \quotes{missbraucht}. Darunter leidet natürlich die Les- und Wartbarkeit von COBOL-Code sehr, da häufig nicht klar ist welche Daten, in welchem Kontext, wie verwendet werden. Zu diesem Thema sei auf \autoref{affixCOBOL} verwiesen.

In COBOL kann Variabilität im Vergleich zu Java nur sehr schwer erreicht werden. Code muss oftmals in hohem Maße kopiert werden um ähnliche Funktionalität abzubilden und so fallen Anpassungen in diesem Bereich unverhältnismäßig groß aus.

Auch ein Bibliothekskonzept ist in COBOL nicht vorhanden. So werden Programme und aufgerufene Unterprogramme beim Kompilieren statisch zu einer ausführbaren Einheit gelinkt. Um dieses Verhalten zumindest soweit zu beeinflussen, dass dynamisch geladenen Unterprogramme entstehen kann als \quotes{Trick} eine Variable eingeführt werden, welche den Namen des Unterprogramms enthält. Wird nun das Programm aufgerufen, welches in dieser Variable definiert ist und nicht in einer festen Zeichkette definiert ist, nimmt der Compiler an, dass das geladene Unterprogramm variieren kann -- auch wenn der Inhalt der Variablen nicht verändert wird -- und vermeidet so ein statisches Linken.

Herr Streit merkte an dieser Stelle an, dass in der Praxis selten die Nutzung von sogenannten 

Zwar unterstützt COBOL in neueren Standards und Compilern eine objektorientierte Entwickung, jedoch ist diese Spracherweiterung in der Praxis irrelevant. Die meisten gängigen Systeme auf denen COBOL Programme betrieben werden verfügen nicht über derartig neue Compiler und auch bei der Verwendung merkt man, dass diese Konzepte nachträglich hinzugefügt wurden und eigentlich nicht Bestandteil der Sprache sind. Hat man das Glück ein System mit einem kompatiblen Compiler zu haben, so bleibt als weiterer Stolperstein der Fakt, dass die ohnehin raren COBOL-Entwickler nicht mit der Verwendung von objektorientierter Entwicklung firm sind. Daher wird diese Spracherweiterung in der vorliegenden Arbeit nicht behandelt. 
    \section{Darstellungsgenauigkeit -- Fließ- und Festkommaarithmetik}
Vorallem in betrieblichen Informationssystemen -- die oftmals Geldbeträge durch eine gewisse Anzahl von Rechenschritten errechnen sollen -- ist es unerlässlich einen Blick auf die Rechengenauigkeit des Systems und der verwendeten Sprachen zu werfen. Diese ist oftmals eine Folge der Speicherrepräsentation -- irrationale Zahlen können durch endlichen Speicherbedarf nicht abgebildet werden -- rationaler Zahlen, die erheblichen Einfluss auf den Darstellungsbereich hat. Man unterscheidet grundsätzlich zwischen Speicherungen in Fließ- und Festkomma-Darstellung.
 
\subsection*{Fließkommaarithmetik}
In modernen Programmiersprachen wie Java werden Datentypen für rationale Zahlen in der Fließkommarepräsentation gespeichert. Daher auch der Name \mintinline{java}{float} für engl. \quotes{floating point}. Diese Darstellung hat den großen Vorteil, dass sowohl kleine Zahlen die gegen Null gehen als auch sehr große Zahlen mit dem gleichen Speicherbedarf dargestellt werden können, da quasi das Dezimaltrennzeichen verschoben werden kann. Java verwendet zur Dartellung standardmäßig den Datentypen \mintinline{java}{double}, also ein \mintinline{java}{float} mit doppelter Darstellungsgenauigkeit bzw. doppeltem Speicherbedarf. Somit wird es möglich als kleinsten Absolutwert $2^{-1074}$ und als größten $(2 - 2^{-52}) \cdot 2^{1023}$ darzustellen.

Dabei werden Zahlen nach \textit{IEEE 754}-Standard in Vorzeichen, Exponent und Mantisse umgerechnet und gespeichert. Ohne näher auf diesen eingehen zu wollen sei kur erwähnt, dass dieser einen Algorithmus festlegt, mit dessen Hilfe Variablen in einem Speicherbereich repräsentiert werden. Dieser Speicherbereich kann sich je nach Datentyp und Programmiersprache zwar unterscheiden, ist jedoch an sich stets fester Größe. Dadurch und durch den Umstand, dass Zahlen vor dem Speichern umgerechnet werden ergibt sich die Problematik, dass bestimmte Zahlen nicht exakt repräsentiert werden können und lediglich die \quotes{näheste} Repräsentation gespeichert werden kann. Dieser Effekt ist schwer absehbar und kann in der Praxis zu ungenauen (Zwischen-)Ergebnissen führen.

\java{PrecisionExample.java}
\sepCodeAndOutputCheck
\begin{shellwindow}
$ javac PrecisionExample.java 
$ java PrecisionExample
0,64999997615814210000000 == 0.65 -> false
0.64999997615814208984375
0.65
\end{shellwindow}
\mintedCaption{Ungenauigkeit am Beispiel einer float-Variable}{floatJava}

\autoref{floatJava} zeigt beispielhaft wie die Repräsentation eines Wertes vom tatsächlichen abweichen kann. Durch die Weiterverwendung eines solchen, nicht-exakt repräsentierten, Wert würden sich unter Umständen Folgefehler in Berechnungen ergeben. Außerdem können wie gezeigt Gleicheitsvergleiche von Zahlen, insbesondere von Berechnungsergebnissen, dadurch fehlerbehaftet sein, weshalb Fließkommadatentypen stets auf ein Werteintervall statt auf Gleichheit geprüft werden sollten.

Diese beschriebenen Fließkommatypen werden stets zur Basis 2 berechnet und heißen daher auch binäre Fließkommatypen. In der \mintinline{java}{java.math}-Bibliothek findet sich jedoch auch ein Objekttyp \mintinline{java}{BigDecimal} welcher ein Fließkommawert zur Basis 10, also ein dezimales Fließkomma darstellt. Die Speicherung beruht in­des­sen auf zwei \mintinline{java}{Integer}-Werten, die ein unskalierten Faktor und einen Exponenten zur Skalierung darstellen. Außerdem ist dieser Typ steuerbar was die Rundung, die Exaktheit von Ergebnissen und das Verhalten bei nichtdarstellbaren Werten angeht. Mit den in \autoref{floatJava} aufgezeigten Effekten lässt sich außerdem festhalten, dass \mintinline{java}{BigDecimal}s nur über andere \mintinline{java}{BigDecimal}-Objekte oder \mintinline{java}{String}s zuverlässig instantiiert werden können. Andere Konstruktoren speichern die übergebenen Werte in primitiven Datentypen zwischen, wodurch eben diese ungewünschten Fehler in der Repräsentation wieder auftreten. \mintinline{java}{BigDecimal} bietet somit eine Möglichkeit Werte exakt abzuspeichern bzw. Kenntnis über unexakte Speicherung -- in der Regel durch Exceptions -- zu erhalten und diese zu steuern. Mit diesem Objekttypen gehen jedoch Speicher- und Laufzeit-Overheads einher die nicht vernachlässigt werden dürfen.

\subsection*{Festkommaarithmetik}
Um die angesprochenen Probleme zu umgehen verwenden manche Sprachen eine Festkommaarithmetik, um rationale Zahlen zu Speichern oder bieten zumindest Datentypen um eine derartige Speicherrepräsentation zu erreichen. 

Dabei wird im Gegensatz zu Fließkommazahlen festgelegt wieviele Stellen einer Zahl vor- bzw. nach dem Komma gespeichert werden sollen. Jede Ziffer wird dabei für sich -- je nach Implementierung durch eine bestimmte Codierung -- gespeichert und erlaubt somit absolute Genauigkeit im Werte- bzw. Darstellungsbereich. Auch ist der Umgang mit Überläufen fest definiert und führt zu konsistentem und abschätzbarem Verhalten. Ergebnisse werden stets zur Speicherung \quotes{abgeschnitten} außer man definiert explizit, dass gerundet werden soll. \autoref{decimalsInCobol} enthält Beispiele zu beiden Varianten. \mintinline{cobolfree}{PIC 9V9(2)} deklariert eine Variable mit genau einer Vor- und zwei Nachkommastellen. Damit wäre beispielsweise sichergestellt, dass alle Geldbeträge $< 10$ -- auch nach Berechnungen -- korrekt dargestellt werden können.

\cobol{PRECISION_EXAMPLE.cbl}
\sepCodeAndOutputCheck
\begin{shellwindow}
0.99
1.00
\end{shellwindow}
\mintedCaption{Dezimalzahlen in COBOL}{decimalsInCobol}

Ein weiterer Vorteil der Abbildung mit Festkomma ist die Tatsache, dass beliebig große Zahlen gespeichert werden können. Dies sorgt neben höherem Speicherbedarf, jedoch auch dafür, dass sofern Grundrechenarten für eine Ziffer implementiert sind, beliebig lange Ziffernfolgen nach dem gleichen Schema verarbeitet werden können. Die Ergebnisse werden dabei auch zeichenweise gespeichert und lassen so keine Rundungsfehler oder Fehler aufgrund von unzureichendem Speicherplatz zur Abbildung zu. Der Speicherbereich kann in COBOL jedoch zum Beispiel durch das Nutzen von \mintinline{cobolfree}{PACKED DECIMAL}s mit dem Schlüsselwort \mintinline{cobolfree}{COMP-3} hinter der \mintinline{cobolfree}{PICTURE}-Anweisung reduziert werden. Hierbei wird lediglich ein Nibble (\nicefrac{1}{2} Byte) pro Ziffer benötigt.

\recap{In betrieblichen Informationssystemen und speziell bei der Verarbeitung von Geldbeträgen, ist es also unerlässlich die Sicherheit einer exakten Darstellung von Zahlen zu haben. Während die binäre Fließkommadarstellung Speicherplatz-Vorteile und Flexibilität des Wertebereichs einer Zahl bietet, jedoch Werte unter Umständen nicht exakt repräsentieren kann, stellt Festkommaarithmetik sicher, dass Zahlen exakt und vorhersehbar repräsentiert werden. Dies wird durch erhöhten Speicherbereich und fehlende Flexibilität erkauft, ist jedoch in der Praxis oftmals unerlässlich. Eine Möglichkeit diese Sicherheit in Java zu erreichen ist das Nutzen des \mintinline{java}{BigDecimal}-Typen, der viele Nachteile und vor allem Unsicherheiten gegenüber binären Fließkommatypen aus dem Weg räumt. Jedoch führt dieser unter Umständen zu Performanz- bzw. Speichereinbußen. COBOL bietet mit Verwendung der Festkommaarithmetik bereits standardmäßig eine Darstellungssicherheit und Vorhersagbarkeit die vielen modernen Sprachen fehlt.}

% Vorallem in betrieblichen Informationssystemen, die oftmals an bestimmten Stellen Geldbeträge durch eine gewisse Anzahl von Rechenschritten errechnen sollen, ist es unerlässlich einen Blick auf die Rechengenauigkeit des Systems und der verwendeten Sprachen zu werfen.
% \subsection*{Rechengenauigkeit in Java}
% Java verwendet zur Speicherung von Fließkommazahlen, wie viele andere moderne Programmiersprachen, den \textit{IEEE 754}-Standard. Ohne näher auf diesen eingehen zu wollen, legt dieser einen Algorithmus fest, mit dessen Hilfe Variablen in einem Speicherbereich repräsentiert werden. Dieser Speicherbereich kann sich je nach Datentyp und Programmiersprache unterscheiden, ist jedoch an sich stets von fester Größe. Daher ist es nicht möglich beliebig genaue Werte abzubilden. 
% \java{PrecisionExample.java}
% \sepCodeAndOutputCheck
% \begin{shellwindow}
% $ javac PrecisionExample.java 
% $ java PrecisionExample
% Double addition: 199999.45
% Float addition: 199999.44
% BigDecimal addition: 199999.45
% \end{shellwindow}
% \mintedCaption{Addition von float und double Variablen in Java}{floatDoubleJava}
% \autoref{floatDoubleJava} zeigt das bereits anhand eines sehr einfachen Beispiels. Bereits die Addition von zwei Zahlen mit einer bzw. zwei Dezimalstellen, welche zugegebenermaßen bewusst so gewählt wurden, legt die Problematik offen. So unterscheiden sich die Ergebnisse der Berechnung je nach Datentyp, was auf die Abbildung der Zahlen im Speicher zurückzuführen ist. Um Sicherheit bei der Berechnung zu erhalten ist es stets nötig den Datentyp \mintinline{java}{BigDecimal}, mitsamt des damit verbundenen Speicher- und Laufzeit-Overheads, wie im letzten Teil des Beispiels gezeigt, zu verwenden. Wichtig hierbei ist, dass die Definition wenn möglich über den Konstruktor erfolgt, der eine String-Repräsentation eines Wertes erhält. Andere Konstruktoren wie z.B. \mintinline{java}{BigDecimal(double)} können zu Problemen führen, da der übergebene Wert durch die Speicherung als z.B. \mintinline{java}{double} bereits an Genauigkeit verlieren kann.
% \subsection*{Rechengenauigkeit in COBOL}
% Wie später in \autoref{variables} noch genauer ausgeführt ist der Entwickler und nicht die Sprache in COBOL dafür zuständig genau festzulegen, wie viele Dezimalstellen eine Variable speichern soll. Diese Eigenschaft in Verbindung mit der Speicherrepräsentation der Daten in COBOL führt zu einer praktisch absolut exakten Genauigkeit. %COBOL speichert dabei Daten im sogennanten \textit{BCD}-Code (Binary Coded Decimals) \todo[inline]{https://www.bernd-leitenberger.de/genauigkeit-von-computern.shtml -> Weitere Quelle!} ab. 
% Jedes Zeichen wird in COBOL separat gespeichert. Dies sorgt neben höherem Speicherbedarf, jedoch auch dafür, dass sofern Grundrechenarten für eine Ziffer implementiert sind, beliebig lange Ziffernfolgen nach dem gleichen Schema verarbeitet werden können. Die Ergebnisse werden dabei wieder auch zeichenweise gespeichert und lassen so keine Rundungsfehler oder Fehler aufgrund von unzureichendem Speicherplatz zur Abbildung zu.
    \section{Reporting}\label{reporting}
\todo[inline]{EVA-Prinzip mit Zitat!}
    \section{Schnittstellen und Datenquellen}\label{schnittstellenDatenquellen}

In betrieblichen Informationssystemen stellen außerdem Schnittstellen ein wichtiges Thema dar. Sowohl das Bereitstellen von standardisierten und dokumentierten Interfaces als auch das Nutzen von anderen Systemen über ihre Schnittstellen ist stets Teil aller Anwendungsfälle. 

Vor allem in der heutigen Zeit, in der Informationssysteme nicht mehr als alleinige Verarbeitungs-, Reporting- und Darstellungsschicht fungieren, sondern eingebettet in einen größeren Kontext aus verschiedensten Modulen, mobilen Applikationen und Websites funktionieren und mit diesen kommunizieren müssen, ist ein ausgereiftes Schnittstellenkonzept und die standardisierte Bereitstellung und Nutzung von Daten und Diensten unerlässlich.

Für Java sind an dieser Stelle eine Fülle an Bibliotheken erhältlich, welche Netzwerkkommunikation über verschiedenste Protokolle auf unterschiedlichen Ebenen ermöglichen. Neben diversen verfügbaren Fremdbibliotheken bietet bereits das JDK unterschiedliche Methoden zur Kommunikation mit Fremdsystemen.

Zudem ist es durch die Definition von Interfaces auch auf Klassenebene möglich, Schnittstellen zu bieten, die eine einfache Erweiterung von und Verbindungen zu Neusystemen möglich machen.

COBOL hingegen lässt an dieser Stelle einige Funktionalität vermissen. Durch das fehlende Bibliothekskonzept -- wie in \autoref{wiederverwendbarkeit} erläutert -- und das gänzliche Fehlen von Netzwerkkommunikationsmechanismen ist es in reinem COBOL nicht möglich, Netzwerkschnittstellen festzulegen, die von außen erreichbar sind, oder solche zu nutzen. 

COBOL-Systeme nutzen daher \idR Systemmodule, um Aufgaben, wie die Netzwerkkommunikation, zu erledigen, die mit reinem COBOL nicht möglich oder aufwendig sind. Außerdem erfordern diese Systeme nicht selten Anpassungen in anderen Systemen, auch wenn es nicht sinnvoll erscheint.

Auch intern kann ein COBOL-System nur bedingt Standards definieren, die zwischen unterschiedlichen Programmteilen für einheitliche Kommunikationskanäle sorgen. Daher basieren COBOL-Systeme, wie \autoref{reporting} beschreibt, auf einfachen EVA-Prinzipien. 

Mit Tools wie IBM MQ, zuvor WebSphere MQ, \footnote{\url{https://www.ibm.com/de-de/marketplace/secure-messaging}} können Systeme allerdings so verbunden werden, dass sie miteinander kommunizieren können. Laut Herrn Lamperstorfer lässt sich damit \quotes{technisch einfach} eine Verbindung zwischen Alt- und Neusystemen herstellen, diese sei jedoch fachlich und im Hinblick auf Performanz recht schwierig umzusetzen.



\mytodo{\todo[inline]{datenquellen, datenbanken}}

\mytodo{\todo[inline]{3. Batchbetrieb -> Schnittstellen (Host nicht COBOL)}}
\mytodo{\todo[inline]{Schnittstellen datenbasiert(copies). Db nicht so relevant da host sehr schnell mit Dateien. }}