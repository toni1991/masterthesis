\section{Langlebigkeit, Wartbarkeit und Verlässlichkeit} \label{verlaesslichkeit}

Durch ihre hohe Komplexität werden betriebliche Informationssysteme \idR über viele Jahre oder sogar Jahrzehnte betrieben und dabei gewartet, erweitert und angepasst. Um die Langlebigkeit, Wartbarkeit und Verlässlichkeit solcher Systeme sicherzustellen, werden diese in der Praxis mehr oder weniger umfangreichen Tests unterzogen. Damit kann beispielsweise sichergestellt werden, dass bestehender Code auch nach der Erweiterung um neue Funktionen weiterhin funktioniert. Eine der wichtigsten Techniken hierbei sind sogenannte Unit-Tests. Dabei werden einzelne isolierte Einheiten getestet und so Einflüsse von anderen Programmteilen minimiert. 

In Java gibt es einige Frameworks, beispielsweise \textit{JUnit}\footnote{\url{https://junit.org/} \visitedOn} oder \textit{Mockito}\footnote{\url{http://site.mockito.org/} \visitedOn}, die das Testen direkt oder indirekt unterstützen. Die Wartbarkeit begünstigt in Java zusätzlich das relativ einfache Durchführen von Refactorings. So können Programme mithilfe von modernen Entwicklungsumgebungen teilweise neu geschrieben \bzw strukturiert werden und bestehender Code so im Zuge von Erweiterungen verbessert werden. Auch hierbei sind Tests zur Überprüfung der Korrektheit von Vorteil. 

COBOL bietet zum Testen weitaus weniger Möglichkeiten. Entwicklung, Testen und Debuggen direkt am Host sind teuer, da die Kosten eines Mainframes oft nach Rechenzeit berechnet werden, und daher wurden, laut Herrn Lamperstorfer vor allem in frühen COBOL-Systemen, oftmals funktionierende und manuell getestete \quotes{Schablonen} beispielsweise für die Dateiverarbeitung wiederverwendet, sodass diese nur minimal angepasst werden mussten. Auch Refactorings nehme man in COBOL-Systemen tendenziell selten vor, da diese ein umfangreiches Testen erfordern würden.

In puncto Verlässlichkeit können sich COBOL-Anwendungen jedoch oftmals auf gut isolierte Infrastrukturen verlassen, die durch ihre Homogenität, wenig Fortentwicklung und eingebaute Ausfallsicherheitsmaßnahmen eine zuverlässige Basis bieten. Java-Systeme hingegen werden auf vielen unterschiedlichen Plattformen betrieben, die sich sehr schnelllebig verändern, wodurch zusätzliches Augenmerk auf die Sicherung der oben genannten Eigenschaften gelegt werden muss.