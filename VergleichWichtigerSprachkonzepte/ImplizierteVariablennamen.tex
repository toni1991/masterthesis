\section{Implizierte Variablennamen}

Dieser Abschnitt behandelt einen Fehler, der typischerweise zu Beginn in der Softwareentwicklung beobachtet werden kann. \autoref{if_variable_error_java} zeigt wie Anfänger häufig versuchen logische Ausdrücke zu konstruieren. Dies entspricht dem intuitiven Gedanken \quotes{Wenn Variable X größer 0 und kleiner 5 ist, dann \dots} oder der mathematischen Definition \quotes{$0 < X < 5$}.\\

\mintedJava{Code/IfVariableError.java}{Keine implizierten Variablennamen in logischen Ausdrücken in Java}{if_variable_error_java}

Versucht man \autoref{if_variable_error_java} zu kompilieren treten einige Fehler auf:\\

\begin{shellwindow}
$ javac -Xmaxerrs 3 IfVariableError.java 
IfVariableError.java:4: error: > expected
        if (System.currentTimeMillis() > 0 && < Long.MAX_VALUE) {
                                                              ^
IfVariableError.java:4: error: ')' expected
        if (System.currentTimeMillis() > 0 && < Long.MAX_VALUE) {
                                                               ^
IfVariableError.java:8: error: illegal start of type
        if (0 < System.currentTimeMillis() < Long.MAX_VALUE) {
        ^
3 errors
\end{shellwindow}

In Java können nur vollständige Logische Ausdrücke mit logischen Operatoren verknüpft werden. Zum anderen kann innerhalb eines logischen Ausdrucks lediglich maximal einmal ein Vergleichsoperator verwendet werden. \autoref{if_variable_no_error_java} demonstriert eine funktionsfähige Implementierung des vorhergehenden Beispiels.\\

\mintedJava{Code/IfVariableNoError.java}{Verwendung von Variablennamen in logischen Ausdrücken in Java}{if_variable_no_error_java}

Dieser Code kann fehlerfrei kompiliert und ausgeführt werden:\\

\begin{shellwindow}
$ javac IfVariableNoError.java 
$ java IfVariableNoError
We get here everytime!
\end{shellwindow}

In COBOL hingegen ist das Schreiben von logischen Ausdrücken mit implizierten Variablennamen möglich.\\

\mintedCobol{Code/implicit_variable_names.cbl}{Implizierte Variablennamen in COBOL}{implicit_names_cobol}

\autoref{implicit_names_cobol} stellt in Zeile 11 die Verwendung von implizierten Variablennamen in COBOL dar, die im Gegensatz zu Java möglich ist.