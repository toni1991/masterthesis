\section{Generische Programmierung}
Moderne Programmiersprachen wie Java oder C\# erlauben eine Programmierung mit sogenannten \textit{Generics}. Dabei handelt es sich um ein Konzept, bei dem Variablentypen generisch sein können, solange sie bestimmte Eigenschaften erfüllen. Diese Eigenschaften werden durch das Implementieren eines bestimmten Interfaces oder durch das Erben von einer bestimmten Klasse beschrieben. Dies soll \autoref{javaGenerics} verdeutlichen. Hieran wird auch deutlich, dass sowohl für Interfaces als auch für Oberklassen stets das Schlüsselwort \jav{extends} verwendet werden muss.

\mintedJava{GenericsExample.java}{Generics in Java}{javaGenerics}

In diesem Beispiel werden die drei generischen Typen \textit{S}, \textit{T} und \textit{U} verwendet. \textit{T} muss dabei die Eigenschaft erfüllen, das \jav{Serializable} Interface zu implementieren. \textit{S} muss eine Unterklasse von \jav{Number} sein. Für \textit{U} hingegen wird keine bestimmte Eigenschaft definiert. Das bedeutet, dass an dieser Stelle jede Klasse, die von der Klasse \jav{Object} erbt -- in Java also \textbf{jede} Klasse -- verwendet werden kann. 

Eine weitere Eigenschaft, die gezeigt werden soll, ist, dass es sowohl generische Klassen als auch generische Methoden geben kann. Generische Typen, die für eine Klasse definiert sind stehen in der gesamten Klasse zur Verfügung, müssen jedoch bereits beim instanziieren festgelegt werden. Generische Methoden hingegen definieren generische Typen nur für den eigenen Scope.

Dieses Konzept sorgt dafür, dass Algorithmen implementiert und als Bibliotheken bereitgestellt werden können, ohne Kenntnis über die tatsächlich verwendeten Datentypen zu haben, was die in \autoref{wiederverwendbarkeit} angesprochenen Modularisierungsmöglichkeiten unterstützt. Beispielsweise kann ein Sortieralgorithmus implementiert werden, welche generische Objekte entgegennimmt, die das \jav{Comparable}-Interface -- ein Java Interface, welches dafür sorgt, dass zwei Objekte in eine Größenbeziehung gesetzt werden können -- implementieren. Folglich können mit diesem Algorithmus alle Objekte sortiert werden, die das Interface implementieren. Dieses Konzept ist in der objektorientierten Programmierung grundlegend und trägt maßgeblich zur Wiederverwendung und Kapselung bei.