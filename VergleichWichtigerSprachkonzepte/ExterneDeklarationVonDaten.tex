\section{Externe Deklaration von Daten}

Sowohl zur Wiederverwendbarkeit, als auch zur Erreichung einer gewissen Typsicherheit ist es in vielen Programmiersprachen möglich Datenstrukturen extern in einer eigenen Datei zu deklarieren und an verschiedenen Stellen wieder zu verwenden. Dies ist auch in COBOL und Java möglich, wenngleich sich die Ansätze stark unterscheiden.

In Java werden Klassen in sogenannten Packages organisiert. Um auf Klassen aus einem anderen Package zuzugreifen, ist ein Importieren der betreffenden Klasse notwendig. Dies geschieht wie in \autoref{importJava} mihilfe des \mintinline{java}{import}-Statements.

\mintedJava{Code/Import.java}{Import in Java}{importJava}