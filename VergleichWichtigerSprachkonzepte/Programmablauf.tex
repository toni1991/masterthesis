\section{Programmablauf}
Durch die in \autoref{sec:functionsAndReturnValues} erläuterten Unterschiede ergeben sich auch im Programmablauf Diskrepanzen. Während Java einen definierten Einstiegspunkt in ein Programm hat (\mintinline{java}{main}-Methode) wird ein COBOL-Programm stets sequenziell von oben nach unten abgearbeitet und durchlaufen. Diese Abarbeitung beginnt am Anfang der \mintinline{cobolfree}{PROCEDURE DIVISION}.\\

\subsection*{Programmablauf in Java}

\java{MainMethod.java}
\sepCodeAndOutputCheck
\begin{shellwindow}
$ javac MainMethod.java 
$ java MainMethod
Running main method!
Running other method!
Continue main method!
\end{shellwindow}
\mintedCaption{Java main-Methode}{javaMainMethod}

\autoref{javaMainMethod} demonstriert einen sehr simplen Programmablauf in Java. Wie bereits erwähnt ist der Startpunkt eines jeden Java-Programms die \mintinline{java}{main}-Methode. Von dieser aus können weitere Methoden aufgerufen werden und sobald das Ende dieser Methode erreicht ist terminiert das Programm. Im vorliegenden Beispiel wird also nach einer Ausgabe in \mintinline{java}{main}, die Funktion \mintinline{java}{otherMethod} aufgerufen, bevor der Ablauf wieder in der \mintinline{java}{main}-Methode fortgesetzt wird. Daran soll folgendes Verhalten deutlich werden: Endet eine aufgerufene Funktion wie geplant -- d.h. ohne eine \mintinline{java}{Exception} \todo{reference exception section} -- wird stets mit der nächsten Anweisung nach dem Funktionsaufruf fortgefahren.\\

\subsection*{Programmablauf in COBOL}

In COBOL gestaltet sich der Programmablauf gänzlich anders. Das Programm wird stets von oben nach unten durchlaufen. Wobei dieser lineare Ablauf durch die Verwendung von \mintinline{cobolfree}{PERFORM}-, \mintinline{cobolfree}{CALL}- und \mintinline{cobolfree}{GO TO}-Anweisungen verändert werden kann. An dieser Stelle sei ausdrücklich erwähnt, dass die Verwendung des \mintinline{cobolfree}{GO TO}\index{GO TO}-Befehls unter allen Umständen unerlassen werden sollte, da ansonsten sehr schwer verständlicher und wartbarer Code entsteht! Die Ausführung eines COBOL-Programms endet beim Erreichen einer \mintinline{cobolfree}{STOP RUN}-Anweisung oder mit dem Ende des Programms (\mintinline{cobolfree}{END PROGRAM}).\\

\cobol{simpleControlFlow.cbl}
\sepCodeAndOutputCheck
\begin{shellwindow}
$ ./simpleControlFlow 
Main paragraph
Enter some number: 0
Main paragraph again
$ ./simpleControlFlow 
Main paragraph
Enter some number: 1 
Main paragraph again
Enter some number: 2
\end{shellwindow}
\mintedCaption{Programmablauf in COBOL}{simpleControlFlowCobol}

Die beiden Ausführungen von \autoref{simpleControlFlowCobol} zeigen das angesprochene Verhalten eines COBOL-Programms. Beim ersten Durchlauf wird für die Variable \mintinline{cobolfree}{INPUT-NUMBER} der Wert 0 eingegeben, was durch das Ausführen der \mintinline{cobolfree}{STOP RUN}-Anweisung in Zeile 15, das Beenden des Programmes bewirkt. Beim zweiten Mal wird hingegen der Wert 1 eingegeben. Dieser Wert verhindert das Abschließen des Programms in Zeile 15, wodurch der Programmablauf in Zeile 17 fortgesetzt wird und somit erneut die Eingabeaufforderung erscheint.\\

Wie in \autoref{sec:scope} beschrieben besteht ein COBOL-Programm aus verschiedenen strukturellen Komponenten. Diese haben auch einen gewissen Einfluss auf den Programmablauf. Dies soll das Beispiel in \autoref{paragraphSecionControlFlowCobol} veranschaulichen.\\

\cobol{paragraphSecionControlFlow.cbl}
\sepCodeAndOutputCheck
\begin{shellwindow}
$ ./paragraphSecionControlFlow 
Calling section:
123 
Calling paragraphs with PERFORM THRU:
123 
Calling paragraph:
1
\end{shellwindow}
\mintedCaption{Programmablaufunterschiede in COBOL mit Sections und Paragraphs}{paragraphSecionControlFlowCobol}

Wird mittels \mintinline{cobolfree}{PERFORM} eine \mintinline{cobolfree}{SECTION} aufgerufen, so werden alle Paragraphs innerhalb dieser \mintinline{cobolfree}{SECTION} der Reihe nach ausgeführt. Ruft man jedoch einen Paragraph auf, so wird nur dieser Paragraph ausgeführt. Eine weitere Möglichkeit ist die Kombination des \mintinline{cobolfree}{PERFORM} mit dem \mintinline{cobolfree}{THRU}-Schlüsselwort. Hierbei werden alle Paragraphs zwischen zwei festgelegten Paragraphs ausgeführt. Der Kontrollfluss geht bei jeder Variante stets an das Statement nach dem \mintinline{cobolfree}{PERFORM}. \todo{EXIT-Statement?}\\