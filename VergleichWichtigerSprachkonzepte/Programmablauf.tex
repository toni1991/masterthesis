\section{Programmablauf}
Ein Punkt in dem sich COBOL und Java sehr stark unterscheiden ist der Programmablauf. Während Java einen definierten Einstiegspunkt in ein Programm hat (\mintinline{java}{main}-Methode) wird ein COBOL-Programm stets sequenziell von oben nach unten abgearbeitet und durchlaufen. Diese Abarbeitung beginnt am Anfang der \mintinline{cobolfree}{PROCEDURE DIVISION}.

\java{MainMethod.java}
\sepCodeAndOutput
\begin{shellwindow}
$ javac MainMethod.java 
$ java MainMethod
Running main method!
Running other method!
Continue main method!
\end{shellwindow}
\mintedCaption{Java main-Methode}{javaMainMethod}

rekursion

unterprogramme

funktionen / perform
