\section{Mehrfachverzweigungen}\label{mehrfachverzweigungen}
Ein wichtiges Konstrukt um den Programmfluß eines Programms zu steuern sind Mehrfachverzweigungen. Obwohl sowohl Java als auch COBOL Mehrfachverzweigungen bieten, sind diese doch leicht unterschiedlich zu verwenden. Im Folgenden sollen verschiedene Verwendungsmöglichkeiten der jeweiligen Konstrukte dargestellt werden.\\

In Java bildet das \mintinline{java}{switch}-\mintinline{java}{case}-Konstrukt eine Mehrfachverzweigung ab. \autoref{switch_case_java} zeigt dabei die wichtigsten Verwendungsmöglickeiten.

\mintedJava{switch_case.java}{Mehrfachverzweigungen in Java}{switch_case_java}

Zum einen ist zu beachten, dass Werte nur mit Literalen und Konstanten verglichen werden können. Ein Vergleich einer Variablen mit einer weiteren ist hierbei nicht zulässig, solange diese nicht als \mintinline{java}{final} deklariert ist. Auch der Vergleich auf bestimmte Wertebreiche oder nicht-primitive Datentypen ist nicht zulässig (Seit Java 7 sind vergleiche mit String-Literalen möglich).\\

Jedoch ist ein gewolltes \quotes{Durchfallen} möglich um bei verschiedenen Werten die gleichen Programmzweige zu durchlaufen. Dabei Spielt das Schlüsselwort \mintinline{java}{break} eine entscheidende Rolle. Wird kein \mintinline{java}{break} am Ende einer \mintinline{java}{case}-Anweisung verwendet, so wird automatisch in den ausführbaren Block der darauffolgenden \mintinline{java}{case}-Anweisung gesprungen. Dies zeigt sich in den Zeilen 6--9 und 17f. von \autoref{switch_case_java}. Die Verwedung der \mintinline{java}{break}-Anweisung wird hingegen in den Zeilen 11, 15 und 20 genutzt um den \mintinline{java}{switch}-Block zu verlassen.\\ 

Das Fehlen eines \mintinline{java}{break} kann in der Praxis schnell zu unerwünschtem und unerklärlichem Verhalten führen. Deshalb folgt in der Regel jedem \mintinline{java}{case} ein \mintinline{java}{break}. \\

Das Pendant in COBOL stellt das Schlüsselwort \mintinline{cobolfree}{EVALUATE} dar. Wenngleich es den gleichen Sinn wie das \mintinline{java}{switch}-\mintinline{java}{case}-Konstrukt in Java erfüllen soll, ist es vielseitiger einsetzbar wie die folgenden Beispiele illustrieren sollen. \autoref{switch_case_cobol1} und \autoref{switch_case_cobol2} sind hierbei semantisch gleich, obwohl das \mintinline{cobolfree}{EVALUATE}-Konstrukt jeweils leicht anders verwendet wird.\\

Das folgende \autoref{switch_case_cobol1} führt die Verwendungsmöglichkeit an, die dem \mintinline{java}{switch}-\mintinline{java}{case}-Konstrukt in Java am nächsten kommt. Nach dem \mintinline{cobolfree}{EVALUATE}-Schlüsselwort werden Variablennamen angegeben, deren Werte anschließend in einer \mintinline{cobolfree}{WHEN}-Bedingung betrachtet werden sollen.\\

Auch hier sieht man in den Zeilen 14, 17, 20, 23 ein gewolltes \quotes{Durchfallen} wie in Java. Einziger Unterschied hierbei ist, dass in COBOL jeder ausführbare Block nach einem \mintinline{cobolfree}{WHEN} eigenständig ist somit kein \mintinline{java}{break} notwendig ist. Ein \quotes{Durchfallen} ist also nur möglich, wenn der komplette Anweisungsblock leer ist. \\

Der \mintinline{cobolfree}{OTHER}-Zweig entspricht in COBOL dem aus Java bekannten \mintinline{java}{default}. Dieser Zweig wird ausgeführt wenn die Kriterien keines anderen zutreffen.

\mintedCobol{switch_case1.cbl}{Mehrfachverzweigungen in COBOL mit ALSO}{switch_case_cobol1}

Die erste Besonderheit, die \autoref{switch_case_cobol1} illustrieren soll ist das Testen von jeweils zwei Bedingungen. Dies geschieht mithilfe des \mintinline{cobolfree}{ALSO}-Schlüsselworts. Während in Java lediglich die Evaluation einer einzelnen Bedingung möglich ist, erlaubt COBOL an dieser Stelle die Überprüfung beliebig vieler Kritierien.\\

Eine weitere Eigenheit zeigt sich in der Auswertung der Variable \mintinline{text}{AGE}. Hierbei werden im Beispiel Wertebereiche angegeben in denen die Variable liegen kann. Auch das ist in Java so nicht möglich, da wie bereits erwähnt nur mit Literale und Konstanten verglichen werden kann.

\mintedCobol{switch_case2.cbl}{Mehrfachverzweigungen in COBOL als EVALUATE TRUE}{switch_case_cobol2}

\autoref{switch_case_cobol2} stellt weitere Unterschiede zu Mehrfachverzweigungen in Java dar. So ist diese \mintinline{cobolfree}{EVALUATE TRUE}-Variante (auch als \mintinline{cobolfree}{EVALUATE FALSE} möglich) mit geschachtelten \mintinline{java}{if}-Abfragen in Java zu vergleichen. Jede Bedingung hinter dem \mintinline{cobolfree}{WHEN}-Schlüsselwort entspricht hierbei einem vollständigen logischen Ausdruck. Daher ist es möglich wie in Zeile 16 gezeigt, Rechenoperationen durchzuführen oder logische Operatoren wie in diesem Fall das \mintinline{cobolfree}{OR} zu verwenden. \\

Abschließend kann noch erwähnt werden, dass sowohl Variablen-Vergleiche wie in \autoref{switch_case_cobol1} als auch \mintinline{cobolfree}{TRUE}-Vergleiche wie in \autoref{switch_case_cobol2} zusammen, mithilfe des \mintinline{cobolfree}{ALSO}-Schlüsselworts verwendet werden können.