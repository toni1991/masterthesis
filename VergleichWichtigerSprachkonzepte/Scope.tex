\section{Programmstruktur}\label{sec:scope}
Dieser Abschnitt behandelt die strukturellen Unterschiede von Java- und COBOL-Programmen. Dazu wird auf den Scope von Variablen eingegangen und erläutert in welche Einheiten sich die Programme der jeweiligen Sprache aufteilen lassen.
Der sog. Scope, zu deutsch Gültigkeitsbereich, gibt in der Programmierung an, in welchem Bereich eine Variable gültig ist.
\todo{enum, interfaces irgendwo einbauen}
\subsection*{Java}

\subsubsection*{Scope}
In Java ist der Scope einer Variablen meist einfach zu erkennen. Eine Variable ist innerhalb der geschweiften Klammern gültig, die die Variablendeklaration beinhalten. Dies verdeutlicht \autoref{javaScope}.\\

Die Variable \mintinline{java}{memberVariable} ist innerhalb der gesamten Klasse \mintinline{java}{ScopeExample}, also in jeder enthaltenen Methode, verschachtelten Klassen und wiederum deren Methoden, gültig. Eine Variable mit selbem Namen kann auch innerhalb einer Methode deklariert werden. Auf die Instanzvariable kann mit dem \mintinline{java}{this}-Schlüsselwort zugegriffen werden. In geschachtelten Klassen muss zusätzlich der Klassenname vorangestellt werden wie Zeile 27 zeigt.  Ist keine lokale Variable mit selbem Namen vorhanden, so kann dieses Schlüsselwort auch weggelassen werden. \\

Wie Zeile 19 zeigt ist es auch nicht möglich auf lokale Variablen einer anderen Funktion zuzugreifen. Gleiches gilt für Instanzvariablen verschachtelter Klassen.\\

\mintedJava{ScopeExample.java}{Variablendeklarationen mit verschiedenen Scopes}{javaScope}

\subsubsection*{Struktur}

Auch zeigt das Beispiel in \autoref{javaScope} die wichtigsten Konzepte der Strukturierung von Java-Code. Zeile 1 beinhaltet die Package-Deklaration, d.h. hiermit wird die Klasse dem hier genannten Package zugeordnet. Diese Deklaration \textbf{muss} gleich der Ordnerhierarchie sein, in denen die Java-Dateien verwaltet werden. \\

Die nächstkleinere Einheit eines Java-Programms stellen Klassen dar. Hierbei handelt es sich um das Kernkonzept der objektorientierten Programmierung. Von dieser Klasse können fortan Objekte instanziiert werden. Um einen tieferen Einblick in die Thematik der Objektorientierung zu erhalten sei an dieser Stelle einschlägige Fachliteratur erwähnt. Diese erwähnte Klasse muss dabei in einer Datei gespeichert sein, die den selben Namen trägt wie die Klasse selbst. Aus dem Klassennamen \mintinline{java}{MasterThesis} folgt also der Dateiname \mintinline{text}{MasterThesis.java}. \\

Teil dieser Klassen können wiederum Methoden, Variablendeklarationen und weitere Klassen sein. Diese können jeweils statisch oder auch einer Instanz zugeordnet sein. Auch dabei handelt es sich um ein gängiges Konzept der objektorientierten Softwareentwicklung. Hierzu sei an dieser stelle lediglich erwähnt, dass statische Methoden, Variablen und Klassen Teil der Klasse sind und kein konkret instanziiertes Objekt benötigen während nicht-statische Komponenten stets ein konkretes Objekt einer Klasse benötigen.

Diese weiteren Klassen haben strukturell die selben Eigenschaften wie die umgebende Klasse, außer, dass sie nicht in einer Datei gespeichert sein müssen deren Name dem Klassennamen entspricht.

\todo[inline]{Diagramm}

Methoden wiederum \todo{methoden aus statements und vars}

\todo{Vars und Statements aus anonymen ding (Auch vars in klasse)} 

\mintedJava{AnonymousClassAndMethodExample.java}{Anonyme Klassen und Funktion in Java}{anonymousJava}

u.a. SENTENCE u. PARAGRAPH \& Statement und \{\}-Scope


-> DIVISIONS, SECTIONS, paragraphs, sentences, and statements