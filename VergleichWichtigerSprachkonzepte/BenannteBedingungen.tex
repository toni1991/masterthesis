\section{Benannte Bedingungen}
Neben den bereits angesprochenen Stufennummern stellt die \textit{88} eine weitere Besonderheit in COBOL dar. Mit ihr ist es möglich einer Variable einen Wahrheitswert zuzuweisen, der von einem anderen Variablenwert abhängt. Es entsteht eine sogenannte benannte Bedingung.\\

\autoref{88_cobol_listing} zeigt die Verwendung der Stufennummer 88. Die Variable \mintinline{text}{AGE} kann dabei zweistellige numerische Werte enthalten die einem beispielhaften Alter entsprechen, welches zu Beginn mit 13 vorbelegt wird. Liegt der Wert zwischen 0 und 17 (\mintinline{cobolfree}{VALUE 0 THRU 17}) so weisen die Variablen \mintinline{text}{ISUNDERAGE} den Wahrheitswert \mintinline{cobolfree}{TRUE} und \mintinline{text}{ISADULT} den Wahrheitswert \mintinline{cobolfree}{FALSE} auf.\\

Der so entstandene Wahrheitswert kann folglich immer dann verwendet werden, wenn getestet werden soll, ob die Variable \mintinline{text}{AGE} im Bereich zwischen 0 und 17 bzw. zwischen 17 un 99 liegt. Also um zu testen, ob das Alter einer minder- oder volljährigen Person entspricht.\\

\cobol{88_section.cbl}
\sepCodeAndOutputCheck
\begin{shellwindow}
$ ./eightyeight
Person is underage (AGE = 13)
AGE = 18
\end{shellwindow}
\mintedCaption{Beispiel für COBOL Stufennummer 88}{88_cobol_listing}

Zeile 14 des Programms illustriert einen weiteren Anwendungsfall der benannten Bedingungen. So lässt sich der Wert der eigentlichen Variable setzen, indem der bedingten Variable der Wahrheitswert \mintinline{cobolfree}{TRUE} zugewiesen wird. Das Ergebnis dieser Zuweisung wird in der Ausgabe von \autoref{88_cobol_listing} in Zeile 3 dargestellt. Zu beachten ist hierbei, dass die meisten COBOL-Compiler nur das Setzen des Wertes \mintinline{cobolfree}{TRUE} erlauben.\\

Dieses Verhalten wird oftmals ausgenutzt um Variablen mit bestimmten Werten zu belegen. Dieses Verhalten beschreibt \autoref{88_cobol_value_set_listing}.\\

\mintedCobol{88_section_value_set.cbl}{Setzen von Werten mithilfe benannter Bedingungen}{88_cobol_value_set_listing}

\subsection*{Abbildung in Java}
Java besitzt kein Sprachkonstrukt, um die Funktionalität der Stufennummer 88 direkt nachzubilden. Eine Möglichkeit gleiches Verhalten darzustellen bietet allerdings die Implementierung spezieller Methoden. Dies soll \autoref{88_java_listing} veranschaulichen.\\

\mintedJava{88_section.java}{Bedingte Werte in Java}{88_java_listing}

Die Methoden \mintinline{text}{isUnderage} und \mintinline{text}{isAdult} geben einen Wahrheitswert in Abhängigkeit des Variablenwertes zurück. Die Funktion \mintinline{text}{setAge} setzt wiederum das Alter.\\

\mintedJava{88_section_value_set.java}{Setzen eines konstanten Wertes mit einem Enum in Java}{88_java_value_set_listing}

Der Anwendungsfall, dass eine benannte Bedingung in COBOL verwendet wird um bestimmte Werte zu setzen lässt sich in Java am elegantesten über den Aufzählungstypen \mintinline{java}{enum} realisieren. Wie der Typ \mintinline{java}{ErrorMessage} in \autoref{88_java_value_set_listing} zeigt, setzt jeder einzelne konstante Wert des Aufzählungstypen eine eigene Fehlernachricht, welche anschließend über die \mintinline{java}{getMessage}-Funktion verfügbar ist.\\