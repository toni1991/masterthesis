\section{Benannte Bedingungen -- Stufennummer 88}
Neben den bereits angesprochenen Stufennummern stellt die \textit{88} eine weitere Besonderheit in COBOL dar. Mit ihr ist es möglich einer Variable einen Wahrheitswert zuzuweisen, der von einem anderen Variablenwert abhängt. Es entsteht eine sogenannte benannte Begingung.

\begin{listing}[H]
  \inputminted{cobol}{Code/88_section.cbl.txt}
  \caption{Beispiel für COBOL Stufennummer 88}
  \label{88_cobol_listing}
\end{listing} 

\autoref{88_cobol_listing} zeigt die beispielhafte Verwendung der Stufennummer 88. Die Variable \mintinline{text}{VAR} kann dabei zweistellige numerische Werte enthalten. Liegt der Wert zwischen 0 und 9 (\mintinline{cobol}{VALUE 0 THRU 9}) so weist die Variable \mintinline{text}{ISLOWERTEN} den Wahrheitswert \mintinline{cobol}{TRUE} auf.

Der so entstandene Wahrheitswert kann folglich immer dann verwendet werden, wenn getestet werden soll, ob die Variable \mintinline{text}{VAR} im Bereich zwischen 0 und 9 liegt.

Zeile 18 des Programms zeigt einen weiteren Anwendungsfall der benannten Bedingungen. So lässt sich der Wert der eigentlichen Variable setzen, indem der bedingten Variable der Wahrheitswerte \mintinline{cobol}{TRUE} zugewiesen wird.

Die Ausgabe des Programms in \autoref{88_cobol_listing} wäre folgende:
\begin{minted}[bgcolor=hellgrau,xleftmargin=20pt,fontsize=\footnotesize]{text}
VAR is over 10.
VAR = 00
\end{minted}

Dieses Verhalten wird oftmals ausgenutzt um Variablen mit bestimmten Werten zu belegen. Dieses Verhalten beschreibt \autoref{88_cobol_value_set_listing}.

\begin{listing}[H]
  \inputminted{cobol}{Code/88_section_value_set.cbl.txt}
  \caption{Setzen von Werten mithilfe benannter Bedingungen}
  \label{88_cobol_value_set_listing}
\end{listing} 

\subsection*{Abbildung in Java}
Java besitzt kein Sprachkonstrukt, um die Funktionalität der Stufennummer 88 direkt nachzubilden. Eine Möglichkeit gleiches Verhalten darzustellen bietet allerdings die Implementierung spezieller getter- und setter-Methoden. Dies soll \autoref{88_java_listing} veranschaulichen.

\begin{listing}[H]
  \inputminted{java}{Code/88_section.java.txt}
  \caption{COBOL Stufennummer 88 in Java}
  \label{88_java_listing}
\end{listing} 

Die Methode \mintinline{text}{getISLOWERTEN} bezieht sich dabei nicht auf eine Variable sondern gibt einen Wahrheitswert in Abhängigkeit des Variablenwertes zurück. \mintinline{text}{setISLOWERTEN} setzt diesen Variablenwert wenn der Wahrheitswert \mintinline{java}{true} ist. Diese Einschränkung in \mintinline{text}{setIS-} \mintinline{text}{LOWERTEN} wurde hinzugefügt, um das Verhalten der meisten COBOL-Compiler nachzubilden, welche nur das Setzen des Wertes \mintinline{cobol}{TRUE} erlauben.

\begin{listing}[H]
  \inputminted{java}{Code/88_section_value_set.java.txt}
  \caption{Setzen eines konstanten Wertes mit benannter Variable in Java}
  \label{88_java_value_set_listing}
\end{listing} 

Der Anwendungsfall, dass eine benannte Bedingung in COBOL verwendet wird um bestimmte Werte zu setzen, lässt sich in Java in verschiedenen Weisen abbilden. \autoref{88_java_value_set_listing} zeigt die wohl einfachste dieser Möglichkeiten.

\begin{listing}[H]
  \inputminted{java}{Code/88_section_value_set2.java.txt}
  \caption{Exakte Abbildung des Setzens einer benannten Bedingung in Java}
  \label{88_java_value_set_listing2}
\end{listing} 

Dabei werden drei statische Konstanten definiert und der Variable \mintinline{text}{errorMessage} die gewünschte zugewiesen. Dieses Konstrukt bildet zwar nicht 1:1 den COBOL-Code nach, in dem es auch möglich wäre Konstanten zu verwenden, jedoch zeigt \autoref{88_java_value_set_listing2}, dass die vorhergehende Lösung in Java eleganter ist.