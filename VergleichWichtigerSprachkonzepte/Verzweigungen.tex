\subsection{Verzweigungen}
Eine wichtige Eigenschaft von Programmiersprachen ist bedingte Verzweigung, sprich die Ausführung von Programmteilen nur unter bestimmten Voraussetzungen. Sowohl Java als auch COBOL bieten hierfür die Schlüsselwörter \jav{if}-\jav{else} (Java) \bzw \cob{IF}-\cob{ELSE}-\cob{END-IF} (COBOL). Auch die Verwendung ist sehr ähnlich, wie folgende Beispiele zeigen.

\mintedJava{IfExample.java}{Verzweigung in Java}{ifJava}

In \autoref{ifJava} wird anhand einer Nutzereingabe eine Fallunterscheidung \bzw Verzweigung gemacht. Dabei soll gezeigt werden, dass es möglich ist sowohl mehrere Zeilen als auch nur eine Zeile konditionell auszuführen. Soll mehr als eine Zeile untergeordnet werden, ist eine Gruppierung als Block -- mit geschweiften Klammern -- nötig. Der \jav{else}-Zweig zeigt eine einzelne Anweisung als bedingt auszuführendes Statement. Das letzte Statement zeigt die Verwendung des konditionalen Operators \quotes{?}. Dabei handelt es sich -- im Gegensatz zu Statements in \jav{if}-\jav{else}-Konstrukten -- um bedingte Expressions, also reine Ausdrücke statt Anweisungen, die abhängig von Wahrheitswerten eingesetzt werden.

\autoref{ifCOBOL} bildet die selbe Logik in COBOL ab. Die beiden Sections \cob{END-IF-EXAMPLE} und \cob{PERIOD-IF-EXAMPLE} zeigen dabei zwei unterschiedliche Wege diese zu konstruieren. Während erstere eine \cob{ELSE}- und eine \cob{END-IF} Anweisung nutzt, um das Konstrukt aufzubauen und zu terminieren, verwendet letztere die Eigenschaft, dass ein \cob{IF} auch durch ein Sentence-Ende -- siehe \autoref{sec:structure} -- abgeschlossen werden kann. Dies erlaubt jedoch keine verschachtelten Verzweigungen und kann -- wie die befragten Experten anmerkten -- in der Praxis schnell zu Fehlern oder zumindest zu schwer durchschaubarem Verhalten führen. Herr Streit betonte, dass bestehende Programme teilweise solche Konstrukte beinhalten, ein \cob{IF} jedoch stets mit einem \cob{END-IF} terminiert werden sollte. Dies sorgt dafür, dass es dem Compiler möglich ist, Fehler in der Verzweigung zu erkennen und eine bessere Lesbarkeit zu erreichen.

\mintedCobol{IF-EXAMPLE.cbl}{Verzweigung in COBOL}{ifCOBOL}

Sowohl in Java als auch in COBOL ist es möglich, arithmetische Ausdrücke in Bedingungen zu verwenden. Während dies in Java üblich ist, wies Herr Streit darauf hin, dass dies in COBOL eher selten verwendet wird, da dabei oftmals nicht ausreichend klar ist, wie viele Nachkommastellen die Ergebnisse dieser Berechnungen tragen.