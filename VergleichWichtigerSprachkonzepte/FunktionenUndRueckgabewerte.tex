\section{Funktionen, Unterprogramme und Rückgabewerte}\label{sec:functionsAndReturnValues}
Ein wichtiger Bestandteil von vielen Programmiersprachen sind Prozeduren und Funktionen. Dabei handelt es sich um Codeabschnitte, die von einer anderen Stelle aus aufgerufen werden können. Im Gegensatz zu Prozeduren, die bestimmte Verarbeitungsschritte durchlaufen, liefern Funktionen dabei noch zusätzlich einen Rückgabewert.

\subsection*{Java}
In Java muss jede Anweisung Teil einer Funktion sein. Wie in \autoref{programmablauf} beschrieben startet ein Java-Programm mit der \jav{main}-Methode. Die Unterscheidung zwischen Methoden und Prozeduren wird in Java gängigerweise nicht getroffen. Oft findet sich auch die Bezeichnung Funktion. Java unterscheidet sich von manchen anderen modernen Sprachen dadurch, dass es keine Methodenschachtelungen erlaubt.

\mintedJava{MethodExample.java}{Methoden in Java}{javaMethods}
\autoref{javaMethods} zeigt verschiedene Methoden. An diesem Beispiel sollen zwei wichtige Konzepte dargestellt werden: 
\begin{itemize}
    \item \textbf{Übergabe von Parametern}\\
    Wie gezeigt erhalten die Funktionen unterschiedliche Parameter. Die Methode \jav{process()} zum Beipiel erhält keinen Parameter, wohingegen \jav{getGreeting(String, String)} zwei Parameter vom Typ \jav{String} erwartet.
    \item \textbf{Rückgabe eines Wertes}\\
    Eine Funktion muss stets den Typ ihres Rückgabewertes definieren. Soll kein Rückgabewert geliefert werden, ist der Typ als \jav{void} zu definieren. Im Gegensatz zu anderen Sprachen kann eine Methode in Java lediglich einen Wert zurückliefern.
\end{itemize}
Der Aufruf einer Funktion erfolgt wie in Zeile 12 gezeigt mittels Methodenname und den zu übergebenen Parametern. Außerdem zeigt sich, dass Funktionen mithilfe des \jav{return}-Statements \textbf{einen} Rückgabewert an die aufrufende Funktion zurückgeben können. Ein \jav{return} kann auch dazu benutzt werden, um Funktionen an anderen Stellen als der letzten Zeile zu verlassen. Eine Funktion ohne Rückgabewert -- \jav{void} -- terminiert implizit in der letzten Zeile und benötigt kein explizites \jav{return}, kann dieses jedoch auch ohne Rückgabewert nutzen, um vorher den Funktionsablauf zu beenden. Eine Methode mit Rückgabewert muss stets einen Wert mit passendem Datentyp durch \jav{return} zurückgeben.

\mintedJava{RecursionExample.java}{Rekursion in Java}{javaRecursion}

Eine elegante Möglichkeit die sich mit Funktionen in Java bietet, ist die Rekursion. Dabei handelt es sich um eine Funktion die sich selbst aufruft. In Java sieht man rekursive Implementierungen selten, obwohl diese oft eleganter und durch weniger Code ausgedrückt werden können als iterative Ansätze. Auf der anderen Seite ist die iterative Implementierung jedoch sicherer, da die rekursive Variante unter Umständen an die nicht fest definierte Grenze der maximalen Rekursionstiefe gelangt. Die Funktionen \jav{facultyRecursive} und \jav{facultyIterative} in \autoref{javaRecursion} berechnen die Fakultät einer übergebenen Zahl rekursiv \bzw iterativ.

\subsection*{COBOL}
Das Konzept einer Funktion existiert in COBOL nicht. Lediglich das Aufrufen einer Section oder eines Paragraphs mithilfe eines \cob{PERFORM} geben ansatzweise ähnliche Möglichkeiten und können daher als Vergleich herangezogen werden. Jedoch können dabei weder Parameter übergeben noch ein Wert zurückgeliefert werden. Darum ist es nötig, Werte, die innerhalb einer Section verwendet oder zurückgeliefert werden sollen, in Variablen zu kopieren. Wie in \autoref{sec:scope} beschrieben sind diese Variablen jedoch immer global innerhalb eines Programms definiert.

Oft bringt das allerdings Probleme mit sich. Zum Beispiel werden in der Praxis oft Variablen, die für etwas anderes gedacht sind, an einer anderen Stelle wiederverwendet. So ist nicht ganz klar, welchen Zweck Variablen erfüllen. Ein weiterer großer Nachteil ist, dass Logik oftmals kopiert und sehr ähnlich nochmals geschrieben werden muss, um auf anderen Daten zu operieren.

Rekursive \cob{PERFORM}-Aufrufe sind zwar syntaktisch möglich, jedoch führt die Ausführung zu einem undefinierten Verhalten des Programms und ist deshalb in jedem Fall zu unterlassen. Es kann quasi festgehalten werden, dass Rekursionen innerhalb eines Programms in COBOL nicht möglich sind. Anders sieht es dabei mit gesamten Programmen aus.

\mintedCobol{recursion.cbl}{Rekursion in COBOL}{recursionCobol}

\autoref{recursionCobol} enthält analog zu gezeigtem Java-Beispiel auch ein Programm, welches rekursiv die Fakultät einer Zahl errechnet und ausgibt. Wichtig ist hierbei vorallem die \cob{RECURSIVE} Definition hinter dem Programmnamen in Zeile 2. Die \cob{WORKING-STORAGE SECTION} enthält dabei Variablen, welche von jeder Instanz des rekursiv aufgerufenen Programms gemeinsam genutzt werden. In \cob{LOCAL-STORAGE SECTION} finden sich Variablen, deren Gültigkeitsbereich sich auf die aktuelle Aufrufinstanz beschränken.

In COBOL ist es, anders als in Java, auch möglich eigenständige Unterprogramme aufzurufen. Wie bereits in \autoref{sec:structure} beschrieben dient die \cob{LINKAGE SECTION} dazu, im Unterprogramm zu definieren, welche Variablen übergeben werden. Mithilfe der \cob{CALL}-Anweisung, des Programmnamens und der Angabe von Variablennamen kann dieses Unterprogramm aufgerufen und die Werte übergeben werden. Wie in \autoref{wiederverwendbarkeit} kann die Angabe des Programmnamens auch pseudo-variabel geschehen, um ein statisches Linken der Programmteile zu vermeiden.

\cobol{CALLING_PROGRAM.cbl}
\cobol{CALLED_PROGRAM.cbl}
\begin{shellwindow}
$ cobc -x -o CALLING_PROGRAM CALLING_PROGRAM.cbl 
$ cobc -m CALLED_PROGRAM.cbl
$ ./CALLING_PROGRAM
22
SUB-PROGRAM got passed: 0022
CALLING PROGRAM: 0044
\end{shellwindow}
\mintedCaption{Unterprogramme in COBOL}{subprogramsCobol}

\autoref{subprogramsCobol} zeigt den Aufruf eines Unterprogramms und die Übergabe von Parametern. Diese Parameter werden per Referenz übergeben. Das heißt, dass Unterprogramme stets auf den gleichen Speicherbereich zugreifen wie das ursprüngliche Program und so auch dessen Daten verändern. Dies gilt es zu beachten, da dadurch ungewünschte Seiteneffekte oder gar Fehler auftreten können. Das Definieren eines Rückgabewertes in einem Unterprogramm ist nicht möglich. Soll also ein Rückgabewert im aufgerufenen Program gesetzt werden, so ist dieser innerhalb der übergebenen Datenstruktur zu setzen, auf die wie erwähnt auch das aufrufende Programm Zugriff hat. 