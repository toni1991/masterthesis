\subsection{Nebenläufigkeit}

In Java ist es möglich und durchaus üblich, Programme nebenläufig zu entwickeln und ablaufen zu lassen. Das heißt, mehrere Threads arbeiten parallel und führen Verarbeitungen -- je nach Hardware nur scheinbar -- gleichzeitig aus. Dabei muss der Entwickler auf die Synchronisation von gemeinsam genutzten Speicherbereichen achten, um gültige Daten zu gewährleisten. Diese nebenläufige Programmierung birgt zwar ein gewisses Fehlerpotential bei der Implementierung, sorgt jedoch dafür, dass Logik tendenziell effizienter ausgeführt wird.

In COBOL ist diese nebenläufige Ausführung nicht möglich. Ein COBOL-Programm führt Verarbeitungsschritte stets sequenziell aus und erlaubt keine parallelen Ausführungen. Durch einen Transaktionsmonitor ist es jedoch möglich, dass verschiedene Programme gleichzeitig ausgeführt werden, die jedoch keine Kenntnis von anderen ausgeführten Programmen haben. Bei einem Transaktionsmonitor handelt es sich um eine Art Middleware, vergleichbar zu Application-Servern in Java, welche Anfragen entgegennimmt, dafür sorgt, dass Ressourcen geöffnet und aufgeräumt werden, auf Host-Systemen Terminal-Masken zur Verfügung stellt und entscheidet, wie viele und welche Programme parallel ausgeführt werden. Ein Beispiel hierfür ist das \textit{Customer Information Control System}, kurz \textit{CICS}. Um aus einem COBOL-Programm Teile des Transaktionsmonitors aufzurufen, gibt es Befehle wie \cob{EXEC CICS}. 