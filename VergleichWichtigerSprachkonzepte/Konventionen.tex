\section{Konventionen}
Im folgenden Abschnitt sollen bestimmte Konventionen dargestellt werden, die in COBOL bzw. Java gelten.

\subsection*{Groß- und Kleinschreibung}
Bei der Programmierung behilft man sich oftmals der Groß- und Kleinschreibung, um ein höheres Maß an Struktur und Lesbarkeit des Codes zu erreichen. 

\subsubsection*{Java}

Wichtig ist hierbei vorneweg zu beachten, dass Java \quotes{case-sensitive} ist, also zwischen Groß- und Kleinbuchstaben unterscheidet, während COBOL \quotes{case-insensitive} ist, also -- außer in Strings -- keinen Unterschied macht.

In Java ist es üblich Methodennamen und veränderbare Variablen mit einem kleinen Buchstaben beginnend zu benennen. Besteht der Name aus mehreren Wörtern so wird dieser im \quotes{camel-case} geschrieben. Das heißt, dass stets Großbuchstaben für den Beginn eines neuen Wortes verwendet werden. Beispiele hierfür wären \mintinline{java}{getAdditionalData()} oder \mintinline{java}{int currentAmountOfMoney}. Klassennamen werden in dem selben Muster geschrieben, beginnen jedoch mit einem Großbuchstaben: \mintinline{java}{ToolBox}. Zu guter letzt sollten konstante Variablen und Werte innerhalb eines Aufzählungstyps durchgehend aus Großbuchstaben bestehen, wobei einzelne Wörter mit einem Unterstrich voneinander getrennt werden (\mintinline{java}{final int MULIPLY_FACTOR = 2}). 

\subsubsection*{COBOL}
\todo{COBOL}

\subsection*{Affixe}
Ein weiteres wichtiges Werkzeug bei der Strukturierung von Programmcode ist das Versehen mit Affixen (Prä- oder Suffixen).

\subsubsection*{Java}
In Java ist es hierbei nicht ratsam Affixe zu verwenden. Da diese jedoch in einigen bestehende Codebasen Verwendung finden, werden an dieser Stelle die üblichsten behandelt.

Oftmals werden Interfaces in Java mit einem vorangestellten \quotes{I} gekennzeichnet. Diese Konvention sorgt jedoch dafür, dass Implementierungen eines Interfaces namentlich nicht immer klar abgegrenzt und definiert sind. Wird beispielsweise ein Interface \mintinline{java}{IOutput} definiert so könnten valide Implementierungen \mintinline{java}{ConsoleOutput} oder \mintinline{java}{Printer-Output} sein. Jedoch erlaubt dies namentlich auch die Interface-Implementierung namens \mintinline{java}{Output}, bei der nicht ausreichend klar ist was der Zweck dieser Klasse ist.

Ein weiterer Codingstil der gelegentlich angewendet wird ist das Nutzen von \quotes{m} und \quotes{s} als Präfix von Variablen. Diese sollen kennzeichnen, dass eine Variable entweder Instanzvariable (\textbf{m}ember) oder statisch (\textbf{s}tatic) ist. Durch die Verwendung von modernen IDEs, die beide farblich unterschiedlich darstellen, und des Schlüsselwortes \mintinline{java}{this}, welches exakt für Referenzen auf Instanzvariablen gedacht ist, werden Variablen jedoch bereits ausreichend gekennzeichnet, sodass der Code durch die Verwendung dieser Präfixe unnötigerweise schwerer lesbar gemacht wird.

\subsubsection*{COBOL}\label{affixCOBOL}

\todo{Ein Statement pro zeile, einrückung}