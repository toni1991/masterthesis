\section{Konventionen}
Im folgenden Abschnitt sollen bestimmte Konventionen dargestellt werden, die in COBOL bzw. Java gelten.

\subsection*{Groß- und Kleinschreibung}
Bei der Programmierung behilft man sich oftmals der Groß- und Kleinschreibung, um ein höheres Maß an Struktur und Lesbarkeit des Codes zu erreichen. 

\subsubsection*{Java}

Wichtig ist hierbei vorneweg zu beachten, dass Java \quotes{case-sensitive} ist, also zwischen Groß- und Kleinbuchstaben unterscheidet, während COBOL \quotes{case-insensitive} ist, also -- außer in Strings -- keinen Unterschied macht.

In Java ist es üblich Methodennamen und veränderbare Variablen mit einem kleinen Buchstaben beginnend zu benennen. Besteht der Name aus mehreren Wörtern so wird dieser im \quotes{camel-case} geschrieben. Das heißt, dass stets Großbuchstaben für den Beginn eines neuen Wortes verwendet werden. Beispiele hierfür wären \mintinline{java}{getAdditionalData()} oder \mintinline{java}{int currentAmountOfMoney}. Klassennamen werden in dem selben Muster geschrieben, beginnen jedoch mit einem Großbuchstaben: \mintinline{java}{ToolBox}. Zu guter letzt sollten konstante Variablen und Werte innerhalb eines Aufzählungstyps durchgehend aus Großbuchstaben bestehen, wobei einzelne Wörter mit einem Unterstrich voneinander getrennt werden (\mintinline{java}{final int MULIPLY_FACTOR = 2}). 

\subsubsection*{COBOL}
COBOL hingegen unterscheidet bei Variablennamen und Schlüsselwörtern nicht zwischen Groß- und Kleinschreibung. Es ist jedoch üblich sowohl Schlüsselwörter als auch Variablennamen komplett groß zu schreiben. 
\todo[inline]{Grund?}

\subsection*{Affixe}
Ein weiteres wichtiges Werkzeug bei der Strukturierung von Programmcode ist das Versehen mit Affixen (Prä- oder Suffixen).

\subsubsection*{Java}
In Java ist es hierbei nicht ratsam Affixe zu verwenden. Da diese jedoch in einigen bestehende Codebasen Verwendung finden, werden an dieser Stelle die üblichsten behandelt.

Oftmals werden Interfaces in Java mit einem vorangestellten \quotes{I} gekennzeichnet. Diese Konvention sorgt jedoch dafür, dass Implementierungen eines Interfaces namentlich nicht immer klar abgegrenzt und definiert sind. Wird beispielsweise ein Interface \mintinline{java}{IOutput} definiert so könnten valide Implementierungen \mintinline{java}{ConsoleOutput} oder \mintinline{java}{Printer-} \mintinline{java}{Output} sein. Jedoch erlaubt dies namentlich auch die Interface-Implementierung namens \mintinline{java}{Output}, bei der nicht ausreichend klar ist was der Zweck dieser Klasse ist.

Ein weiterer Codingstil der gelegentlich angewendet wird ist das Nutzen von \quotes{m} und \quotes{s} als Präfix von Variablen. Diese sollen kennzeichnen, dass eine Variable entweder Instanzvariable (\textbf{m}ember) oder statisch (\textbf{s}tatic) ist. Durch die Verwendung von modernen IDEs, die beide farblich unterschiedlich darstellen, und des Schlüsselwortes \mintinline{java}{this}, welches exakt für Referenzen auf Instanzvariablen gedacht ist, werden Variablen jedoch bereits ausreichend gekennzeichnet, sodass der Code durch die Verwendung dieser Präfixe unnötigerweise schwerer lesbar gemacht wird.

\subsubsection*{COBOL}\label{affixCOBOL}
In COBOL ist es in der Praxis dagegen sehr sinnvoll Affixe zu verwenden. Dadurch, dass es nicht möglich ist lokale Variablen zu definieren erlauben es Präfixe schnell und übersichtlich kenntlich zu machen, welche Variablen zu welchem Programmteil gehören. Dabei handelt es sich laut Experten um eine Best-Practice-Methode, um den Code verständlicher und leichter lesbar zu machen.\\

So ist es üblich den Namen einer \mintinline{cobolfree}{SECTION} oder eines Paragraph mit einem Präfix zu versehen und die darin genutzten Variablen mit demselbigen zu kennzeichnen. Beispielsweise sollte eine Variable \mintinline{cobolfree}{100-VALUE} nur in der \mintinline{cobolfree}{SECTION} \mintinline{cobolfree}{100-PROCESS} verwendet werden. \\

Oft finden sich auch Präfixe welche die Art des Speichers -- z.B. \mintinline{cobolfree}{WS-} für \mintinline{cobolfree}{WORKING-STORAGE} -- kennzeichnen. Eine Benennung nach den genutzten Programmteilen ist jedoch vorzuziehen und sorgt für klarere Struktur und Lesbarkeit.

\subsection*{Schlüsselwörter}
In COBOL ist es möglich bestimmte Schlüsselwörter zu verkürzen. Dabei kann jedoch nicht beliebig gekürzt werden. Es sind lediglich weitere Schlüsselwörter mit selber Funktion definiert, die genutzt werden können. Ein Beispiel dafür ist die \mintinline{cobolfree}{PICTURE}-Anweisung, die auch als \mintinline{cobolfree}{PIC} geschrieben werden kann. Oftmals findet sich in der Praxis COBOL-Code welcher gekürzte Schlüsselwörter verwendet.\\

In Java ist ein verkürzen von Schlüsselwörtern hingegen nicht möglich.

\todo[inline]{Ein Statement pro Zeile, Einrückungen... -> Struktur / Aufbau des Codes (Verweis auch auf das Strukturkapitel, welches Strukturen enthält die nicht konventionell sondern fest vorgegeben sind.)}