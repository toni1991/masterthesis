\section{Problemstellung}\label{problemstellung}
Der folgende Abschnitt soll die Problemstellung verdeutlichen, welche der Arbeit zugrunde liegt. 
Dazu wird erläutert, welche Wichtigkeit COBOL innehat und anschließend mit der Bedeutung, die der Sprache heutzutage in der Lehre tatsächlich beigemessen wird gegenübergestellt. Anhand dessen werden Schwierigkeiten für den Arbeitsmarkt analysiert und beschrieben.

\subsection*{Wichtigkeit von COBOL}\label{wichtigkeit}
\quotes{Viele Millionen Cobol-Programme existieren weltweit und müssen laufend gepflegt werden.
Es ist bei dieser Situation undenkbar und unter wirtschaftlichen Gesichtspunkten unvertretbar in den nächsten Jahren eine Umstellung dieser Programme auf eine andere Sprache durchzuführen.} \cite{_ist_1979}

Was Herr Dr. Strunz neben vielen anderen Experten bereits 1979 prophezeite hat auch heute noch Gültigkeit. Obwohl COBOL zum Ende der 50er Jahre entstand, 1959 veröffentlicht wurde und damit fast 60 Jahre alt ist, trifft man es auch heute in betrieblichen Informationssystemen noch häufig an. In der britischen Tageszeitung \textit{The Guardian}, zitiert der Autor Scott Colvey in seinem Artikel \cite{colvey_cobol_2009} anlässlich des 50. Geburtstages von COBOL den Micro Focus Manager David Stephenson: \quotes{`some 70\% to 80\% of UK plc business transactions are still based on Cobol'}. Weiter führt er darin Aussagen von IBM Software-Leiter Charles Chu an, welcher die Aussagen von Stephenson bestätigt: \quotes{$[\ldots]$ there are 250bn lines of Cobol code working well worldwide. Why would companies replace systems that are working well?'}. Stephen Kelly, Geschäftsführer von Micro Focus, betont zudem, dass sich Stand 2009 über 220 Milliarden COBOL-Codezeilen im produktiven Einsatz befanden, welche vermutlich 80\% der insgesamt weltweit aktiven Codezeilen ausmachten. Außerdem wurden zum damaligen Zeitpunkt, Schätzungen zufolge 200-mal mehr COBOL-Transaktionen ausgeführt als Google Suchanfragen verzeichnen konnte. \cite{kelly_cobol_2009} Diese Aussagen decken sich mit den Angaben in \citeWithTitle{doke_cobol_2005}. Auch darin betonen \citeauthor{doke_cobol_2005}, dass -- Stand 2005 -- mit 225 Milliarden Codezeilen, etwa 70\% des weltweiten Codes in COBOL geschrieben sind.

Daran wird nicht nur deutlich, dass COBOL in den vergangenen Jahren einen enormen Marktanteil ausmachte, sondern auch die weitere Bedeutung der Sprache für die Zukunft: Wieso sollte funktionierender Code mit Hilfe von teuren und riskanten Prozessen ersetzt werden? Da sich viele Unternehmen dieser Frage eines Umstiegs von COBOL auf eine modernere Lösung ausgesetzt sehen, auf die sich nur schwer eine Antwort finden lässt, welche die Risiken und Kosten aufwiegt, stieg die Zahl des weltweit betriebenen COBOL-Codes über die vergangenen Jahre sogar noch weiter an. Dieses Risiko ergibt sich vorrangig durch Transaktionen immenser Geldsummen, die mit COBOL-Systemen durchgeführt werden. Denn \quotes{Täglich werden Transaktionen mit einem Volumen von schätzungsweise drei Billionen Dollar über Cobol-Systeme abgewickelt. Dabei geht es um Girokonten, Kartennetze, Geldautomaten und die Abwicklung von Immobilienkrediten. Weil die Banken aggressiv auf eine Digitalisierung ihres Geschäftes setzen, wird Cobol sogar noch wichtiger. Denn Apps für Smartphones etwa sind in modernen Sprachen geschrieben, müssen aber mit den alten Systemen harmonieren.} \cite{beat_balzli_cobol-programmierer_2017}

Im TIOBE-Index \cite{_tiobe_} für April 2018 rangiert COBOL auf Platz 25 mit einem Rating von 0.541\%. Dieser Index wird auf Basis von Suchanfragen nach den entsprechenden Programmiersprachen, auf den meist frequentiertesten Internetseiten, erstellt. COBOL ist somit zwar nur Teil jeder 200. Suchanfrage, rangiert jedoch damit trotzdem vor anderen etablierten oder aufstrebenden Sprachen wie \textit{Kotlin}, \textit{Scala} oder \textit{Haskell}. Außerdem gilt es hier zu beachten, dass COBOL zu einer Zeit entstand, in der das Internet noch lange nicht existierte und Informationen über die Sprache mittels Büchern verbreitet und vermittelt wurden. Daher ist auch heute noch das Internet nicht die vorrangige Quelle, um Wissen über COBOL zu akquirieren. Unter diesen Gesichtspunkten ist das TIOBE-Rating von COBOL als noch höher einzuschätzen.


\subsection*{Bedeutung in der Lehre}
Da COBOL bereits 60 Jahre alt ist, haben heutzutage bereits viele einstige COBOL-Entwickler das Rentenalter erreicht. Im Artikel \citeWithTitle{beat_balzli_cobol-programmierer_2017} beschreibt der Autor exemplarisch den Fall eines 75-Jährigen Entwickler, der wegen seiner Erfahrung trotz seines Alters immer noch in der Branche tätig ist.

Junge COBOL-Entwickler sind rar, da COBOL nur noch selten Teil der Ausbildung ist. \citeauthor{doke_cobol_2005} führen in \citeWithTitle{doke_cobol_2005} an, dass im Jahr 2002 lediglich für 36.2\% der Studenten COBOL Teil des Gundstudiums war, obwohl im Jahr 1995 noch 89.7\% der befragten Bildungseinrichtungen angaben COBOL-Kurse als festen Bestandteil der Ausbildung zu haben. Sieht man sich dagegen die Zahlen zu Java als Vertreter moderner Programmiersprachen an, lässt sich ein klarer Trend erkennen. Erst 1995 entstanden, stieg die Zahl der  Universitäten, die Java lehrten von 42.5\% im Jahr 1998 auf 90.0\% im Jahr 2002. Spinnt man diesen Wandel, zu dem sich in der Zwischenzeit noch eine Fülle neuerer Sprachen hinzugesellt hat, ins heutige Jahr weiter, lässt sich erahnen, wie selten Lehrveranstaltung zum Thema COBOL inzwischen geworden sind.

Man sieht also, dass sich die Lehre, obwohl der Bedarf an COBOL-Programmierern weiterhin immens ist, stark weg von dem Unterrichten von COBOL fokussiert hat, was die Wirtschaft zusammen mit dem zunehmenden Alter erfahrener COBOL-Entwickler vor Probleme beim Stillen der Nachfrage an Arbeitskräften stellt.

\subsection*{Kontroverse Beurteilungen von COBOL}
Die in \autoref{wichtigkeit} angeführten Aussagen und Meinungen stammen oftmals von Personen aus dem Umfeld von Unternehmen, die teils stark vom Weiterbestehen COBOLs profitieren. Diese Aussagen sind daher, wenn auch sicherlich nicht falsch, vorsichtig und vor allem sehr differenziert zu betrachten.

Der mehrfach prämierte Informatiker \citeauthor{edsger_wybe_dijkstra_how_1975} z.B. findet sehr klare, andere Worte zu COBOL: \quotes{The use of COBOL cripples the mind; its teaching should, therefore, be regarded as a criminal offence.} \cite{edsger_wybe_dijkstra_how_1975}

\citeauthor{florian_hamann_banken_2017} nennt in seinem Artikel \citeWithTitle{florian_hamann_banken_2017} die bereits erwähnte zunehmende Knappheit von Arbeitskräften auch als einen wichtigen Faktor dafür, weshalb COBOL über kurz oder lang von moderneren Systemen und Sprachen verdrängt und abgelöst wird.

Trotz dieser Kontroversen kann festgehalten werden, dass es nach wie vor einen gleichbleibend hohen Bedarf an Entwicklern gibt, den es zu decken gilt. Allerdings entstand die Sprache weit vor wichtigen Entwicklungen und Innovationen in der Informatik und bildet so keine zeitgemäße Grundlage für umfangreiche und noch weniger für neue Systeme.

%Allerdings darf nicht vergessen werden, dass die Sprache wegen ihres hohen Alters vor vielen Entwicklungen und Innovationen in der Informatik entstand. Dadurch ist der Umgang mit der Sprache nicht zeitgemäß, viele Konstrukte, die nicht nur syntaktische Einfachheit bieten, sondern auch performanten und redundanzfreien Code erlauben, sind nicht vorhanden und nachträgliche Erweiterungen des Sprachstandards wirken 