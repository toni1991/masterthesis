\section{Aufbau der Arbeit}
Um leichter Verständnis über die vorliegende Arbeit zu erlangen, wird im Folgenden kurz auf den Aufbau dieser eingegangen.

In \autoref{ch:einleitung} wird die Motivation zum Bearbeiteten des Problems und das damit verfolgte Ziel erklärt. 

Die verwendete Literatur und das Vorgehen bei der Erstellung der Arbeit werden in \autoref{ch:methodik} erläutert.

\autoref{ch:herausforderungen} behandelt die grundlegenden Herausforderungen bei der Entwicklung von betrieblichen Informationssystemen und zeigen wie sich in COBOL und Java die Problemstellungen adressieren lassen.

Die wichtigsten Sprachmittel und Konzepte werden in \autoref{ch:sprachkonzepte} aufgezeigt und gegenübergestellt.

\autoref{ch:cobolinjava} veranschaulicht wichtige und häufig auftretende Muster der Sprachen und zeigt wie und ob diese in der jeweils anderen abgebildet werden.

Das \autoref{ch:fazit} beinhaltet eine Zusammenfassung und Interpretation der Arbeit.