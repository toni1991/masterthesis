\section{Aufbau der Arbeit}
Um leichter Verständnis über die vorliegende Arbeit zu erlangen, wird im Folgenden kurz auf den Aufbau dieser eingegangen.

In \autoref{ch:einleitung} wird die Motivation zum Bearbeiteten des Problems und das damit verfolgte Ziel erklärt. 

Die verwendete Literatur und das Vorgehen bei der Erstellung der Arbeit werden in \autoref{ch:methodik} erläutert.

\autoref{ch:herausforderungen} behandelt die grundlegenden Unterschiede der Sprachkonzepte von COBOL sowie Java und zeigt die Herausforderungen für Entwickler auf, die mit Kenntnissen einer der Sprachen als Grundlage die jeweils andere erlernen.

Die wichtigsten Sprachmittel und Konzepte werden in \autoref{ch:sprachkonzepte} aufgezeigt und gegenübergestellt.

\autoref{ch:cobolinjava} veranschaulicht wichtige COBOL-Konstrukte und skizziert wie diese in Java abgebildet werden können.

Das \autoref{ch:fazit} beinhaltet eine Zusammenfassung und Interpretation der Arbeit.