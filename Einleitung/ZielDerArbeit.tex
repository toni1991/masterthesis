\section{Ziel der Arbeit}
Die vorliegende Arbeit soll einen Beitrag zur Lösung der in \autoref{problemstellung} beschriebenen Probleme leisten. Dies soll mit Hilfe eines Leitfadens geschehen, der fachkundigen Java-Entwicklern den Einstieg in COBOL erleichtert, indem gängige Sprachkonzepte gegenübergestellt und verglichen werden. 

So soll es möglich sein, vorhandenes Wissen über die Softwareentwicklung, im speziellen mit Java, in einen COBOL-Kontext zu bringen und passende Sprachkonzepte nutzen zu lernen. Des Weiteren soll aufgezeigt werden, welche konzeptuellen Herausforderungen sich bei der COBOL-Entwicklung und -Migration ergeben.

Im Fokus steht hierbei neben der Einführung in relevante Sprachkonstrukte stets auch die Experteneinschätzung zur Nutzung der verschiedenen Konzepte. Daher wird, wenn möglich, zusätzlich zu den erklärten Paradigmen erläutert, wie die Verwendung in der Praxis aussehen \bzw nicht aussehen sollte und je nach Sprachmittel gegebenenfalls in der Praxis zu verwendende Alternativen aufgezeigt.

Ziel der Arbeit ist jedoch nicht, vorhandene Java-Entwickler zu Neuentwicklungen mit COBOL zu animieren oder diese gar zu COBOL-Entwicklern umzuschulen. Wichtig ist in diesem Zusammenhang vielmehr ihr Wissensspektrum so zu erweitern, dass es ihnen möglich wird, komplexe fachliche Zusammenhänge -- vor allem die \quotes{business logic} -- bestehender COBOL-Architekturen zu erkennen und zu verstehen. Dadurch sind diese Entwickler fortan flexibler einsetzbar und geschult, um mit Migrations-, Renovierungs- und Wartungsaufgaben von COBOL-Systemen betraut werden zu können.