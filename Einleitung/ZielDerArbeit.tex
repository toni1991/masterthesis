\section{Ziel der Arbeit}
Die vorliegende Arbeit leistet einen Beitrag zur Lösung der in \autoref{problemstellung} beschriebenen Probleme. Dies geschieht mithilfe eines Leitfadens, der fachkundigen Java-Entwicklern den Einstieg in COBOL erleichtert, indem gängige Sprachkonzepte gegenübergestellt und verglichen werden. 

Das ermöglicht es, vorhandenes Wissen über Softwareentwicklung, im speziellen mit Java, in einen COBOL-Kontext zu bringen und passende Sprachkonzepte nutzen zu lernen. Des Weiteren wird aufgezeigt, welche konzeptuellen Herausforderungen sich bei der COBOL-Entwicklung und -Migration ergeben.

Im Fokus steht hierbei neben der Einführung in relevante Sprachkonstrukte stets auch die Experteneinschätzung zur Nutzung der verschiedenen Konzepte. Daher wird, wenn möglich, zusätzlich zu den erklärten Paradigmen erläutert, wie die Verwendung in der Praxis aussehen \bzw nicht aussehen sollte und je nach Sprachmittel gegebenenfalls in der Praxis zu verwendende Alternativen aufgezeigt.

Ziel der Arbeit ist jedoch nicht, vorhandene Java-Entwickler zu Neuentwicklungen mit COBOL zu animieren oder diese gar zu COBOL-Entwicklern umzuschulen. Wichtig ist in diesem Zusammenhang vielmehr ihr Wissensspektrum so zu erweitern, dass es ihnen möglich wird, komplexe fachliche Zusammenhänge -- vor allem die \quotes{business logic} -- bestehender COBOL-Architekturen zu erkennen und zu verstehen. Dadurch sind diese Entwickler fortan flexibler einsetzbar und geschult, um mit Migrations-, Renovierungs- und Wartungsaufgaben von COBOL-Systemen betraut werden zu können.

Neben den praxisrelevanten Aspekten erfasst diese Arbeit COBOL konzeptuell und bringt die Sprache so in einen universitären Kontext. Dies dient dem Zweck die zugrunde liegenden Ansätze der Sprache, statt der syntaktischen Elemente zu untersuchen und mit denen einer modernen Sprache wie Java gegenüberzustellen. Außerdem werden dabei bekannte, in der Praxis häufig zu findende Muster untersucht und verglichen, um Aufschluss darüber zu geben, wie stark sich Theorie und Praxis unterscheiden. \todo{passt das so?}