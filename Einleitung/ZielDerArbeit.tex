\section{Ziel der Arbeit}
Die vorliegende Arbeit soll einen Beitrag zur Lösung der in \autoref{problemstellung} beschriebenen Probleme leisten. Dies soll mit Hilfe eines Leitfadens geschehen, der fachkundigen Java-Entwicklern den Einstieg in COBOL erleichtert, indem gängige Sprachkonzepte gegenübergestellt und verglichen werden. 
So soll es möglich sein vorhandenes Wissen über Softwareentwicklung in einen COBOL-Kontext zu bringen und passende Sprachkonzepte nutzen zu lernen. 
Des weiteren soll aufgezeigt werden, welche konzeptuellen Herausforderungen sich bei der COBOL-Entwicklung und -Migration ergeben

Ziel der Arbeit ist es jedoch nicht vorhandene Java-Entwickler umzuschulen und zu COBOL-Entwicklern zu machen. Wichtig ist vielmehr ihr Wissensspektrum so zu verbreitern, dass es ihnen möglich wird flexibler eingesetzt zu und mit Migrations- und Renovierungsaufgaben betraut zu werden.