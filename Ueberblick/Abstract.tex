\chapter*{Abstract} \addcontentsline{toc}{chapter}{Abstract}

COBOL, which stands for the \quotes{\textbf{Co}mmon \textbf{B}usiness \textbf{O}riented \textbf{L}anguage}, began to rise in the late 50s as a project of the US government. The aim was to design a language, which enables persons without knowledge of programming to engineer software systems. At that time nobody could forebode, that it's still used in many existing operational information systems and not uncommon that these have different requirements than modern desktop or web applications regarding the environment and development. So, whilst COBOL is getting less and less attention of new programmers the demand for highly trained and experienced professionals is high yet. This thesis outlines key challenges in terms of those operational information systems and reveals how COBOL copes with them. Furthermore, COBOL gets conceptually surveyed and compared with Java, which represents state of the art programming languages. The practical approach is always on focus in this comparison, and therefore, along with available literature, experts were interviewed to get the best possible insight of development and maintenance of those systems. The purpose was to devise a guide for Java developers, that enables them to familiarize with COBOL by contrasting known concepts, pattern and constructs. The interviews led to best practice advices in combination with those concept descriptions and hints on how those are used in practice and how they should be.