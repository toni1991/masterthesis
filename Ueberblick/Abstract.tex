\chapter*{Abstract} \addcontentsline{toc}{chapter}{Abstract}

COBOL, which stands for the \quotes{\textbf{Co}mmon \textbf{B}usiness \textbf{O}riented \textbf{L}anguage}, began to rise in the late 1950s and therefore offers just a few of modern language concepts. Because of that education tends to focus on teaching programming languages that provide more modern object oriented concepts. In contrast, COBOL is still used in many existing operational information systems, which have to be maintained and extended. So, whilst COBOL is getting less and less attention in the education of new programmers, the demand for highly trained and experienced professionals is still high. 

This thesis outlines key challenges in terms of those operational information systems and reveals how COBOL copes with them. Furthermore, COBOL gets surveyed and conceptually compared to Java, which represents state of the art programming languages. The practical approach is always on focus in this comparison, and therefore, along with available literature, experts were interviewed to get the best possible insight of development and maintenance of those systems. The purpose was to devise a guide for Java developers, which enables them to familiarize with COBOL by contrasting known concepts, pattern and constructs. The interviews led to best practice advice in combination with those concept descriptions and hints on how those are used in practice and how they should be used.