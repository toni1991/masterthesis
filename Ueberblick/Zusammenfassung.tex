\chapter*{Zusammenfassung} \addcontentsline{toc}{chapter}{Zusammenfassung}

Ende der 50er Jahre entstand die \quotes{\textbf{Co}mmon \textbf{B}usiness \textbf{O}riented \textbf{L}anguage}, kurz COBOL genannt, als Projekt der US Regierung. Ziel war es eine Sprache zu entwerfen, die es auch Menschen ohne informationstechnische Ausbildung ermöglichte Programme zu schreiben. Was damals nicht abzusehen war, ist der Einfluss, den COBOL auch gegenwärtig noch in betrieblichen Informationssystemen hat. Diese Systeme stellen oftmals andere Anforderungen an Umgebungen und Entwickler, als moderne Desktop- oder Webanwendungen. Während diese Sprache heutzutage also immer seltener zum Repertoire von Programmierern gehört, ist durch die Vielzahl vorhandener COBOL-Systeme, die Nachfrage nach Experten weiterhin hoch. Diese Arbeit gibt einen Überblick über Herausforderungen die sich in Verbindung mit betrieblichen Informationssystemen ergeben und zeigt wie COBOL diesen Problemen begegnet. Ferner wird versucht COBOL konzeptuell zu erfassen und mit Java, als Vertreter moderner Sprachen, zu vergleichen. Dabei steht stets die praktische Anwendung der Sprachen im Vordergrund, weshalb Experteninterviews geführt wurden, um neben bestehender Fachliteratur bestmögliche Einsicht in die Entwicklung und Wartung angesprochener Systeme zu erhalten. Damit entstand ein Leitfaden, der es Programmierern mit Java-Kenntnissen erlaubt sich mit COBOL vertraut zu machen, indem bekannte Konzepte, Muster und Konstrukte gegenübergestellt werden. Zusätzlich wird, als Ergebnis der Experteninterviews, darauf hingewiesen wie sich der Umgang mit diesen Konzepten in der Praxis gestaltet \bzw gestalten sollte. 