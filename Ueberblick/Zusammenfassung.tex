\chapter*{Zusammenfassung} \addcontentsline{toc}{chapter}{Zusammenfassung}

Da die \quotes{\textbf{Co}mmon \textbf{B}usiness \textbf{O}riented \textbf{L}anguage}, kurz \mbox{COBOL} genannt, bereits Ende der 1950er Jahre entstand und daher nur wenige moderne Sprachkonzepte bietet, wird der Fokus in der Ausbildung neuer Informatiker immer mehr auf Programmiersprachen mit moderneren objektorientierten Konzepten gelegt. Dem steht gegenüber, dass COBOL immer noch wichtiger Bestandteil bestehender betrieblicher Informationssysteme ist, die es zu warten und zu erweitern gilt. Während diese Sprache heutzutage also zunehmend seltener Teil der Ausbildung von Programmierern ist, besteht, durch die Vielzahl vorhandener COBOL-Systeme, weiterhin eine hohe Nachfrage nach Experten.

Diese Arbeit gibt einen generellen Überblick über Herausforderungen, die sich in Verbindung mit betrieblichen Informationssystemen ergeben, und zeigt, wie COBOL und Java diesen Problemen begegnet. Ferner wird COBOL konzeptuell erfasst und mit Java, als Vertreter moderner Sprachen, verglichen. Dabei steht stets die praktische Anwendung der Sprachen im Vordergrund, weshalb Experteninterviews geführt wurden, um neben bestehender Fachliteratur bestmögliche Einsicht in die Entwicklung und Wartung angesprochener Systeme zu erhalten. Damit entstand ein Leitfaden, der es Programmierern mit Java-Kenntnissen erlaubt, sich mit COBOL vertraut zu machen, indem bekannte Konzepte, Muster und Konstrukte gegenübergestellt werden. Zusätzlich wird, als Ergebnis der Experteninterviews, darauf hingewiesen, wie sich der Umgang mit diesen Konzepten in der Praxis gestaltet und gestalten sollte. 