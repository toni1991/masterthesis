\section{Experten-Interviews}\label{interviews}
Die vorliegende Arbeit soll vorhandene Expertise von Experten nutzen, um statt einem Nachschlagewerk für syntaktische Zwecke einen Leitfaden zu erarbeiten der von praktischer Relevanz getrieben die wichtigen Feinheiten von COBOL und Java beleuchtet und gegenüberstellt. Daher stellen neben den angesprochenen literarischen Quellen, vor allem Experteninterviews einen Kernpunkt dieser Arbeit dar. Um den Praxisbezug zu gewährleisten wurden diese geführt, transkribiert und darauf aufbauend relevante Themenbereiche und Praktiken ausgemacht und analysiert.\\

Interviewt wurden Experten aus dem Hause der \textbf{itestra GmbH}, die sich als \quotes{ein international anerkannter Kompetenzträger in der Neuentwicklung von Individualsoftware sowie der Optimierung und Renovierung bestehender Lösungen}\footnote{https://www.kununu.com/de/itestra1} versteht. Die Mitglieder der Entwicklerteams können dabei auf mehrjährige Erfahrungen im Bereich der Renovierung und dem Reengineering von COBOL-Systemen blicken, was die befragten Personen zur wohl wichtigsten Quelle dieser Arbeit macht.\\

Befragt wurden die drei kundigen COBOL-Entwickler \textit{Ivaylo Bonev}, \textit{Jonathan Streit} und \textit{Thomas Lamperstorfer}. Dabei ging es nicht um repräsentative Meinungen zu bestimmten vorausgewählten Fragen, sondern um die individuelle Einschätzung der Schwierigkeiten und Stolpersteine bei der Entwicklung, Wartung und dem Verständnis von bestehenden und neuen COBOL-Systemen, sowie der Einschätzung zu Parallelen und Diskrepanzen mit Java. \\

Daher wurde kein Fragenkatalog ausgearbeitet sondern offener Input der erwähnten Personen gefordert, um die gewünschten subjektiven Meinungen zu bekommen. Abschriften dieser Interviews finden sich in \autoref{appendixInterviews}.