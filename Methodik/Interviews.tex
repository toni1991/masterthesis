\section{Experten-Interviews}\mylabel{interviews}
Die vorliegende Arbeit nutzt vorhandenes Expertenwissen, um, statt eines Nachschlagewerks für syntaktische Zwecke, einen Leitfaden zu erarbeiten -- der von praktischer Relevanz getrieben -- die wichtigsten Eigenschaften von COBOL und Java beleuchtet und gegenüberstellt. Daher stellen neben den angesprochenen literarischen Quellen, vor allem Experteninterviews einen Kernpunkt dieser Arbeit dar. Um den Praxisbezug zu gewährleisten, wurden diese geführt, transkribiert und darauf aufbauend relevante Themenbereiche und Praktiken ausgemacht und analysiert.

Interviewt wurden Experten aus dem Hause der \textbf{itestra GmbH}. Diese \quotes{Mitarbeiter kombinieren eine exzellente Informatik-Ausbildung mit Branchen-Know-how}\footref{itestraFootnote}, \quotes{kennen sowohl Legacy-Technologien wie Assembler, RPG und COBOL als auch Java, JS, C\# und iOS}\footref{itestraFootnote} und \quotes{verstehen alte Systeme und setzen moderne Technologien ein}\footnote{\mylabel{itestraFootnote}\url{https://itestra.com/leistungen/software-renovation/} \visitedOn}. Die Mitglieder der Entwicklerteams können dabei auf mehrjährige Erfahrungen im Bereich der Renovierung und dem Reengineering von COBOL-Systemen blicken, was die befragten Personen zur wohl wichtigsten Quelle dieser Arbeit macht.

Befragt wurden die drei kundigen COBOL-Entwickler \textit{Ivaylo Bonev}, \textit{Jonathan Streit} und \textit{Thomas Lamperstorfer}. Dabei ging es nicht darum, eine repräsentative Stichprobe nach statistischem Vorgehen zu erheben, sondern darum eine individuelle Bewertung der Schwierigkeiten und Stolpersteine bei der Entwicklung, Wartung und dem Verständnis von bestehenden und neuen COBOL-Systemen sowie eine Einschätzung zu Parallelen und Diskrepanzen mit Java zu erhalten. Daher wurde kein Fragenkatalog ausgearbeitet, sondern offener Input gefordert, um die gewünschten subjektiven Meinungen zu bekommen. 

Der Umfang der Interviews beläuft sich auf 3 Stunden Audiomaterial \bzw 30 A4-Seiten Transkription in der ersten Phase und zusätzlichem Feedback in der Zwischenphase, bei dem die Experten diese Arbeit beurteilt und weitere Anregungen gegeben haben.