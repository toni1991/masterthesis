\section{Entwicklungsumgebungen}
Um Codebeipiele für diese Arbeit zu erstellen, zu kompilieren und auszuführen wurden jeweils für Java und COBOL IDEs verwendet.\\

Für Java-Code wurde die bekannte Eclipse\footnote{http://www.eclipse.org/} Umgebung verwendet. Dabei handelt es sich um einen etablierte IDE, welche eine Vielzahl von Funktionen zur Entwicklung und zum Debugging liefert.\\

Der COBOL-Code dieser Arbeit wurde in der OpenCobolIDE\footnote{https://github.com/OpenCobolIDE/OpenCobolIDE} entwickelt. Dabei handelt es sich um eine minimalistische IDE, welche zum Beispiel Syntax-Highlighting oder eine übersichtliche Darstellung von Fehlern bietet. Der darunterliegende Compiler GnuCOBOL\footnote{https://sourceforge.net/projects/open-cobol/} wurde jedoch auch teilweise direkt als Kommandozeilenwerkzeug ausgeführt. Im Gegensatz zur sonst üblichen COBOL-Entwicklung auf einem Hostsystem ermöglicht dieser Compiler das erzeugen von ausführbaren Dateien für gängige Linux Betriebssysteme. Dies war in dieser Arbeit sehr wichtig, um nicht auf ein System angewiesen zu sein, welches meist nur ein reeller Kunde in Betrieb hat und der Zugriff oft mühsam, teuer oder schlichtweg nicht möglich ist.