\section{Entwicklungsumgebungen}
Um Codebeipiele für diese Arbeit zu erstellen, zu kompilieren und auszuführen, wurden jeweils für Java und COBOL IDEs verwendet.

Für Java-Code wurde die bekannte Eclipse\footnote{http://www.eclipse.org/} Umgebung verwendet. Dabei handelt es sich um einen etablierte IDE, welche eine Vielzahl von Funktionen zur Entwicklung und zum Debugging liefert.

Der COBOL-Code dieser Arbeit wurde in der OpenCobolIDE\footnote{https://github.com/OpenCobolIDE/OpenCobolIDE} entwickelt. Dabei handelt es sich um eine minimalistische IDE, welche zum Beispiel Syntax-Highlighting oder eine übersichtliche Darstellung von Fehlern bietet. Der darunterliegende Compiler GnuCOBOL\footnote{https://sourceforge.net/projects/open-cobol/} wurde jedoch auch teilweise direkt als Kommandozeilenwerkzeug ausgeführt. Im Gegensatz zur sonst üblichen COBOL-Entwicklung auf einem Hostsystem ermöglicht dieser Compiler das Erzeugen von ausführbaren Dateien für gängige Linuxsysteme. Dies war in dieser Arbeit sehr wichtig, um nicht auf ein System angewiesen zu sein, welches meist nur in Produktivumgebungen betrieben wird und zu dem der Zugriff oft mühsam und -- durch die verschiedenen Abrechnungsmodelle dieser Hostrechner -- teuer oder schlichtweg nicht möglich ist.

\subsection*{Das erste COBOL-Programm}

Um bereits an dieser Stelle einen kleinen Einblick in COBOL, die Programmierung und die Ausführung mit der OpenCobolIDE zu bekommen, wird ein kurzes COBOL-Programm implementiert. Die einzelnen Bestandteile davon werden im Laufe der Arbeit genauer beschrieben.

\cobolNotFree{newCobol.cbl}\mintedCaption{Erstellen eines neuen COBOL Programms}{firstCOBOLProgram}

Legt man in der OpenCobolIDE ein neues Programm an, so enthält die Datei das bekannte \quotes{Hello world} als Beispielprogramm. Wir haben dieses Programm nun so erweitert, dass es die Eingabe eines Benutzernamens erwartet und eine persönliche Begrüßung ausgibt. \autoref{firstCOBOLProgram} zeigt das fertige Programm.

Als erstes wird die \cob{PROGRAM-ID} festgelegt. Dies ist der Programmname, wie er auch nach außen -- für eventuelle andere Programme -- sichtbar wird, und sollte daher eindeutig sein. Wichtig hierbei ist auch das Setzen des richtigen Namens in der letzten Zeile, die das \cob{END PROGRAM} enthält.

Anschließend wurde eine Variable mit dem Namen \cob{USERNAME} angelegt, die aus 20 alphanumerischen Zeichen (\cob{PIC X(20)}) besteht und mit Leerzeichen (\cob{VALUE SPACES}) initialisiert wird.

Mittels \cob{DISPLAY} wird der Nutzer aufgefordert seinen Namen einzugeben, den das \cob{ACCEPT}-Schlüsselwort dann in die angesprochene Variable schreibt.

\begin{shellwindow}
$ cobc -x HELLO_USER.cbl
$ ./HELLO_USER
Your name: Toni
Hello Toni
\end{shellwindow}
\mintedCaption{Erstes COBOL-Programm in der Kommandozeile}{cobc}

Anschließend wird geprüft, ob der Nutzer eine Eingabe gemacht hat. Ist dies der Fall, wird eine persönliche Begrüßung ausgegeben. Andernfalls erscheint die generische Meldung \quotes{\cob{Hello world}}. Diese Ausgaben werden wie bereits die Eingabeaufforderung mit \cob{DISPLAY} ausgegeben.

Kompiliert wird das Programm nun mit den Tasten \textit{F8} (kompilieren) \bzw \textit{F5} (kompilieren und ausführen). Soll der GnuCOBOL-Compiler direkt ausgeführt werden, reicht ein einfaches Kommando, um eine ausführbare Datei zu erstellen. Dies wird in \autoref{cobc} dargestellt.