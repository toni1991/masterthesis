\section{Vorhandene Literatur}
Die aufgeführte Literatur gibt oftmals einen sehr detaillierten Einblick in COBOL und bietet Hilfestellungen mit \quotes{Nachschlage-Charakter}. So wie beispielsweise in \citeWithTitle{budlong_teach_1997} oder \citeWithTitle{university_of_limerick_department} werden häufig möglichst viele der vorhandenen COBOL-Konstrukte vorgestellt, mit Beispielen beschrieben und so ihre Verwendung gezeigt. 

Neuere Literatur wie \citeWithTitle{stern_cobol_2006} betrachtet häufig zusätzliche Neuerungen wie die objektorientierte Verwendung von COBOL. \citeWithTitle{rozanski_cobol_2004} 
hingegen stellt mehr ein Syntax-Wörterbuch dar als eine wirkliche Beschreibung oder Einführung in COBOL.

\citeWithTitle{coughlan_beginning_2014} bietet den wohl umfassendsten Überblick, sowie ausführliche Beispiele und Erklärungen zur Verwendung und wirkt dabei nicht wie ein klassisches Nachschlagewerk, sondern wie ein klar strukturiertes Fachbuch, das mit einem klaren roten Faden durch die Bestandteile von COBOL führt. Dabei zieht es vor allem in der Einführung an einigen, wenigen Stellen Parallelen zu Java. 

Alle diese Werke setzen ein gewisses generelles Vorwissen im Bereich der Programmierung und Informatik voraus, was auch in dieser Arbeit der Fall ist. Jedoch ist an nur wenigen Stellen ein vergleichender Charakter zu anderen Sprachen zu erkennen und sehr selten die Erwähnung der jeweiligen Praxisrelevanz oder der besten Einsatzmöglichkeiten entsprechender Konstrukte zu finden. 

In dieser Arbeit wird dagegen bewusst nur selten in die Tiefe der einzelnen Bestandteile gegangen und alle möglichen Verwendungsarten beschrieben, sondern die praktisch relevanten Aspekte beleuchtet. Für einen tieferen Einblick in die gesamten Sprachfeinheiten bietet sich die genannte Literatur an, welche auch bei der Erstellung der Inhalte als Informationsquelle genutzt wurde.

\citeWithTitle{doke_cobol_2005} basiert auf den gleichen Ideen, versucht jedoch, Personen mit fundierten COBOL-Kenntnissen die Entwicklung in Java beizubringen, indem Konzepte gegenübergestellt werden. Teilweise wichtige Details werden dabei ausgelassen oder an manchen Stellen missverständlich beschrieben, weshalb dieses Buch aus fachlicher Sicht zwar als eine gute Brücke von COBOL zu Java einzuschätzen ist, aber nicht als einzige Quelle dienen sollte. Den Brückenschlag in die andere Richtung -- von Java zu COBOL -- lässt es hingegen nicht ohne weiteres zu, was nicht zuletzt daran liegt, dass -- wie der Name bereits andeutet -- ein großer Teil des Buches grafischen Oberflächen mit \textit{Swing} gewidmet ist.

Im Gegensatz zu vorherigem steigt \citeWithTitle{byrne_java_2009-1} tiefer in die syntaktischen Konstrukte von Java ein und behandelt die JavaEE (Enterprise Edition). Außerdem zielt es weniger auf grafische Systeme ab als auf solche, die Daten verarbeiten. So werden verschiedene Ein- und Ausgabe-Mechanismen erklärt, der Umgang mit XML-Formaten beschrieben und -- als Teil der Java EE -- die Verwendung von Datenbanken erläutert.

\citeWithTitle{doke_cobol_2005} und \citeWithTitle{byrne_java_2009-1} ermöglichen COBOL-Entwicklern den schnellen Einstieg in Java und bieten durch die jeweils unterschiedlichen Schwerpunkte einen guten Überblick. Nicht nur, dass in diesen beiden Quellen die Sicht -- Java lernen als COBOL-Entwickler -- eine andere als in dieser Arbeit ist, auch wird nur selten \bzw gar nicht auf die praktische Relevanz der beschriebenen Konstrukte eingegangen. Dadurch fehlt der Charakter eines Leitfadens, welcher diese Arbeit prägt. 