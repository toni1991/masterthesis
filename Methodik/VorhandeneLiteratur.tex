\section{Vorhandene Literatur}
Die aufgeführte Literatur gibt oftmals mehr einen gesamtheitlichen Einblick in COBOL und bietet Hilfestellungen mit \quotes{Nachschlage-Charakter}. So wie in beispielsweise \citeWithTitle{budlong_teach_1997}, \citeWithTitle{university_of_limerick_department}  oder \citeWithTitle{jia_walker_cobol_2004} werden häufig viele der möglichen Konstrukte in COBOL vorgestellt, mit Beispielen beschrieben und so ihre Verwendung gezeigt. 

Neuere Literatur wie \citeWithTitle{stern_cobol_2006} betrachten dabei häufig zusätzlich Neuerungen wie die objektorientierte Verwendung von COBOL. \citeWithTitle{rozanski_cobol_2004} 
hingegen stellt mehr ein Syntax-Wörterbuch dar, als eine wirkliche Beschreibung oder Einführung in COBOL.

\citeWithTitle{coughlan_beginning_2014} bietet den wohl umfassensten Überblick, ausführliche Beispiele und Erklärungen zur Verwendung und wirkt dabei nicht wie ein klassisches Nachschlagewerk sondern wie ein klar strukturiertes Fachbuch das jedoch mit einem klaren roten Faden durch die Bestandteile von COBOL führt. Dabei zieht es vor allem in der Einführung auch an einigen wenigen Stellen Parallelen zu Java. 

Alle dieser Werke setzen ein ein gewisses generelles Vorwissen im Bereich der Programmierung und Informatik voraus, was auch in dieser Arbeit der Fall sein soll. Jedoch ist an nur wenigen Stellen ein vergleichender Charakter zu anderen Sprachen zu erkennen und sehr selten die Erwähnung der jeweiligen Praxisrelevanz oder der besten Einsatzmöglichkeiten zu finden. 

Diese Arbeit soll daher nicht die vielfältig bestehende Literatur um ein weiteres ähnliches Werk ergänzen sondern wichtige Bestandteile der Sprachen gegenüberstellen und Vorgehensweisen bei der Entwicklung aufzeigen. Dabei wird bewusst nur selten in die Tiefe der einzelnen Bestandteile eingegangen und alle möglichen Verwendungsarten beschrieben, sondern versucht praktisch relevante Aspekte zu beleuchten. Für einen tieferen Einblick in die gesamten Sprachfeinheiten bietet sich die bereits erwähnte Literatur an, welche auch bei der Erstellung der Inhalte als Informationsquelle genutzt wurde.
\todo[inline]{\citeWithTitle{byrne_java_2009}}
\todo[inline]{\citeWithTitle{doke_cobol_2005}}