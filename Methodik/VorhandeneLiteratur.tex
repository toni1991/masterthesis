\section{Vorhandene Literatur}
Die aufgeführte Literatur gibt oftmals mehr einen detaillierten Einblick in COBOL und bietet Hilfestellungen mit \quotes{Nachschlage-Charakter}. So wie in beispielsweise \citeWithTitle{budlong_teach_1997}, \citeWithTitle{university_of_limerick_department}  oder \citeWithTitle{jia_walker_cobol_2004} werden häufig möglichst viele der möglichen COBOL-Konstrukte vorgestellt, mit Beispielen beschrieben und so ihre Verwendung gezeigt. 

Neuere Literatur wie \citeWithTitle{stern_cobol_2006} betrachten dabei häufig zusätzlich Neuerungen wie die objektorientierte Verwendung von COBOL. \citeWithTitle{rozanski_cobol_2004} 
hingegen stellt mehr ein Syntax-Wörterbuch dar, als eine wirkliche Beschreibung oder Einführung in COBOL.

\citeWithTitle{coughlan_beginning_2014} bietet den wohl umfassensten Überblick, ausführliche Beispiele und Erklärungen zur Verwendung und wirkt dabei nicht wie ein klassisches Nachschlagewerk sondern wie ein klar strukturiertes Fachbuch das jedoch mit einem klaren roten Faden durch die Bestandteile von COBOL führt. Dabei zieht es vor allem in der Einführung auch an einigen wenigen Stellen Parallelen zu Java. 

Alle dieser Werke setzen ein ein gewisses generelles Vorwissen im Bereich der Programmierung und Informatik voraus, was auch in dieser Arbeit der Fall sein soll. Jedoch ist an nur wenigen Stellen ein vergleichender Charakter zu anderen Sprachen zu erkennen und sehr selten die Erwähnung der jeweiligen Praxisrelevanz oder der besten Einsatzmöglichkeiten zu finden. 

Diese Arbeit soll nicht die vielfältig bestehende Literatur um ein weiteres ähnliches Werk ergänzen sondern wichtige Bestandteile der Sprachen gegenüberstellen und Vorgehensweisen bei der Entwicklung aufzeigen. Dabei wird bewusst nur selten in die Tiefe der einzelnen Bestandteile eingegangen und alle möglichen Verwendungsarten beschrieben, sondern versucht praktisch relevante Aspekte zu beleuchten. Für einen tieferen Einblick in die gesamten Sprachfeinheiten bietet sich die bereits erwähnte Literatur an, welche auch bei der Erstellung der Inhalte als Informationsquelle genutzt wurde.

\citeWithTitle{doke_cobol_2005} basiert auf den gleichen Ideen. Jedoch versucht dieses Buch Personen mit fundierten COBOL-Kenntnissen die Entwicklung in Java beizubringen indem Konzepte gegenübergestellt werden. Allerdings werden teilweise wichtige Details nicht erwähnt oder an manchen Stellen sogar falsch beschrieben, weshalb dieses Buch aus fachlicher Sicht zwar als eine gute Brücke von COBOL zu Java einzuschätzen ist, jedoch nicht als einzige Quelle dienen sollte. Den Brückenschlag in die andere Richtung -- von Java zu COBOL -- lässt es allerdings nicht ohne weiteres zu, was nicht zuletzt daran liegt, dass -- wie der Name bereits andeutet -- ein großer Teil des Buches grafischen Oberflächen mit \textit{Swing} gewidmet ist.

Im Gegensatz zu vorherigem steigt \citeWithTitle{byrne_java_2009-1} tiefer in die syntaktischen Konstrukte von Java ein und behandelt außerdem Java EE (Enterprise Edition). Außerdem zielt es weniger auf grafische Systeme ab, als auf Systeme die Daten verarbeiten. So werden verschiedene Ein- und Ausgabe-Mechanismen erklärt, der Umgang mit XML-Formaten beschrieben und -- als Teil der Java EE -- die Verwendung von Datenbanken erläutert.

\citeWithTitle{doke_cobol_2005} und \citeWithTitle{byrne_java_2009-1} ermöglichen also COBOL-Entwicklern den schnellen Einstieg in Java und bieten durch die jeweil unterschiedlichen Schwerpunkte einen guten Überblick. Nicht nur, dass in diesen beiden Quellen die Sicht -- Java lernen als COBOL-Entwickler -- eine andere als in dieser Arbeit ist, auch wird nur selten bzw. gar nicht auf die praktische Relevanz der beschriebenen Konstrukte eingegangen. Dadurch fehlt der Charakter eines Leitfadens, welcher diese Arbeit prägen soll. 