% Eigene Befehle und typographische Auszeichnungen für diese

% einfaches Wechseln der Schrift, z.B.: \changefont{cmss}{sbc}{n}
\newcommand{\changefont}[3]{\fontfamily{#1} \fontseries{#2} \fontshape{#3} \selectfont}

% Abkürzungen mit korrektem Leerraum 
\newcommand{\ua}{\mbox{u.\,a.\ }}
\newcommand{\zB}{\mbox{z.\,B.\ }}
\newcommand{\uU}{\mbox{u.\,U.\ }}
\newcommand{\uA}{\mbox{u.\,A.\ }}
\newcommand{\iA}{\mbox{i.\,A.\ }}
\newcommand{\dahe}{\mbox{d.\,h.\ }}
\newcommand{\Vgl}{Vgl.\ }
\newcommand{\ggf}{ggf.\ }
\newcommand{\bzw}{bzw.\ }
\newcommand{\evtl}{evtl.\ }
\newcommand{\szs}{szs.\ }
\newcommand{\sog}{sog.\ }
\newcommand{\engl}{engl.\ }
\newcommand{\lat}{lat.\ }

\newcommand{\abbildung}[1]{Abbildung~\ref{fig:#1}}

\newcommand{\bs}{$\backslash$}

% erzeugt ein Listenelement mit fetter Überschrift 
\newcommand{\itemd}[2]{\item{\textbf{#1}}\\{#2}}

% einige Befehle zum Zitieren --------------------------------------------------
%\newcommand{\Zitat}[2][\empty]{\ifthenelse{\equal{#1}{\empty}}{\citep{#2}}{\citep[#1]{#2}}}

% zum Ausgeben von Autoren
%\newcommand{\AutorName}[1]{\textsc{#1}}
%\newcommand{\Autor}[1]{\AutorName{\citeauthor{#1}}}

% verschiedene Befehle um Wörter semantisch auszuzeichnen ----------------------
\newcommand{\NeuerBegriff}[1]{\textbf{#1}}
\newcommand{\Fachbegriff}[1]{\textit{#1}}

\newcommand{\Eingabe}[1]{\texttt{#1}}
\newcommand{\Code}[1]{\texttt{#1}}
\newcommand{\Datei}[1]{\texttt{#1}}

\newcommand{\Datentyp}[1]{\textsf{#1}}
\newcommand{\XMLElement}[1]{\textsf{#1}}
\newcommand{\Webservice}[1]{\textsf{#1}}

\newcommand{\quotes}[1]{»#1«}

\newcommand{\citeWithTitle}[1]{\citetitle{#1} \cite{#1}}
%\newcommand{\citeVgl}[1]{\cite[vgl.][]{#1}}w

\newcommand{\till}[2]{#1 -- #2}

%Minted
\newcommand{\cobolNotFree}[1]{\inputminted[bgcolor=hellgrau,fontsize=\scriptsize]{cobol}{Code/#1}}
\newcommand{\cobol}[1]{\inputminted[bgcolor=hellgrau,fontsize=\scriptsize]{cobolfree}{Code/#1}}
\newcommand{\java}[1]{\inputminted[bgcolor=hellgrau,fontsize=\footnotesize]{java}{Code/#1}}
\newcommand{\mintedCaption}[2]{\begingroup\captionsetup{type=listing}\captionof{listing}{#1\label{#2}}\endgroup}
\newcommand{\mintedCobolWithOutput}[4]{\cobol{#1}#4\mintedCaption{#2}{#3}}
\newcommand{\mintedCobol}[3]{\cobol{#1}\mintedCaption{#2}{#3}}
\newcommand{\mintedJavaWithOutput}[4]{\java{#1}#4\mintedCaption{#2}{#3}}
\newcommand{\mintedJava}[3]{\java{#1}\mintedCaption{#2}{#3}}


% Interviews
\newcommand{\interviewExpert}[2]{\subsubsection*{#1}#2}
\newcommand{\toni}[1]{\subsubsection*{Antonio Grieco}\textit{#1}}
\newcommand{\jona}[1]{\interviewExpert{Jonathan Streit}{#1}}
\newcommand{\ivo}[1]{\interviewExpert{Ivaylo Bonev}{#1}}
\newcommand{\thomas}[1]{\interviewExpert{Thomas Lamperstorfer}{#1}}

% recap
\newcommand{\recap}[1]{
    \begin{wrapfigure}{l}{.05\textwidth}
        %\vspace{-20pt}
        \begin{center}
            \includegraphics[width=0.9\linewidth]{Bilder/exclamation_mark}
        \end{center}
        %\vspace{-20pt}
    \end{wrapfigure}
    #1
    %\noindent\parbox{\linewidth}{#1}
}