% Eigene Befehle und typographische Auszeichnungen für diese

% einfaches Wechseln der Schrift, \zB: \changefont{cmss}{sbc}{n}
\newcommand{\changefont}[3]{\fontfamily{#1} \fontseries{#2} \fontshape{#3} \selectfont}

% Abkürzungen mit korrektem Leerraum 
\newcommand{\bzw}{bzw.\ }
\newcommand{\dahe}{\mbox{d.\,h.\ }}
\newcommand{\engl}{engl.\ }
\newcommand{\evtl}{evtl.\ }
\newcommand{\ggf}{ggf.\ }
\newcommand{\iA}{\mbox{i.\,A.\ }}
\newcommand{\idR}{\mbox{i.\,d.\,R.\ }}
\newcommand{\lat}{lat.\ }
\newcommand{\sog}{sog.\ }
\newcommand{\szs}{szs.\ }
\newcommand{\ua}{\mbox{u.\,a.\ }}
\newcommand{\uA}{\mbox{u.\,A.\ }}
\newcommand{\uU}{\mbox{u.\,U.\ }}
\newcommand{\Vgl}{Vgl.\ }
\newcommand{\vgl}{vgl.\ }
\newcommand{\zB}{\mbox{z.\,B.\ }}

\newcommand{\abbildung}[1]{Abbildung~\ref{fig:#1}}

\newcommand{\bs}{$\backslash$}

% erzeugt ein Listenelement mit fetter Überschrift 
\newcommand{\itemd}[2]{\item{\textbf{#1}}\\{#2}}

% einige Befehle zum Zitieren --------------------------------------------------
%\newcommand{\Zitat}[2][\empty]{\ifthenelse{\equal{#1}{\empty}}{\citep{#2}}{\citep[#1]{#2}}}

% zum Ausgeben von Autoren
%\newcommand{\AutorName}[1]{\textsc{#1}}
%\newcommand{\Autor}[1]{\AutorName{\citeauthor{#1}}}

% verschiedene Befehle um Wörter semantisch auszuzeichnen ----------------------
\newcommand{\NeuerBegriff}[1]{\textbf{#1}}
\newcommand{\Fachbegriff}[1]{\textit{#1}}

\newcommand{\Eingabe}[1]{\texttt{#1}}
\newcommand{\Code}[1]{\texttt{#1}}
\newcommand{\Datei}[1]{\texttt{#1}}

\newcommand{\Datentyp}[1]{\textsf{#1}}
\newcommand{\XMLElement}[1]{\textsf{#1}}
\newcommand{\Webservice}[1]{\textsf{#1}}

\newcommand{\quotes}[1]{»#1«}

\newcommand{\citeWithTitle}[1]{\citetitle{#1} \cite{#1}}
\newcommand{\citeVgl}[1]{\cite[vgl.][]{#1}}

\newcommand{\till}[2]{#1 -- #2}

%Minted
\usepackage{xparse}
\NewDocumentEnvironment{codeWithCaption}{mm} % one m per argument
  {}
  {
    \vspace*{-3.5em}
    \begingroup\captionsetup{type=listing}\captionof{listing}{#1\label{#2}}\endgroup
  }

\newcommand{\cob}[1]{\mintinline[breaklines]{cobolfree}{#1}}
\newcommand{\jav}[1]{\mintinline[breaklines]{java}{#1}}

\newcommand{\inputCode}[3]{
    \inputminted[frame=lines,framesep=2mm,baselinestretch=1.2,linenos,
    bgcolor=hellgrau,fontsize=#1]{#2}{Code/#3}
}
\newcommand{\cobolNotFree}[1]{\inputCode{\scriptsize}{cobol}{#1}}
\newcommand{\cobol}[1]{\inputCode{\scriptsize}{cobolfree}{#1}}
\newcommand{\java}[1]{\inputCode{\footnotesize}{java}{#1}}

\newcommand{\mintedCobolNotfree}[3]{
    \begin{codeWithCaption}{#2}{#3}
    \cobolNotFree{#1}
    \end{codeWithCaption}
}

\newcommand{\mintedCobol}[3]{
    \begin{codeWithCaption}{#2}{#3}
    \cobol{#1}
    \end{codeWithCaption}
}

\newcommand{\mintedJava}[3]{
    \begin{codeWithCaption}{#2}{#3}
    \java{#1}
    \end{codeWithCaption}
}

%\newcommand{\mintedCaption}[2]{\begingroup\captionsetup{type=listing}\captionof{listing}{#1\mylabel{#2}}\endgroup}

% Interviews
\newcommand{\interviewExpert}[2]{\subsubsection*{#1}#2}
\newcommand{\toni}[1]{\subsubsection*{Antonio Grieco}\textit{#1}}
\newcommand{\jona}[1]{\interviewExpert{Jonathan Streit}{#1}}
\newcommand{\ivo}[1]{\interviewExpert{Ivaylo Bonev}{#1}}
\newcommand{\thomas}[1]{\interviewExpert{Thomas Lamperstorfer}{#1}}

% recap
\newcommand{\recappage}[2]{
    \begin{minipage}[c]{\linewidth}
        \begin{wrapfigure}{l}{.2\linewidth}
            \centering
            \vspace{-15pt}
            \includegraphics[width=.75\linewidth]{Bilder/recap}
            \textsc{\autoref{#1}}
            \vspace{-10pt}
        \end{wrapfigure}
        #2
    \end{minipage}
}

\newcommand{\recap}[2]{
    \setlength{\fboxsep}{1.5em}
    \colorbox{gray!25}{\parbox{\linewidth-2\fboxsep}{\recappage{#1}{#2}}}
}

\newcommand{\mytodo}[1]{}
%\newcommand{\mytodo}[1]{
%    \noindent\colorbox{blue!25}{\parbox{\linewidth-2\fboxsep}{#1}}
%}

\newcommand{\visitedOn}{(besucht am 19.05.2018)}