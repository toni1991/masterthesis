% Anpassung des Seitenlayouts --------------------------------------------------
%   siehe Seitenstil.tex
% ------------------------------------------------------------------------------
\usepackage[
    automark, % Kapitelangaben in Kopfzeile automatisch erstellen
    headsepline, % Trennlinie unter Kopfzeile
    ilines % Trennlinie linksbündig ausrichten
]{scrpage2}

% Anpassung an Landessprache ---------------------------------------------------
\usepackage[ngerman]{babel}

% Umlaute ----------------------------------------------------------------------
%   Umlaute/Sonderzeichen wie äüöß direkt im Quelltext verwenden (CodePage).
%   Erlaubt automatische Trennung von Worten mit Umlauten.
% ------------------------------------------------------------------------------
\usepackage[utf8]{inputenc}
\usepackage[T1]{fontenc}
\usepackage{textcomp} % Euro-Zeichen etc.

% Schrift ----------------------------------------------------------------------
\usepackage{lmodern} % bessere Fonts
\usepackage{relsize} % Schriftgröße relativ festlegen

% Grafiken ---------------------------------------------------------------------
% Einbinden von JPG-Grafiken ermöglichen
\usepackage[dvips,final]{graphicx}
% hier liegen die Bilder des Dokuments
\graphicspath{{Bilder/}}

% Befehle aus AMSTeX für mathematische Symbole z.B. \boldsymbol \mathbb --------
\usepackage{amsmath,amsfonts}

% für Index-Ausgabe mit \printindex --------------------------------------------
\usepackage{makeidx}

% Einfache Definition der Zeilenabstände und Seitenränder etc. -----------------
\usepackage{setspace}
\usepackage{geometry}

% Symbolverzeichnis ------------------------------------------------------------
%   Symbolverzeichnisse bequem erstellen. Beruht auf MakeIndex:
%     makeindex.exe %Name%.nlo -s nomencl.ist -o %Name%.nls
%   erzeugt dann das Verzeichnis. Dieser Befehl kann z.B. im TeXnicCenter
%   als Postprozessor eingetragen werden, damit er nicht ständig manuell
%   ausgeführt werden muss.
%   Die Definitionen sind ausgegliedert in die Datei "Glossar.tex".
% ------------------------------------------------------------------------------
% \usepackage[intoc]{nomencl}
% \let\abbrev\nomenclature
% \renewcommand{\nomname}{Abkürzungsverzeichnis}
% \setlength{\nomlabelwidth}{.25\hsize}
% \renewcommand{\nomlabel}[1]{#1 \dotfill}
% \setlength{\nomitemsep}{-\parsep}

% Abkürzungsverzeichnis
\usepackage[printonlyused]{acronym} %Abkürzungsverzeichnis erstellen, Langtext als Fußnote, Auflistung nur bei Verwendung

% zum Umfließen von Bildern ----------------------------------------------------
\usepackage{floatflt}
\usepackage{float}

% Abstand der float caption
\usepackage[skip=5pt]{caption} % example skip set to 2pt

% ensure floats do not go into the next section.
\usepackage[section]{placeins}

% Farben
\usepackage[dvipsnames]{xcolor}
\definecolor{hellgrau}{rgb}{0.93,0.93,0.93}
\definecolor{colKeys}{rgb}{0,0,0.9}
\definecolor{colIdentifier}{rgb}{0,0,0}
\definecolor{colComments}{rgb}{0.6,0,0}
\definecolor{colString}{rgb}{0,0.7,0}

% zum Einbinden von Programmcode -----------------------------------------------
\usepackage{listings}
\usepackage[newfloat]{minted}

\setminted{
    bgcolor=hellgrau,
    xleftmargin=20pt,
    linenos=true,
    fontsize=\footnotesize
}

\lstset{
    float=hbp,
    basicstyle=\ttfamily\color{black}\small\smaller,
    keywordstyle=\color{colKeys},
    stringstyle=\color{colString},
    commentstyle=\color{colComments},
    columns=flexible,
    tabsize=4,
    frame=single,
    extendedchars=true,
    showspaces=false,
    showstringspaces=false,
    numbers=left,
    numberstyle=\tiny,
    breaklines=true,
    backgroundcolor=\color{hellgrau},
    captionpos=b,
    breakautoindent=true
}

% URL verlinken, lange URLs umbrechen etc. -------------------------------------
\usepackage{url}
\makeatletter
\g@addto@macro{\UrlBreaks}{\UrlOrds}
\makeatother

% PDF-Optionen -----------------------------------------------------------------
\usepackage[
    bookmarks,
    bookmarksopen=true,
    colorlinks=true,
% diese Farbdefinitionen zeichnen Links im PDF farblich aus
    linkcolor=black, % einfache interne Verknüpfungen
    anchorcolor=black,% Ankertext
    citecolor=black, % Verweise auf Literaturverzeichniseinträge im Text
    filecolor=black, % Verknüpfungen, die lokale Dateien öffnen
    menucolor=black, % Acrobat-Menüpunkte
    urlcolor=black, 
% diese Farbdefinitionen sollten für den Druck verwendet werden (alles schwarz)
    %linkcolor=black, % einfache interne Verknüpfungen
    %anchorcolor=black, % Ankertext
    %citecolor=black, % Verweise auf Literaturverzeichniseinträge im Text
    %filecolor=black, % Verknüpfungen, die lokale Dateien öffnen
    %menucolor=black, % Acrobat-Menüpunkte
    %urlcolor=black,
    plainpages=false, % zur korrekten Erstellung der Bookmarks
    pdfpagelabels, % zur korrekten Erstellung der Bookmarks
    hypertexnames=false, % zur korrekten Erstellung der Bookmarks
    linktoc=all, % Seitenzahlen und Text im Inhaltsverzeichnis verlinken
]{hyperref}

% Befehle, die Umlaute ausgeben, führen zu Fehlern, wenn sie hyperref als Optionen übergeben werden
\hypersetup{
    pdftitle={\mytitel},
    pdfauthor={\myautor},
    pdfcreator={\myautor},
    pdfsubject={\mytitel},
    pdfkeywords={\mytitel \myart \myautor}
}

%% Roman pagenumbers good aligned in toc
%\usepackage{tocloft}
%\cftsetpnumwidth{2em}
%\setlength\cftafterfigskip{10pt}
%\renewcommand\cftchapafterpnum{\vskip10pt}
%\renewcommand\cftsecafterpnum{\vskip15pt}

% fortlaufendes Durchnummerieren der Fußnoten ----------------------------------
\usepackage{chngcntr}

% für lange Tabellen -----------------------------------------------------------
\usepackage{longtable}
\usepackage{array}
\usepackage{ragged2e}

% seiten rotieren
\usepackage{pdflscape}

% Spaltendefinition rechtsbündig mit definierter Breite ------------------------
\newcolumntype{w}[1]{>{\raggedleft\hspace{0pt}}p{#1}}

% Formatierung von Listen ändern -----------------------------------------------
\usepackage{paralist}

% bei der Definition eigener Befehle benötigt
\usepackage{ifthen}

% definiert u.a. die Befehle \todo und \listoftodos
\usepackage{todonotes}

% sorgt dafür, dass Leerzeichen hinter parameterlosen Makros nicht als Makroendezeichen interpretiert werden
\usepackage{xspace}

% einbinden anderer PDF Dateien (Deckblatt etc.)
\usepackage{pdfpages}

% Bibliographics
% Naturwissenschaftliche Bibliographien
%\usepackage[square, comma, numbers]{natbib}
% Stil der Zitate und der Bibliographie
%\bibliographystyle{natdin}
\usepackage[backend=biber]{biblatex}
\addbibresource{Sonstiges/Literatur.bib}

% Tikz libraries
\usetikzlibrary{arrows.meta}
\usetikzlibrary{intersections}

% Spalten
\usepackage{multicol}

% Anhang
\usepackage[toc,page]{appendix}

% SVG Biler
\usepackage{svg}

% euro sign
\usepackage{eurosym}

% subfigures
\usepackage{subfig}
\usepackage{floatflt}

% Quotes in babel language
\usepackage{csquotes}