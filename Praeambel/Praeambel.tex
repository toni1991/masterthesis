\documentclass[
    12pt, % Schriftgröße
    DIV=10,
    ngerman, % für Umlaute, Silbentrennung etc.
    a4paper, % Papierformat
    titlepage, % es wird eine Titelseite verwendet
    parskip=full, % Abstand zwischen Absätzen (halbe Zeile)
    headings=normal, % Größe der Überschriften verkleinern
    listof=totoc, % Verzeichnisse im Inhaltsverzeichnis aufführen
    listof=entryprefix,
    bibliography=totoc, % Literaturverzeichnis im Inhaltsverzeichnis aufführen
    index=totoc, % Index im Inhaltsverzeichnis aufführen
    captions=tableheading, % Beschriftung von Tabellen unterhalb ausgeben
    final % Status des Dokuments (final/draft)
]{scrreprt}

\KOMAoptions{twoside=false}

\listfiles

\makeatletter
\DeclareOldFontCommand{\rm}{\normalfont\rmfamily}{\mathrm}
\DeclareOldFontCommand{\sf}{\normalfont\sffamily}{\mathsf}
\DeclareOldFontCommand{\tt}{\normalfont\ttfamily}{\mathtt}
\DeclareOldFontCommand{\bf}{\normalfont\bfseries}{\mathbf}
\DeclareOldFontCommand{\it}{\normalfont\itshape}{\mathit}
\DeclareOldFontCommand{\sl}{\normalfont\slshape}{\@nomath\sl}
\DeclareOldFontCommand{\sc}{\normalfont\scshape}{\@nomath\sc}
\makeatother

% Weise compiler an, nicht bei Fehlern anzuhalten!
% \nonstopmode

% Meta-Informationen -----------------------------------------------------------
%   Informationen über das Dokument, wie \zB Titel, Autor, Matrikelnr. etc
%   werden in der Datei Meta.tex definiert und können danach global
%   verwendet werden.
% ------------------------------------------------------------------------------
% Meta-Informationen -----------------------------------------------------------
%   Definition von globalen Parametern, die im gesamten Dokument verwendet
%   werden k�nnen (z.B auf dem Deckblatt etc.).
%
%   ACHTUNG: Wenn die Texte Umlaute oder ein Esszet enthalten, muss der folgende
%            Befehl bereits an dieser Stelle aktiviert werden:
%            \usepackage[latin1]{inputenc}
% ------------------------------------------------------------------------------

\newcommand{\titel}{Mobile ereignisbasierte \\Anlagen�berwachung in der Industrie 4.0}
\newcommand{\art}{Bachelorarbeit}
\newcommand{\fachgebiet}{Informatik}
\newcommand{\autor}{Antonio Grieco}
\newcommand{\studienbereich}{Software-Engineering}
\newcommand{\matrikelnr}{930 190}
\newcommand{\erstgutachter}{Prof. Dr.-Ing. Thorsten Sch�ler}
\newcommand{\zweitgutachter}{Dipl.-Inf. Lukas Podolski}
\newcommand{\jahr}{2015}
\newcommand{\ort}{Augsburg}

% Befehle, die Umlaute ausgeben, f�hren zu Fehlern, wenn sie hyperref als Optionen �bergeben werden
\hypersetup{
    pdftitle={\titel},
    pdfauthor={\autor},
    pdfcreator={\autor},
    pdfsubject={\titel},
    pdfkeywords={\titel}
}

% benötigte Packages -----------------------------------------------------------
%   LaTeX-Packages, die benötigt werden, sind in die Datei Packages.tex
%   "ausgelagert", um diese Vorlage möglichst übersichtlich zu halten.
% ------------------------------------------------------------------------------
% Anpassung des Seitenlayouts --------------------------------------------------
%   siehe Seitenstil.tex
% ------------------------------------------------------------------------------
\usepackage[
    automark, % Kapitelangaben in Kopfzeile automatisch erstellen
    headsepline, % Trennlinie unter Kopfzeile
    ilines % Trennlinie linksbündig ausrichten
]{scrpage2}

% Anpassung an Landessprache ---------------------------------------------------
\usepackage[ngerman]{babel}

% Umlaute ----------------------------------------------------------------------
%   Umlaute/Sonderzeichen wie äüöß direkt im Quelltext verwenden (CodePage).
%   Erlaubt automatische Trennung von Worten mit Umlauten.
% ------------------------------------------------------------------------------
\usepackage[utf8]{inputenc}
\usepackage[T1]{fontenc}
\usepackage{textcomp} % Euro-Zeichen etc.

% Schrift ----------------------------------------------------------------------
\usepackage{lmodern} % bessere Fonts
\usepackage{relsize} % Schriftgröße relativ festlegen

% Grafiken ---------------------------------------------------------------------
% Einbinden von JPG-Grafiken ermöglichen
\usepackage[dvips,final]{graphicx}
% hier liegen die Bilder des Dokuments
\graphicspath{{Bilder/}}

% Befehle aus AMSTeX für mathematische Symbole z.B. \boldsymbol \mathbb --------
\usepackage{amsmath,amsfonts}

% für Index-Ausgabe mit \printindex --------------------------------------------
\usepackage{makeidx}

% Einfache Definition der Zeilenabstände und Seitenränder etc. -----------------
\usepackage{setspace}
\usepackage{geometry}

% Symbolverzeichnis ------------------------------------------------------------
%   Symbolverzeichnisse bequem erstellen. Beruht auf MakeIndex:
%     makeindex.exe %Name%.nlo -s nomencl.ist -o %Name%.nls
%   erzeugt dann das Verzeichnis. Dieser Befehl kann z.B. im TeXnicCenter
%   als Postprozessor eingetragen werden, damit er nicht ständig manuell
%   ausgeführt werden muss.
%   Die Definitionen sind ausgegliedert in die Datei "Glossar.tex".
% ------------------------------------------------------------------------------
% \usepackage[intoc]{nomencl}
% \let\abbrev\nomenclature
% \renewcommand{\nomname}{Abkürzungsverzeichnis}
% \setlength{\nomlabelwidth}{.25\hsize}
% \renewcommand{\nomlabel}[1]{#1 \dotfill}
% \setlength{\nomitemsep}{-\parsep}

% Abkürzungsverzeichnis
\usepackage[printonlyused]{acronym} %Abkürzungsverzeichnis erstellen, Langtext als Fußnote, Auflistung nur bei Verwendung

% zum Umfließen von Bildern ----------------------------------------------------
\usepackage{floatflt}
\usepackage{float}

% Abstand der float caption
\usepackage[skip=5pt]{caption} % example skip set to 2pt

% ensure floats do not go into the next section.
\usepackage[section]{placeins}

% Farben
\usepackage[dvipsnames]{xcolor}
\definecolor{hellgrau}{rgb}{0.93,0.93,0.93}
\definecolor{colKeys}{rgb}{0,0,0.9}
\definecolor{colIdentifier}{rgb}{0,0,0}
\definecolor{colComments}{rgb}{0.6,0,0}
\definecolor{colString}{rgb}{0,0.7,0}

% zum Einbinden von Programmcode -----------------------------------------------
\usepackage{listings}
\usepackage[newfloat]{minted}

\setminted{
    bgcolor=hellgrau,
    xleftmargin=20pt,
    linenos=true,
    fontsize=\footnotesize
}

\lstset{
    float=hbp,
    basicstyle=\ttfamily\color{black}\small\smaller,
    keywordstyle=\color{colKeys},
    stringstyle=\color{colString},
    commentstyle=\color{colComments},
    columns=flexible,
    tabsize=4,
    frame=single,
    extendedchars=true,
    showspaces=false,
    showstringspaces=false,
    numbers=left,
    numberstyle=\tiny,
    breaklines=true,
    backgroundcolor=\color{hellgrau},
    captionpos=b,
    breakautoindent=true
}

% URL verlinken, lange URLs umbrechen etc. -------------------------------------
\usepackage{url}
\makeatletter
\g@addto@macro{\UrlBreaks}{\UrlOrds}
\makeatother

% PDF-Optionen -----------------------------------------------------------------
\usepackage[
    bookmarks,
    bookmarksopen=true,
    colorlinks=true,
% diese Farbdefinitionen zeichnen Links im PDF farblich aus
    linkcolor=black, % einfache interne Verknüpfungen
    anchorcolor=black,% Ankertext
    citecolor=black, % Verweise auf Literaturverzeichniseinträge im Text
    filecolor=black, % Verknüpfungen, die lokale Dateien öffnen
    menucolor=black, % Acrobat-Menüpunkte
    urlcolor=black, 
% diese Farbdefinitionen sollten für den Druck verwendet werden (alles schwarz)
    %linkcolor=black, % einfache interne Verknüpfungen
    %anchorcolor=black, % Ankertext
    %citecolor=black, % Verweise auf Literaturverzeichniseinträge im Text
    %filecolor=black, % Verknüpfungen, die lokale Dateien öffnen
    %menucolor=black, % Acrobat-Menüpunkte
    %urlcolor=black,
    plainpages=false, % zur korrekten Erstellung der Bookmarks
    pdfpagelabels, % zur korrekten Erstellung der Bookmarks
    hypertexnames=false, % zur korrekten Erstellung der Bookmarks
    linktoc=all, % Seitenzahlen und Text im Inhaltsverzeichnis verlinken
]{hyperref}

% Befehle, die Umlaute ausgeben, führen zu Fehlern, wenn sie hyperref als Optionen übergeben werden
\hypersetup{
    pdftitle={\mytitel},
    pdfauthor={\myautor},
    pdfcreator={\myautor},
    pdfsubject={\mytitel},
    pdfkeywords={\mytitel \myart \myautor}
}

%% Roman pagenumbers good aligned in toc
%\usepackage{tocloft}
%\cftsetpnumwidth{2em}
%\setlength\cftafterfigskip{10pt}
%\renewcommand\cftchapafterpnum{\vskip10pt}
%\renewcommand\cftsecafterpnum{\vskip15pt}

% fortlaufendes Durchnummerieren der Fußnoten ----------------------------------
\usepackage{chngcntr}

% für lange Tabellen -----------------------------------------------------------
\usepackage{longtable}
\usepackage{array}
\usepackage{ragged2e}

% seiten rotieren
\usepackage{pdflscape}

% Spaltendefinition rechtsbündig mit definierter Breite ------------------------
\newcolumntype{w}[1]{>{\raggedleft\hspace{0pt}}p{#1}}

% Formatierung von Listen ändern -----------------------------------------------
\usepackage{paralist}

% bei der Definition eigener Befehle benötigt
\usepackage{ifthen}

% definiert u.a. die Befehle \todo und \listoftodos
\usepackage{todonotes}

% sorgt dafür, dass Leerzeichen hinter parameterlosen Makros nicht als Makroendezeichen interpretiert werden
\usepackage{xspace}

% einbinden anderer PDF Dateien (Deckblatt etc.)
\usepackage{pdfpages}

% Bibliographics
% Naturwissenschaftliche Bibliographien
%\usepackage[square, comma, numbers]{natbib}
% Stil der Zitate und der Bibliographie
%\bibliographystyle{natdin}
\usepackage[backend=biber]{biblatex}
\addbibresource{Sonstiges/Literatur.bib}

% Tikz libraries
\usetikzlibrary{arrows.meta}
\usetikzlibrary{intersections}

% Spalten
\usepackage{multicol}

% Anhang
\usepackage[toc,page]{appendix}

% SVG Biler
\usepackage{svg}

% euro sign
\usepackage{eurosym}

% subfigures
\usepackage{subfig}
\usepackage{floatflt}

% Quotes in babel language
\usepackage{csquotes}

% Kopf- und Fußzeilen, Seitenränder etc. ---------------------------------------
% Zeilenabstand 1,5 Zeilen -----------------------------------------------------
\onehalfspacing

% Seitenränder -----------------------------------------------------------------
\setlength{\topskip}{\ht\strutbox} % behebt Warnung von geometry
%\setlength{\bottomskip}{\ht\strutbox} % behebt Warnung von geometry
\geometry{paper=a4paper,left=30mm,right=25mm,top=25mm,bottom=25mm}

% Kopf- und Fußzeilen ----------------------------------------------------------
\pagestyle{scrheadings}
% Kopf- und Fußzeile auch auf Kapitelanfangsseiten
\renewcommand*{\chapterpagestyle}{scrheadings} 
% Schriftform der Kopfzeile
\renewcommand{\headfont}{\normalfont}
\renewcommand{\footfont}{\normalfont}

% Kopfzeile
%\ihead{\mytitel}
\chead{}
\ohead{\textit{\headmark}}
\setlength{\headheight}{1.1\baselineskip}
%\setlength{\headheight}{21mm} % Höhe der Kopfzeile
% \setheadwidth[0pt]{textwithmarginpar} % Kopfzeile über den Text hinaus verbreitern
\setheadsepline[text]{0.4pt} % Trennlinie unter Kopfzeile
% \setfootsepline[text]{0.4pt} % Trennlinie über Fußzeile

% Fußzeile
\ifoot{}
\cfoot{\pagemark}
\ofoot{}
\setlength{\footheight}{1.1\baselineskip}

% sonstige typographische Einstellungen ----------------------------------------

% erzeugt ein wenig mehr Platz hinter einem Punkt
\frenchspacing 

% Schusterjungen und Hurenkinder vermeiden
\clubpenalty = 10000
\widowpenalty = 10000 
\displaywidowpenalty = 10000

% Quellcode-Ausgabe formatieren
\lstset{numbers=left, numberstyle=\tiny, numbersep=5pt, breaklines=true}
\lstset{emph={square}, emphstyle=\color{red}, emph={[2]root,base}, emphstyle={[2]\color{blue}}}

% Fußnoten fortlaufend durchnummerieren
\counterwithout{footnote}{chapter}


% urls in bib
%%% --- The following two lines are what needs to be added --- %%%
\setcounter{biburllcpenalty}{7000}
\setcounter{biburlucpenalty}{8000}

% eigene Definitionen für Silbentrennung
\input{Praeambel/Silbentrennung}

% shell
\usepackage[minted]{tcolorbox}
\tcbuselibrary{skins}
\definecolor{topbar}{RGB}{220,220,220}
\definecolor{main}{RGB}{60,60,60}
\definecolor{quit}{RGB}{248,73,73}
\definecolor{min}{RGB}{252,182,37}
\definecolor{max}{RGB}{41,198,52}
\colorlet{offwhite}{white!96!black}
\newtcblisting{shellwindow}{%
  listing engine=minted, 
  minted language=text, 
  title={\strut}, 
  listing only, 
  enhanced, 
  colbacktitle=topbar,
  boxrule=0cm,
  left=2mm,
  width=\textwidth,
  frame hidden, 
  colback=main, 
  coltext=offwhite, 
  overlay={
    \fill [fill=quit] ([xshift=3mm]title.west) circle (1mm);
    \fill [fill=min] ([xshift=6mm]title.west) circle (1mm);
    \fill [fill=max] ([xshift=9mm]title.west) circle (1mm);
  }%
}

\newcommand{\sepCodeAndOutput}[0]{\vspace{-1cm}}

% eigene LaTeX-Befehle
% Eigene Befehle und typographische Auszeichnungen für diese

% einfaches Wechseln der Schrift, \zB: \changefont{cmss}{sbc}{n}
\newcommand{\changefont}[3]{\fontfamily{#1} \fontseries{#2} \fontshape{#3} \selectfont}

% Abkürzungen mit korrektem Leerraum 
\newcommand{\bzw}{bzw.\ }
\newcommand{\dahe}{\mbox{d.\,h.\ }}
\newcommand{\engl}{engl.\ }
\newcommand{\evtl}{evtl.\ }
\newcommand{\ggf}{ggf.\ }
\newcommand{\iA}{\mbox{i.\,A.\ }}
\newcommand{\idR}{\mbox{i.\,d.\,R.\ }}
\newcommand{\lat}{lat.\ }
\newcommand{\sog}{sog.\ }
\newcommand{\szs}{szs.\ }
\newcommand{\ua}{\mbox{u.\,a.\ }}
\newcommand{\uA}{\mbox{u.\,A.\ }}
\newcommand{\uU}{\mbox{u.\,U.\ }}
\newcommand{\Vgl}{Vgl.\ }
\newcommand{\zB}{\mbox{z.\,B.\ }}

\newcommand{\abbildung}[1]{Abbildung~\ref{fig:#1}}

\newcommand{\bs}{$\backslash$}

% erzeugt ein Listenelement mit fetter Überschrift 
\newcommand{\itemd}[2]{\item{\textbf{#1}}\\{#2}}

% einige Befehle zum Zitieren --------------------------------------------------
%\newcommand{\Zitat}[2][\empty]{\ifthenelse{\equal{#1}{\empty}}{\citep{#2}}{\citep[#1]{#2}}}

% zum Ausgeben von Autoren
%\newcommand{\AutorName}[1]{\textsc{#1}}
%\newcommand{\Autor}[1]{\AutorName{\citeauthor{#1}}}

% verschiedene Befehle um Wörter semantisch auszuzeichnen ----------------------
\newcommand{\NeuerBegriff}[1]{\textbf{#1}}
\newcommand{\Fachbegriff}[1]{\textit{#1}}

\newcommand{\Eingabe}[1]{\texttt{#1}}
\newcommand{\Code}[1]{\texttt{#1}}
\newcommand{\Datei}[1]{\texttt{#1}}

\newcommand{\Datentyp}[1]{\textsf{#1}}
\newcommand{\XMLElement}[1]{\textsf{#1}}
\newcommand{\Webservice}[1]{\textsf{#1}}

\newcommand{\quotes}[1]{»#1«}

\newcommand{\citeWithTitle}[1]{\citetitle{#1} \cite{#1}}
\newcommand{\citeVgl}[1]{\cite[vgl.][]{#1}}

\newcommand{\till}[2]{#1 -- #2}

%Minted
\newcommand{\cob}[1]{\mintinline[breaklines]{cobolfree}{#1}}
\newcommand{\jav}[1]{\mintinline[breaklines]{java}{#1}}
\newcommand{\cobolNotFree}[1]{\inputminted[bgcolor=hellgrau,fontsize=\scriptsize]{cobol}{Code/#1}}
\newcommand{\cobol}[1]{\inputminted[bgcolor=hellgrau,fontsize=\scriptsize]{cobolfree}{Code/#1}}
\newcommand{\java}[1]{\inputminted[bgcolor=hellgrau,fontsize=\footnotesize]{java}{Code/#1}}
\newcommand{\mintedCaption}[2]{\begingroup\captionsetup{type=listing}\captionof{listing}{#1\label{#2}}\endgroup}
\newcommand{\mintedCobolWithOutput}[4]{\cobol{#1}#4\mintedCaption{#2}{#3}}
\newcommand{\mintedCobol}[3]{\cobol{#1}\mintedCaption{#2}{#3}}
\newcommand{\mintedJavaWithOutput}[4]{\java{#1}#4\mintedCaption{#2}{#3}}
\newcommand{\mintedJava}[3]{\java{#1}\mintedCaption{#2}{#3}}

% Interviews
\newcommand{\interviewExpert}[2]{\subsubsection*{#1}#2}
\newcommand{\toni}[1]{\subsubsection*{Antonio Grieco}\textit{#1}}
\newcommand{\jona}[1]{\interviewExpert{Jonathan Streit}{#1}}
\newcommand{\ivo}[1]{\interviewExpert{Ivaylo Bonev}{#1}}
\newcommand{\thomas}[1]{\interviewExpert{Thomas Lamperstorfer}{#1}}

% recap
\newcommand{\recappage}[1]{
    \begin{minipage}[c]{\linewidth}
        \begin{wrapfigure}{l}{.1\linewidth}
            \vspace{-15pt}
            \includegraphics[width=\linewidth]{Bilder/recap}
            \vspace{-25pt}
        \end{wrapfigure}
        #1
    \end{minipage}
}

\newcommand{\recap}[1]{
    \setlength{\fboxsep}{1.5em}
    \colorbox{gray!25}{\parbox{\linewidth-2\fboxsep}{\recappage{#1}}}
}

% detect forward references
\newwrite\refs
\openout\refs=\jobname.refs
\makeatletter
\renewcommand\@setref[3]{%
        \ifx#1\relax
                \write\refs{'#3' \thepage\space undefined}%
                \protect \G@refundefinedtrue
                \nfss@text{\reset@font\bfseries ??}%
                \@latex@warning{Reference `#3' on page \thepage\space
                                undefined}%
                \PackageWarning{todonotes}{Undefined reference!}
        \else
                \write\refs{'#3' \thepage\space
                            \expandafter\@secondoftwo#1}%
                \PackageWarning{todonotes}{Check references: '#3' \thepage\space
                            \expandafter\@secondoftwo#1}
                \expandafter#2#1\null
        \fi
}
\makeatother

% sonstige Präambel
% Workaround für lstlistoflistings --------------------------------
\makeatletter
\@ifundefined{float@listhead}{}{%
    \renewcommand*{\lstlistoflistings}{%
        \begingroup
    	    \if@twocolumn
                \@restonecoltrue\onecolumn
            \else
                \@restonecolfalse
            \fi
            \float@listhead{\lstlistlistingname}%
            \setlength{\parskip}{\z@}%
            \setlength{\parindent}{\z@}%
            \setlength{\parfillskip}{\z@ \@plus 1fil}%
            \@starttoc{lol}%
            \if@restonecol\twocolumn\fi
        \endgroup
    }%
}
\makeatother

% Nummerierungstiefe
\setcounter{secnumdepth}{3}
\setcounter{tocdepth}{3}

% Farben
\definecolor{LightGray}{gray}{0.95}

\lstdefinelanguage{QML} 
{morekeywords={color,background,margin, visible, width, height, title, id, fill, text, anchors, centerIn, onClicked, host, onMessageReceived, Component.onCompleted, Component, onCompleted}, 
	emph={ApplicationWindow, Rectangle, Button, QmlQmqtt},
	sensitive=false, 
	morecomment=[l]{//}, 
	morecomment=[s]{/*}{*/},
	morestring=[b]", 
} 


\lstdefinelanguage{json}  
{
	emph={:, \,},
	sensitive=false, 
	morecomment=[l]{//}, 
	morecomment=[s]{/*}{*/},
	morestring=[b]", 
} 

\newcommand{\specialcell}[2][c]{%
	\begin{tabular}[#1]{@{}c@{}}#2\end{tabular}}

\titlehead{
Universität Augsburg \\ Fakultät für angewandte Informatik \\ Institute for Software \& Systems Engineering
}
\title{Programmiersprachliche Konzepte von COBOL im Vergleich mit Java -- Eine praxisorientierte Einführung}
\subject{Masterarbeit \\\normalsize{Informatik und Multimedia}}
\author{
    \Huge{Antonio Grieco} \\\\
        \small{Matrikelnummer: 1498410} \\ 
        \small{antonio.grieco@gmx.de} \\
        \small{Rosenaustraße 70} \\ 
        \small{86152 Augsburg}
} 
\publishers{
     \vfill
      Erstgutachter: Prof. Dr. Alexander  Knapp\\
      Zweitgutachter: Prof. Dr. Bernhard  Bauer\\
      Betreuer: Jonathan Streit
}
\date{\vfill \vfill \vfill \today}

% Belegung der notwendigen Kommandos f\"ur die Titelseite
\newcommand{\uni}{Universität Augsburg}
\newcommand{\fak}{Fakultät für angewandte Informatik}
\newcommand{\institut}{Institute for Software \& Systems Engineering}
\newcommand{\veranstaltung}{Masterarbeit} 
\newcommand{\autor}{Antonio Grieco}
\newcommand{\semester}{}
\newcommand{\datum}{\today}
\newcommand{\thema}{Programmiersprachliche Konzepte von COBOL im Vergleich mit Java -- Eine praxisorientierte Einführung}  	
\newcommand{\matrikelnr}{Matrikelnummer 1408410}			

\newcommand{\ownline}{\vspace{.7em}\hrule\vspace{.7em}} 				

\newcommand{\titelblatt}{
	\begin{titlepage}
	\begin{center}
		\textsc{\LARGE \uni}\\[0.5cm]
		\textsc{\fak}\\
		\textsc{\institut}\\[1.5cm]
		%\textsc{\lehrstuhl}\\[1.5cm]
		\includegraphics[width=4.7cm]{Bilder/uniasiegel}\\ [1.5cm]
		\textsc{\Large \veranstaltung}\\[0.3cm]
		\textsc{\Large von}\\[0.3cm]
		\textsc{\Large \autor}\\[0.3cm]
		\textsc{\large \matrikelnr}\\[0.5cm]
		\newcommand{\HRule}{\rule{\linewidth}{0.5mm}}
		
		\HRule \\[0.4cm]
		{\textbf{\Large{\thema}}}\\[0.4cm]
		\HRule \\[1.5cm]
		
		Erstgutachter: Prof. Dr. Alexander Knapp\\ [0.5cm]
      	Zweitgutachter: Prof. Dr. Bernhard Bauer\\ [0.5cm]
      	Betreuer: Prof. Dr. Alexander Knapp \& Jonathan Streit\\ [1.5cm]
		{\large{24. Mai 2018}}
	\end{center}
	\end{titlepage}
}

% muss als letztes geladen werden
\usepackage{scrhack}

%Markierung der overfull warnings im dokument
%\overfullrule=3mm