\section{Aufbau der Arbeit}

\begin{framed}
\begin{enumerate}[label=\arabic*.]
	\item[] \hfill \textbf{65 Seiten}
	\item[] \textbf{Zusammenfassung / Abstract}
    \item[] \textbf{Vorwort}
    \item[] \textbf{Eidesstattliche Erklärung}
	\item 
    	\textbf{Einleitung} \dotfill \textbf{5 Seiten}
    	\begin{enumerate}[label=\arabic*.]
    		\item \textbf{Problemstellung} \dotfill 2 Seiten
        	\item \textbf{Zielsetzung} \dotfill 2 Seiten
            \item \textbf{Aufbau der Arbeit} \dotfill 1 Seiten
    	\end{enumerate}
    \item \textbf{Vorhandene Literatur} \dotfill \textbf{5 Seiten}
    \item \textbf{Grundlagen der COBOL-Programmierung} \dotfill \textbf{30 Seiten}
    	\begin{enumerate}[label=\arabic*.]
    	\item \textbf{Aufbau eines COBOL-Programms} \dotfill 5 Seiten
        \item \textbf{Pattern} \dotfill 20 Seiten
        \item \textbf{Anti-Pattern} \dotfill 10 Seiten
    	\end{enumerate}
    \item \textbf{Umsetzung von Java-Entwurfsmustern in COBOL} \dotfill \textbf{15 Seiten}
    \item \textbf{Zusammenfassung und Ausblick} \dotfill \textbf{5 Seiten}
\end{enumerate}
\end{framed}