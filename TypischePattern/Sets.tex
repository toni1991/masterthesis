\subsection{Sets}
Neben den in \autoref{lists} beschriebenen Listen bieten Sets in der Programmierung eine weitere häufig genutzte Datenstruktur. Die zwei wesentlichen Unterschiede im Gegensatz zu Listen sind zum einen eine fehlende Ordnung der Elemente und die Eigenschaft, dass ein und das selbe Element nur genau einmal innerhalb eines Sets vorkommen darf. Das Set entspricht somit weitestgehend der mathematischen Definition einer Menge.

\subsubsection*{Sets in Java}
Wie für Listen bietet Java auch für Sets das \jav{Set}-Interface. Die wohl am häufigsten genutzte Implementierung dieses Interfaces stellt die \jav{HashSet}-Klasse dar, welche die \jav{hashCode}-Methode eines Objektes nutzt, um es pseudo-eindeutig identifizierbar zu machen.

\begin{codeWithCaption}{HashSet Beispiel in Java}{javaHashSet}
    \java{HashSetExample.java}
\begin{shellwindow}
0 1 2 3 4 5 6 7 8 9 
0 1 2 3 4 5 6 7 8 9 10 11 12 13 14 
\end{shellwindow}
\end{codeWithCaption}

In \autoref{javaHashSet} wird gezeigt, dass das Einfügen von Elementen, die bereits im Set enthalten sind, keine Auswirkung hat. Die Eigenschaft, dass ein Set ungeordnet ist, lässt sich leider nicht zeigen, da Java die Werte bei der gezeigten Ausgabe ordnet. Dieses Verhalten tritt auch auf, wenn die Werte in umgekehrter Reihenfolge dem Set hinzugefügt werden, was jedoch in keinem Fall bedeutet, dass es sich beim Set um eine stets sortierte Liste handelt, auch wenn es den Eindruck vermittelt.

\subsubsection*{Sets in COBOL}
Analog zu Listen kann festgehalten werden, dass eine Implementierung von Sets in COBOL nicht ohne weiteres möglich ist. Die Einschränkung der Größe der Datenstruktur bestünde auch hier. 

\begin{codeWithCaption}{Einfache Set Implementierung in COBOL}{cobolSet}
    \cobol{Set.cbl}
\begin{shellwindow}
 SIZE: 000
002, SIZE: 001
002,004, SIZE: 002
002,004, SIZE: 002
002, SIZE: 001
\end{shellwindow}
\end{codeWithCaption}

\autoref{cobolSet} greift jedoch beispielhaft die Kernaspekte von Sets auf und zeigt eine mögliche Implementierung in COBOL. Wie auch in \autoref{cobolSet} sind nur Einfüge- und Löschoperationen realisiert.