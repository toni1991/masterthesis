\subsection{Maps}
Maps erlauben den Zugriff auf bestimmte Elemente in konstanter Zeit, d.h. $O(n)$. Erreicht wird dies -- vereinfacht dargestellt -- dadurch, dass der Speicherbereich in dem ein Element der Datenstruktur gespeichert werden soll, mithilfe der Daten des Objekts berechnet wird. Das ganze nennt sich \textit{Hash-Funktion}. Dies eignet sich vorallem wenn viele Elemente in der Datenstruktur gespeichert werden sollen oder konstante Zugriffszeit auf jedes Element notwendig ist.

In Java steht dafür das Interface \mintinline{java}{Map} und Implementierungen wie \mintinline{java}{HashMap} bereit. Damit werden generische Objekte anhand ihrer \mintinline{java}{hashCode}-Funktion in einer Map verwaltet.

COBOL bietet auch hier keine Konstrukte um eine Map abzubilden. Wie die Experten Bonev und Streit jedoch angaben wurden in der Praxis bereits Möglichkeiten entwickelt ein solches Verhalten in COBOL abzubilden. Eine einfache wäre eine eigene Hash-Funktion für den genauen Anwendungsfall zu implementieren, die Eingabewerte eindeutig auf Indizes eines Arrays fester Größe abbildet. Eine weitere Möglichkeit.