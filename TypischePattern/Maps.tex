\subsection{Maps}
Maps erlauben schnellen Zugriff auf bestimmte Elemente. Erreicht wird dies oft durch eine sogenannte \textit{HashMap}, wobei auch andere Implementierungen wie Bäume denkbar sind. Bei diesen \textit{HashMaps} wird der Speicherbereich, in dem ein Element der Datenstruktur gespeichert werden, mithilfe der Daten des Objekts berechnet, was auch als \textit{gehasht} bezeichnet wird. Dies eignet sich vor allem, wenn viele Elemente in der Datenstruktur gespeichert werden und konstante Zugriffszeit -- $O(1)$ -- auf Elemente notwendig ist. Diese Zugriffszeit hängt von der zugrundeliegenden Implementierung ab.

In Java steht dafür das Interface \jav{Map} und Implementierungen wie \jav{HashMap} bereit. Damit werden generische Objekte anhand ihrer \jav{hashCode}-Funktion in einer Map verwaltet. 

COBOL bietet keine Konstrukte, um eine Map abzubilden. Wie die Experten Bonev und Streit angaben, wurden in der Praxis allerdings Möglichkeiten entwickelt, eine solche Datenstruktur in COBOL abzubilden, indem \zB eine eigene Hash-Funktion für den genauen Anwendungsfall implementiert wird, die die Eingabewerte eindeutig auf Indizes eines Arrays fester Größe abbildet. Eine weitere bereits beobachtete Variante ist das Nutzen von zwei Listen, wobei eine nach einem bestimmten Attribut der enthaltenen Daten sortiert wird und den Index des eigentlichen Objektes in der zweiten Liste beinhaltet, wodurch sich durch effiziente Indexsuche eine Map abbilden lässt.