\section{Redundanzen durch Wertekopien}
\autoref{variables} beschreibt Variablendeklarationen und -definitionen in Java und COBOL. Wie bereits geschildert sind Datentyp und Repräsentation in COBOL eng miteinander verwoben. Das sorgt in der Praxis häufig für redundanten Code, da nur zu Darstellungszwecken häufig Werte kopiert werden müssen. Dies hat nicht nur Auswirkungen auf die Lesbarkeit des Codes sondern auch auf die Performanz.

Laut Herrn Bonev ist beobachtbar, dass der dafür nötige \cob{MOVE}-Befehl ca. 80\% von COBOL-Code ausmacht, da häufig etliche Variablen für den selben Wert existieren und der Inhalt synchron gehalten werden muss. Auch beim Aufrufen von Unterprogrammen -- wie in \autoref{sec:functionsAndReturnValues} beschrieben -- wird dies, den Experten nach, häufig genutzt, um eine gewisse Sicherheit zu haben, dass Daten an anderer Stelle nicht verändert werden. Dazu werden Daten vor einem Aufruf in Variablen kopiert, das entsprechende Programm aufgerufen und anschließend wieder in die ursprünglichen Strukturen kopiert. 

Eine Möglichkeit, welche laut Herrn Streit in Betracht gezogen werden sollte, ist es, verschiedene Variablen als \textit{Union}-Verbunddatenstruktur, wie in \autoref{verbunddatenstrukturen} beschrieben, mithilfe des \cob{REDEFINES}-Schlüsselworts anzulegen und so massiv \cob{MOVE}-Befehle einzusparen. Diese Lösung bringt auch Sicherheit im Bezug auf Synchronisation, da das Kopieren nicht versehentlich vergessen werden kann. Häufig wird dies jedoch nicht eingesetzt, da sich Seiteneffekte und Fehler ergeben können, wenn Daten an unterschiedlichen Stellen modifiziert werden. Daher sollte dabei wie in \autoref{affix} beschrieben auf klare Prä- und Suffixe geachtet werden.

Seltener lässt sich die Nutzung von \cob{MOVE CORRESPONDING} beobachten. Dabei werden alle Untervariablen einer gegliederten Datenstruktur in eine andere Datenstruktur kopiert, die eine exakte namentliche Entsprechung in der Zielstruktur haben. Dies sei, den Experten nach, jedoch nicht weit verbreitet und sollte auch vermieden werden, da Programmlogik und -semantik abhängig von Variablennamen gemacht werden.