\section{Redundanzen durch Wertekopien}
\autoref{variables} beschreibt Variablendeklarationen und -definitionen in Java und COBOL. Wie bereits geschildert sind Datentyp und Repräsentation in COBOL eng miteinander verwoben. Das sorgt in der Praxis häufig für redundanten Code, da nur zu Darstellungszwecken häufig Werte kopiert werden müssen. Dies hat nicht nur Auswirkungen auf die Lesbarkeit des Codes sondern auch auf die Performanz.

Laut Herrn Bonev ist beobachtbar, dass der dafür nötige \cob{MOVE}-Befehl große Teile von COBOL-Code ausmacht, da häufig etliche Variablen für den selben Wert existieren und der Inhalt synchron gehalten werden muss, weil COBOL innerhalb eines Programms keine Referenzen auf Variablen oder Objekte bietet. Auch beim Aufrufen von Unterprogrammen -- wie in \autoref{sec:functionsAndReturnValues} beschrieben -- wird dies, den Experten nach, häufig genutzt, um eine gewisse Sicherheit zu haben, dass Daten an anderer Stelle nicht verändert werden oder um Daten in Strukturen zu packen, die vom aufgerufenen Programm verstanden werden. Dazu werden Daten vor einem Aufruf in Variablen kopiert, das entsprechende Programm aufgerufen und die Ergebnisse anschließend wieder in die ursprünglichen Strukturen kopiert.

Eine Möglichkeit, welche laut Herrn Streit in Betracht gezogen werden kann, ist es, verschiedene Variablen als \textit{Union}-Verbunddatenstruktur, wie in \autoref{verbunddatenstrukturen} beschrieben, mithilfe des \cob{REDEFINES}-Schlüsselworts anzulegen und so \cob{MOVE}-Befehle einzusparen. Diese Lösung bringt auch Sicherheit im Bezug auf Synchronisation, da das Kopieren nicht versehentlich vergessen werden kann. Häufig wird dies jedoch nicht eingesetzt, da sich Seiteneffekte und Fehler ergeben können, wenn Daten an unterschiedlichen Stellen modifiziert werden. Dabei sollte wie in \autoref{affix} beschrieben auf klare Affixe geachtet werden, was in der Folge dazu führt, dass mehr \cob{MOVE}-Befehle verwendet werden müssen, die nur schlecht vermeidbar sind. Während diese \cob{MOVE}-Befehle zur Übersichtlichkeit beitragen und daher in Kauf genommen werden können, sollten andere genau untersucht und wenn möglich entfernt werden.

Eine weitere Strategie \cob{MOVE}-Befehle einzusparen ist die Nutzung des \cob{CALL USING} Befehls wie in \autoref{sec:functionsAndReturnValues}. Laut Herrn Streit ist tendenziell selten zu beobachten, dass Entwickler an dieser Stelle die Variablen übergeben, die das Programm wirklich benötigt. Stattdessen werden wie in \autoref{copy} gezeigt Copybooks angelegt, die an mehreren Stellen genutzt werden, um eine Struktursicherheit zu haben, und nur diese einem Unterprogramm übergeben. Das macht das kopieren von und in ursprüngliche Strukturen notwendig. Hierbei wäre es allerdings möglich vorhandene Strukturen per Variablennamen zu übergeben, solange der Aufbau dem in den aufgerufenen Unterprogrammen entspricht.

Seltener lässt sich die Nutzung von \cob{MOVE CORRESPONDING} beobachten. Dabei werden alle Untervariablen einer gegliederten Datenstruktur in eine andere Datenstruktur kopiert, die eine exakte namentliche Entsprechung in der Zielstruktur haben. Dies sei, den Experten nach, jedoch nicht weit verbreitet und sollte vermieden werden, da Programmlogik und -semantik abhängig von Variablennamen gemacht werden.