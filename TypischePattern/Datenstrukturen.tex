\section{Komplexe Datenstrukturen}
\subsection{Listen}\label{lists}
Während \autoref{sec:felder} Felder behandelt, welche wie angesprochen eine feste Größe haben, bietet das Konzept von Listen deutliche Vorteile, wenn die Anzahl der Elemente variabel sein soll und zum Zeitpunkt des Erstellens nicht bekannt ist.
\subsubsection*{Listen in Java}
In Java kann das \jav{List}-Interface implementiert werden bzw. ein Objekt dieser Implementierung instanziiert werden, um eine Liste variabler Größe zu erhalten. Die wohl gebräuchlichste Implementierung dieses Interfaces stellt die Klasse \jav{ArrayList} dar. Intern hält diese -- wie der Name schon vermuten lässt -- ein Array, welches bei Bedarf in ein neues, größeres, Array kopiert wird. Die einfache Handhabung dieser Klasse wird in  \autoref{javaArrayList} dargestellt.
\java{ArrayListExample.java}
\begin{shellwindow}
Elements: [] - Size: 0
Elements: [0, 1, 2, 3, 4, 5, 6, 7, 8, 9] - Size: 10
Elements: [0, 1, 2, 3, 4, 6, 7, 8, 9] - Size: 9
Elements: [0, 1, 2, 3, 4, 6, 7, 8, 9, 2147483647] - Size: 10
\end{shellwindow}
\mintedCaption{ArrayList Beispiel in Java}{javaArrayList}
\subsubsection*{Listen in COBOL}
Eine exakte Abbildung von Listen ist in COBOL nicht möglich, da hier bereits zum Zeitpunkt des Kompilierens feststehen muss, wie groß ein Feld ist.
\cobol{List.cbl}
\begin{shellwindow}
 SIZE: 00
002, SIZE: 01
002,004, SIZE: 02
004, SIZE: 01
\end{shellwindow}
\mintedCaption{Einfache Listen Implementierung in COBOL}{cobolList}
\autoref{cobolList} zeigt jedoch beispielhaft eine einfache und unvollständige Implementierung einer Liste in COBOL. Hierbei sind lediglich Einfüge- und Löschoperationen realisiert. Zu beachten ist, dass aus bereits genannten Gründen auch diese Liste eine maximale Größe hat, die unter Umständen nicht ausreichend ist. Weitere Funktionalitäten der Liste müssten analog implementiert werden.%
\subsection{Sets}
Neben den in \autoref{lists} beschriebenen Listen bieten Sets in der Programmierung eine weitere häufig genutzte Datenstruktur. Die zwei wesentlichen Unterschiede im Gegensatz zu Listen sind zum einen eine fehlende Ordnung der Elemente und zum anderen die Eigenschaft, dass ein und das selbe Element nur genau einmal innerhalb eines Sets vorkommen darf. Das Set entspricht somit weitestgehend der mathematischen Definition einer Menge.

\subsubsection*{Sets in Java}
Wie für Listen bietet Java auch für Sets das \jav{Set}-Interface. Die wohl am häufigsten genutzte Implementierung dieses Interfaces stellt die \jav{HashSet}-Klasse dar, welche die \jav{hashCode}-Methode eines Objektes nutzt, um es pseudo-eindeutig identifizierbar zu machen. Dadurch wird vermieden, dass die gesamte Liste durchlaufen werden muss, um Duplikate zu finden.

In \autoref{javaHashSet} wird gezeigt, dass das Einfügen von Elementen, die bereits im Set enthalten sind, keine Auswirkung hat. Die Eigenschaft, dass ein Set ungeordnet ist, lässt sich leider nicht zeigen, da Java die Werte bei der gezeigten Ausgabe ordnet. Dieses Verhalten tritt auch auf, wenn die Werte in umgekehrter Reihenfolge dem Set hinzugefügt werden, was jedoch in keinem Fall bedeutet, dass es sich beim Set um eine stets sortierte Liste handelt, auch wenn es den Eindruck vermittelt.

\begin{codeWithCaption}{HashSet Beispiel in Java}{javaHashSet}
    \java{HashSetExample.java} \cFollow
\begin{shellwindow}
0 1 2 3 4 5 6 7 8 9 
0 1 2 3 4 5 6 7 8 9 10 11 12 13 14 
\end{shellwindow}
\end{codeWithCaption}

\subsubsection*{Sets in COBOL}
Analog zu Listen kann festgehalten werden, dass eine Implementierung von Sets in COBOL nicht ohne weiteres möglich ist. Die Einschränkung der Größe der Datenstruktur bestünde auch hier. 

\clearpage

\begin{codeWithCaption}{Einfache Set Implementierung in COBOL}{cobolSet}
    \cobol{Set.cbl} \cFollow
\begin{shellwindow}
SIZE: 000
002,SIZE: 001
002,004,SIZE: 002
002,004,SIZE: 002
002,SIZE: 001
\end{shellwindow}
\end{codeWithCaption}

\autoref{cobolSet} greift jedoch die Kernaspekte von Sets auf und zeigt eine mögliche Implementierung in COBOL. Wie auch in \autoref{cobolSet} sind nur Einfüge- und Löschoperationen realisiert. 

Wenngleich diese Implementierung beispielhaft zeigt, dass die Grundidee eines Sets auch in COBOL abbildbar ist, so hat sie gegenüber Java doch einige Nachteile. Zum einen ist sie ineffizient, da -- anders als \zB bei \jav{HashSet}s -- stets alle Werte der Liste durchlaufen werden müssen, um Redundanz zu vermeiden. Zum anderen liegt die große Stärke von Datenstrukturen in Java in der Generizität (siehe \autoref{generics}). Diese kann in COBOL nicht erreicht werden, wodurch ein Set stets neu für die zugrundeliegenden Datentypen implementiert werden muss.%
\subsection{Maps}
Maps erlauben den Zugriff auf bestimmte Elemente in konstanter Zeit, d.h. $O(n)$. Erreicht wird dies -- vereinfacht dargestellt -- dadurch, dass der Speicherbereich in dem ein Element der Datenstruktur gespeichert werden soll, mithilfe der Daten des Objekts berechnet wird. Das ganze nennt sich \textit{Hash-Funktion}. Dies eignet sich vorallem wenn viele Elemente in der Datenstruktur gespeichert werden sollen oder konstante Zugriffszeit auf jedes Element notwendig ist.

In Java steht dafür das Interface \jav{Map} und Implementierungen wie \jav{HashMap} bereit. Damit werden generische Objekte anhand ihrer \jav{hashCode}-Funktion in einer Map verwaltet.

COBOL bietet auch hier keine Konstrukte um eine Map abzubilden. Wie die Experten Bonev und Streit jedoch angaben wurden in der Praxis bereits Möglichkeiten entwickelt ein solches Verhalten in COBOL abzubilden. Eine einfache wäre eine eigene Hash-Funktion für den genauen Anwendungsfall zu implementieren, die Eingabewerte eindeutig auf Indizes eines Arrays fester Größe abbildet. Eine weitere Möglichkeit.
\subsection{Verbunddatenstrukturen}\label{verbunddatenstrukturen}
\textit{Structs} und \textit{Unions} sind weitere in der Programmierung verbreitete Datenstrukturen. Beide Typen bilden sogenannte Verbunddatenstrukturen, \dahe sie bieten eine Art Kapselung von mehreren Variablen, welche unterschiedlichen Typs sein können. Structs bezeichnen dabei einen Speicherbereich, der aus mehreren Variablen besteht, die so gruppiert gespeichert werden. Unions hingegen bezeichnen einen Speicherbereich, der auf verschiedene Weisen -- in unterschiedlichen Datentypen -- interpretiert werden kann. Sie gruppieren mehrere Variablen von denen stets nur eine Gültigkeit aufweisen kann.

\subsubsection*{Structs und Unions in COBOL}

Das fundamentale Konzept von Variablendeklarationen in COBOL ist -- wie bereits in \autoref{variables} angesprochen -- die Nutzung von Stufennummern und eine Untergliederung mithilfe dieser Stufennummern. Das sorgt dafür, dass jede untergliederte Variable in COBOL eine Verbunddatenstruktur darstellt.

\begin{codeWithCaption}{Structs und Unions in COBOL}{structUnionCOBOL}
\cobol{UNION_PROGRAM.cbl}
\cobol{UNION_SUB_PROGRAM.cbl}
\begin{shellwindow}
$ cobc -x -o UNION-PROGRAM UNION-PROGRAM.cbl 
$ cobc -m UNION-SUB-PROGRAM.cbl
$ ./UNION-PROGRAM
INTEGER: 190742
DECIMAL: 1907.42
UNION-SUB-PROGRAM: 190.742
\end{shellwindow}
\end{codeWithCaption}

\autoref{structUnionCOBOL} zeigt sowohl die Verwendung als Struktur-Typ, als auch den Zugriff als Union-Typ in einem Unterprogramm und mithilfe der in \autoref{variables} beschriebenen \cob{REDEFINES} Anweisung. Wie bereits in dieser Arbeit gezeigt, wird -- in diesem Falle zur Ausgabe -- auf eine gesamte Datenstruktur zugegriffen, die mehrere unterschiedliche Variablen enthält. Diese Datenstruktur stellt also ein \textit{Struct} dar. Im gezeigten Unterprogramm hingegen wird der selbe Speicherbereich anders interpretiert

Dies wird laut Herrn Streit in der Praxis auch oft ausgenutzt, um Erweiterungen von Programmen zu realisieren. Da Unterprogramme lediglich einen Zeiger auf Datenstrukturen übergeben bekommen, können diese Strukturen erweitert werden, ohne das Unterprogramm zu ändern, solange neue Datenfeldern an das Ende angehängt werden. Weitere Unterprogramme, welche an neue Datenfelder angepasst wurden, können wiederum die Daten, die hinten angehängt wurden, nutzen. Das Einfügen von Feldern zwischen bestehenden, hat unweigerlich die Anpassung aller Komponenten zur Folge.

\subsubsection*{Structs und Unions in Java}

In Java lassen sich Verbunddatenstrukturen auf verschiedene Weisen realisieren. \textit{Structs} stellen eine einfache Form von Klassen dar, die, wie \autoref{javaBean} zeigt, unterschiedliche Attribute kapseln. Diese werden auch \textit{Java Beans} genannt.

\mintedJava{Bean.java}{Java Bean}{javaBean}

\todo{\textit{Unions} können über \jav{enum}s realisiert werden.}

\recap{
    Die vorgestellten Datenstrukturen bieten eine Möglichkeit, je nach Anwendungsfall, Daten effizient und dynamisch zu speichern und bereitzustellen. Während Java Listen, Sets und Maps bereits im Sprachstandard unterstützt, mässen diese von Entwicklern in COBOL selbst implementiert werden, was selbstverständlich Wissen über die Algorithmik voraussetzt. Herr Streit betonte, dass etwaige Datenstrukturen manchmal in bestehenden COBOL-Programmen zu finden sind, jedoch -- durch das Fehlen des entsprechenden Vokabulars in der Entstehungszeit -- oft andere Namen tragen und daher als solches nicht zu erkennen sind. 

    COBOL bietet Verbunddatenstrukturen die vielfältig genutzt, aber auch ausgenutzt werden können, als fundamentalen Teil des Variablenkonzepts. Bei der Nutzung ist darauf zu achten, dass 
}