\subsection{Verbunddatenstrukturen}\label{verbunddatenstrukturen}
\textit{Structs} und \textit{Unions} sind weitere in der Programmierung verbreitete Datenstrukturen. Beide Typen bilden sogenannte Verbunddatenstrukturen, \dahe sie bieten eine Art Kapselung von mehreren Variablen, welche unterschiedlichen Typs sein können. Structs bezeichnen dabei einen Speicherbereich, der aus mehreren Variablen besteht, die so gruppiert gespeichert werden. Unions hingegen bezeichnen einen Speicherbereich, der auf verschiedene Weisen -- in unterschiedlichen Datentypen -- interpretiert werden kann. Sie gruppieren mehrere Variablen, von denen stets nur eine Gültigkeit aufweisen kann.

\subsubsection*{Structs und Unions in COBOL}

Das fundamentale Konzept von Variablendeklarationen in COBOL ist -- wie bereits in \autoref{variables} angesprochen -- die Nutzung von Stufennummern und eine Untergliederung mithilfe dieser Stufennummern. Das sorgt dafür, dass jede untergliederte Variable in COBOL eine Verbunddatenstruktur darstellt.

\begin{codeWithCaption}{Structs und Unions in COBOL}{structUnionCOBOL}
\cobol{UNION_PROGRAM.cbl}
\cobol{UNION_SUB_PROGRAM.cbl}
\begin{shellwindow}
$ cobc -x -o UNION-PROGRAM UNION-PROGRAM.cbl 
$ cobc -m UNION-SUB-PROGRAM.cbl
$ ./UNION-PROGRAM
INTEGER: 190742
DECIMAL: 1907.42
UNION-SUB-PROGRAM: 190.742
\end{shellwindow}
\end{codeWithCaption}

\autoref{structUnionCOBOL} zeigt sowohl die Verwendung als Struktur-Typ, als auch den Zugriff als Union-Typ in einem Unterprogramm und mithilfe der in \autoref{variables} beschriebenen \cob{REDEFINES} Anweisung. Wie bereits in dieser Arbeit gezeigt, wird -- in diesem Falle zur Ausgabe -- auf eine gesamte Datenstruktur zugegriffen, die mehrere unterschiedliche Variablen enthält. Diese Datenstruktur stellt ein Struct dar. Im gezeigten Unterprogramm hingegen wird der selbe Speicherbereich anders interpretiert.

Dies wird laut Herrn Streit in der Praxis auch oft ausgenutzt, um Erweiterungen von Programmen zu realisieren. Da Unterprogramme lediglich einen Zeiger auf Datenstrukturen übergeben bekommen, können diese Strukturen erweitert werden, ohne das Unterprogramm zu ändern, solange neue Datenfeldern an das Ende angehängt werden. Weitere Unterprogramme, welche an neue Datenfelder angepasst wurden, können wiederum die Daten, die hinten angehängt wurden, nutzen. Das Einfügen von Feldern zwischen bestehenden, hat unweigerlich die Anpassung aller Komponenten zur Folge.

\subsubsection*{Structs und Unions in Java}

In Java gehen Verbunddatenstrukturen in Klassen und Aufzählungstypen (jav{enum}) auf. Klassen stellen weiterentwickelte Formen von Structs dar, die, wie \autoref{javaBean} zeigt, unterschiedliche Attribute kapseln. Diese Klassen werden auch \textit{Java Beans} genannt. 

\mintedJava{Bean.java}{Struct in Java (JavaBean)}{javaBean}

Unions dagegen lassen sich durch verschiedene getter-Methoden abbilden. Dabei bietet eine Klasse mehere Methoden, die sich auf das selbe Feld zurückführen lassen, aber dieses -- wie \autoref{javaUnion} zeigt -- in verschiedenen Datentypen zurückliefern oder Felder verbinden. 

\mintedJava{Union.java}{Union in Java}{javaUnion}